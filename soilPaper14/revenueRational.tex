\section{Rationale behind the revenue strategy \label{app:revenueRational}}    

This section explains the rationale in a bit more detail, why the allocation
proportional to the revenue instead of into single best revenue is optimal
from a community perspective.

For a single microbe it would be optimal to maximise growth by investing all
resources in that enzyme that maximises the return per investment for the
currently limiting element. Hence, it should allocate all resources to the
enzyme type yielding the maximum revenue.
However, if many microbes compete for the same best substrate, they also have to
share the return of the extracellular decomposition process. If another microbe
targets the second-best substrate at a different location by producing a
different set of enzymes, it has an advantage of first accessing the
returns before those diffuse to the majority of microbes located at the
substrate with the hightest revenue.
When taking this competition into account, it makes sense to allocate the most
resources for the best revenue but also some resources to the other
possibilities. Hence, the revenue strategy allocates resources proportional to
their revenue. Note, however, that this arguments assumes a DOM pool that is
not completely mixed, whereas SEAM employs the simplifying
assuption of a single common DOM pool.

Another argument draws from a similarity to the restriction of risk in financial
investments. It is reasonable to invest most into the best revenues, but it is
dangerous to invest solely in a single alternative. If microbial community
expressed only one type of enzyme, resources might not be sufficient to
newly produce the other enzyme if the best resource becomes
unavailable, e.g. with changing pore connections with changing soil moisture.





