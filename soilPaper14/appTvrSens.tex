\section{Sensitivity to microbial turnover mineralization \label{app:TvrSens}}

The importance of N mineralization of microbial turnover, which is caused mainly
by predators that graze on microbes \citep{Clarholm85, Raynaud06}, was one of
the hypotheses in the development of the SEAM.
This section discusses SEAM's sensitivity to parameterization of microbial
turnover mineralization.

To this end we performed the \chem{CO_2}-Fertilization experiment using the
revenue strategy again with varying parameter $\epsilon_{\operatorname{tvr}}$,
the part of microbial turnover that is not mineralized. We also adjusted
microbial anabolic efficiency $\epsilon$ by the same but inverse factor so that
simulation results start from similar steady state of SOM stocks,
which change with model parameterization.

\begin{figure}[t] \vspace*{2mm}
\begin{center}
\includegraphics[width=8.3cm]{fig/CO2IncreaseSens}
\end{center}
\caption{
C-Stocks in the \chem{CO_2}-Fertilization experiment with varying
mineralization of microbial turnover ($1-\epsilon_{\operatorname{tvr}}$): The
patterns are similar, unless the system is shifted to another limitation regime.
\label{fig:CO2IncreaseSens}}

\end{figure}
\begin{figure}[t] \vspace*{2mm}
\begin{center}
\includegraphics[width=8.3cm]{fig/CO2IncreaseImbSens} 
\end{center}
\caption{
N Mineralization in the \chem{CO_2}-Fertilization experiment:
Mineralization of microbial turnover contributed most of the liberation of SOM-N
with the Revenue strategy during microbial N limitation.
After the end of the fertilisation at year 60, microbes with the Revenue
strategy continued to more strongly immobilize N (negative flux $\Phi_B$).
\label{fig:CO2IncreaseImbSens}}
\end{figure}

The change of the residue pool during the period of increased C inputs was very
similar across different parameterizations as long as the system followed the
same switches between several limitation states (Fig.
\ref{fig:CO2IncreaseSens}).
Contrary, if the re-parameterization shifted the system to different limitation
states then the dynamics changed qualitatively.
For example with a value of $\epsilon_{\operatorname{tvr}}=0.34$, there was an
initial net N mineralization instead of N immobilization, i.e. positive $\Phi_B$
(Fig.
\ref{fig:CO2IncreaseImbSens}). In the case of an initially large difference
between $\Phi_B$ and the maximum immobilization flux, the change in amount
and stoichiometry of litter inputs did not drive the system into microbial N
limitation ($-\Phi_B < u_{immo,Pot}$). This case resulted in the absense of the
simulated decrease of the residue pool (Fig \ref{fig:CO2IncreaseSens}).
The high initial $\Phi_B$ values resulted from the requirement that with the
long term steady state, the decomposer system must balance its organic litter N
inputs by N mineralization. The required increase in litter C/N ratio that could
shift a system simulated without turnover mineralization to N limitation was
unreasonably large. 

Hence, including the process of mineralization of microbial turnover is crucial
to SEAM for simulating a reasonable dynamics for shifts between C and N
limitation.
Although the SEAM is not sensitive to the exact specification in turnover
parameters if other parameters are recalibrated, there are thresholds than
can drive the model to different stoichiometric limitations and can lead to
substantial changes in model dynamics.

% The simulation of N mining in the \chem{CO_2}-Fertilization experiment was
% simulated based on a shift from C limitation or substrate N limitation to
% microbial N limitation and successive associated shift in enzyme allocation.
 

