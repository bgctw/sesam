In order to understand the coupling of carbon (C) and nitrogen (N) cycles, it is
necessary to understand C and N-use efficiencies of microbial soil organic
matter (SOM) decomposition. While important controls of those efficiencies by
microbial community adaptations have been shown at the scale of a soil pore, an
abstract simplified representation of community adaptations is needed at
ecosystem scale.

Therefore we developed the soil enzyme allocation model (SEAM), which takes a
holistic, partly optimality based approach to describe C and N dynamics at the
spatial scale of an ecosystem and time-scales of years and longer. We explicitly
modelled community adaptation strategies of resource allocation to extracellular
enzymes and enzyme limitations on SOM decomposition. Using SEAM, we explored
whether alternative strategy-hypotheses can have strong effects on SOM and
inorganic N cycling.

Results from prototypical simulations and a calibration to observations of an
intensive pasture site showed that the so-called revenue enzyme allocation
strategy was most viable. This strategy accounts for microbial adaptations to
both, stoichiometry and amount of different SOM resources, and supported the
largest microbial biomass under a wide range of conditions. Predictions of the
SEAM model were qualitatively similar to models explicitly representing
competing microbial groups. With adaptive enzyme allocation under conditions of
high C/N ratio of litter inputs, N in formerly locked in slowly degrading SOM
pools was made accessible, whereas with high N inputs, N was sequestered in
SOM and protected from leaching.

The findings imply that it is important for ecosystem scale models to account
for adaptation of C and N use efficiencies in order to represent C-N couplings.
The combination of stoichiometry and optimality principles is a promising route
to yield simple formulations of such adaptations at community level suitable for
incorporation into land surface models.
