In order to understand the coupling of global (C) and nitrogen (N) cycles, it is
necessary to understand C and N use efficiencies of soil microbial soil organic
matter (SOM) decomposers. While important controls of those efficiencies by
microbial community adaptations have been shown at pore scale, an abstract
simplified representation of community adaptations is needed at ecosystem scale.
Therefore I developed the SEAM model at soil core scale and modeled several
alternative community adaptation strategies of resource allocation to
extracellular enzymes. Using SEAM, I explored the effects of those alternative
strategy-hypotheses on SOM and inorganic N cycling. Simulations showed that the
revenue enzyme allocation strategy was most competitive, which accounted for
adaptations to both, stoichiometry and amount of different SOM resources.
Predictions of the holistic SEAM model were qualitatively similar to microbial
group explicit models with the ability to represent priming effects, bank
mechanism, and a continuous SOM sequestration with high inorganic N inputs. They
showed that soil enzyme allocation strategies strongly affect long term soil
organic matter cycling and nutrient recycling.
The findings imply that ecosystem scale models should account for adaptation of
C and N use efficiencies to represent and understand C-N couplings. The
combination of stoichiometry and optimality principles is a promising route to
yield simple formulations of such adaptations at community level suitable for
incorporation into land surface models.
