%\section{\\ \\ \hspace*{-7mm}  SEAM equations}
\section{\hspace*{-7mm}  SEAM equations \label{app:SEAM}
}    
%% Appendix A

\subsection{Carbon fluxes}
\begin{subequations}
\label{eq:SEAM}
\begin{align}
\frac{dB}{dt} &= \operatorname{syn}_B - \operatorname{tvr}_B \\
\frac{dE_L}{dt} &= (1 - \alpha)  \, \operatorname{syn}_E -\operatorname{tvr}_{EL}\\
\frac{dE_R}{dt} &= \alpha \, \operatorname{syn}_E - \operatorname{tvr}_{ER} \\
\frac{dL}{dt} &=  - \operatorname{dec}_L + \operatorname{i}_L
\\
\frac{dR}{dt} &=  - \operatorname{dec}_R +
\epsilon_{\operatorname{tvr}}  \operatorname{tvr}_B + (1 -k_{NB})
(\operatorname{tvr}_{ER} + \operatorname{tvr}_{EL})
\text{,} 
\end{align}
\end{subequations}
where $k_{NB}$ is the fraction of enzyme turnover that is transferred to the
DOM instead of the $R$ pool, and $\operatorname{i}_L$ is the carbon input
to the system. $\alpha$ is the proportion of total investment into enzymes
that is allocated to the residue pool $R$ (section \ref{sec:AllocStrategies}).
The specific fluxes are detailed below.

Total enzyme production $\operatorname{syn}_E$, maintenance respiration
$\operatorname{r}_{M}$, and microbial turnover $\operatorname{tvr}_B$ are
modelled as a first order kinetics of biomass:
\begin{subequations}
\begin{align}
\label{eq:synE} \operatorname{syn}_E &= a_E B \\
\label{eq:rM} \operatorname{r}_{M} &= m B \\
\label{eq:tvrB} \operatorname{tvr}_B &= \tau B 
\end{align}
\end{subequations}
% \begin{equation}
% \end{equation}

Enzyme turnover ($\operatorname{tvr}_{ER}$ and $\operatorname{tvr}_{EL}$) is
modelled as first order kinetics of enzyme levels.
\begin{equation}
\label{eq:tvrE}
\operatorname{tvr}_{ES} = k_{NS} E_S \text{,}
\end{equation}
where $S$ represents either labile $L$ or residue $R$ substrate pools.  

% \begin{subequations}
% \begin{align}
% \operatorname{tvr}_{ER} &= k_{NR} E_R \\
% \operatorname{tvr}_{EL} &= k_{NL} E_L
% \end{align}
% \end{subequations}
% 
Substrate depolymerization is modelled first order to substrate
availability an with a saturating Michaelis-Menten kinetics to enzyme levels:
\begin{subequations}
\label{eq:dec}
\begin{align}
\operatorname{dec}_{S,Pot} &= k_S S
\\ 
\operatorname{dec}_S &= dec_{S,Pot} \, \frac{E_S}{k_{m,S} + E_S}
\end{align}
\end{subequations}
 
The DOM pool is assumed to be in fast quasi steady state and all the sum of all
influxes to the DOM pool (decomposition + part of the enzyme turnover) is taken
up by microbial community.
\begin{equation}
\label{eq:uC}
u_C = \operatorname{dec}_L + \operatorname{dec}_R +  k_{NB}
(\operatorname{tvr}_{ER} + \operatorname{tvr}_{EL})
\end{equation}

Under carbon limitation, carbon available for synthesis of new biomass and
associated catabolic growth respiration, $C_{\operatorname{synBC}}$, is the
difference between carbon uptake and expenses for enzyme synthesis (eq.
\ref{eq:synE}) and maintenance respiration (eq.
\ref{eq:rM}).

\begin{equation}
\label{eq:synBC} 
C_{\operatorname{synBC}} = u_C -
\operatorname{syn}_E/\epsilon - \operatorname{r}_{M}
\end{equation}

If carbon balance for biomass synthesis, $\operatorname{syn}_B$ (eq.
\ref{eq:NCLim}), is positive, only a fraction $\epsilon$, the anabolic
carbon use efficiency, is used for synthesis of biomass and enzymes
and the rest is required for catabolic growth respiration $r_G$ to support this
synthesis.
The model assumes that requirements for enzyme synthesis and maintenance must be
met. Hence, the balance could become negative where microbial biomass starves
and declines.

\begin{subequations}
\label{eq:synB}
\begin{align}
\operatorname{syn}_B &= \begin{cases}
  \epsilon \, C_{\operatorname{synB}},  & \text{if} C_{\operatorname{synB}} >
  0\\
  C_{\operatorname{synB}}, & \text{otherwise}
\end{cases} \\  
\operatorname{r}_G &= \begin{cases}
  (1 - \epsilon) \, C_{\operatorname{synB}},  & \text{if}
  C_{\operatorname{synB}} > 0\\
  0, & \text{otherwise ,}
\end{cases}  
\end{align}
\end{subequations}
%\noident
where $C_{\operatorname{synB}}$ is the carbon balance for biomass
synthesis and is given below by eq.
\ref{eq:NCLim}.



\subsection{Nitrogen fluxes}
Nitrogen fluxes and pools are derived by dividing the respective fluxes with the
C/N ratio, $\beta$, of their source.

The C/N ratios $\beta_B$ and $\beta_E$ of the microbial biomass and of the
enzymes, are fixed. The C/N ratio of the litter pools, however, may change over
time and the N pools are modelled explicitly.
\begin{subequations}
\label{eq:SEAMN}
\begin{align}
\frac{dL_N}{dt} &=  - \operatorname{dec}_L /\beta_{L} +
\operatorname{i}_L/\beta_{\operatorname{i}} 
\\
\frac{dR_N}{dt} &=  - \operatorname{dec}_R /\beta_{R} +
\epsilon_{\operatorname{tvr}} \operatorname{tvr}_B /\beta_{B} + \notag 
\\
& \qquad (1 -k_{NB}) (\operatorname{tvr}_{ER} +
\operatorname{tvr}_{EL})/\beta_{E} 
\\
\frac{dI}{dt} &= +i_I -k_{IP} -l I +(1-\nu) u_{N,OM} +\Phi_B +
\operatorname{tvr}_N
\text{,} 
\end{align}
\end{subequations}

\noindent where the balance of the inorganic nitrogen pool $I$ sums inorganic
inputs $i_I$, plant uptake $k_{IP}$, leaching $l I$, the apparent mineralization
due to soil heterogeneity $(1 -\nu) u_{N,OM}$, which is explaind below,
mineralization-immobilization flux $\Phi_B$, and mineralization of a part of
microbial turnover $\operatorname{tvr}_N$.

N uptake was modelled as a PAR scheme, where apparantly part of the organic N is
mineralized because of soil spots with high N concentration in DOM
\citep{Manzoni08}. Potential nitrogen uptake is the sum of uptake from DOM and
the potential immobilization flux ($u_{\operatorname{imm,Pot}}$). Uptake from
DOM is assumed equal to influxes to DOM times the apparent nitrogen use 
efficiency $\nu$.
\begin{subequations}
\label{eq:uN}
\begin{align}
u_{N,OM} &= \operatorname{dec}_L/\beta_L + \operatorname{dec}_R/\beta_R +
k_{NB} (\operatorname{tvr}_{ER} + \operatorname{tvr}_{EL})/\beta_E
\\
u_N &= \nu u_{N,OM} + u_{\operatorname{imm,Pot}}
\\
u_{\operatorname{imm,Pot}} = i_B I 
\text{,}
\end{align}
\end{subequations}
where C/N ratios $\beta_L$ and $\beta_R$ are calculated based on current carbon
and nitrogen substrate pools.  

The nitrogen available for biomass synthesis is the difference of nitrogen
uptake and expenses for enzyme synthesis. This translates to an nitrogen
constraint for the carbon used for for biomass synthesis and its associated
catabolic growth respiration.
\begin{subequations}
\label{eq:synBN}
\begin{align}
N_{\operatorname{synBN}} &= u_N - \operatorname{syn}_E/\beta_E \text{,} \\
C_{\operatorname{synBN}} &= \beta_B \, N_{\operatorname{synBN}}  / \epsilon
\end{align}
\end{subequations}


\subsection{Imbalance fluxes of C versus N limited microbes }
There are constraints of each element for the carbon flux for synthesis of new
biomass and associated growth respiration. The minimum of these fluxes (eq.
\ref{eq:NCLim}) constrains the synthesis of new biomass. 

\begin{subequations}
\begin{align}
\label{eq:NCLim} 
C_{\operatorname{synB}} &=
min(C_{\operatorname{synBC}}, C_{\operatorname{synBN}} )
\end{align}
\end{subequations}

The excess elements are lost by imbalance fluxes (eq. \ref{eq:imbalance}).
The excess carbon is respired by overflow respiration, $r_O$, and the excess
nitrogen is mineralized, $M_{\operatorname{Imb}}$, so that the mass
balance is closed.
\begin{subequations}
\label{eq:imbalance}
\begin{align}
r_O &= u_C - (\operatorname{syn}_E /\epsilon  + \operatorname{syn}_B
+ \operatorname{r}_G + r_M )\\
M_{\operatorname{Imb}} &= u_N - (\operatorname{syn}_E/\beta_E +
\operatorname{syn}_{B}/\beta_B)
\end{align}
\end{subequations}

Potential biomass synthesis carbon constrained by N, $C_{\operatorname{synBN}}$,
is computed based on the potential immobilization flux
$u_{\operatorname{imm,Pot}}$. The real mineralization-immobilization flux $\Phi_B$ is the difference
between potential immobilization and excess nitrogen mineralization
$M_{\operatorname{Imb}}$. With carbon limitation this will be a positive
mineralization flux. With substrate nitrogen limitaiton, this will be a
negative flux corresponding to immobilization. Its magnitude is smaller than
the potential immobilization.
With uptake nitrogen limitation (i.e. required nitrogen flux is larger than
potential immobilization) $\Phi_B = -u_{\operatorname{imm,Pot}}$ and stoichiometry
is balanced by overflow respiration.

\subsection{Weight of an element limitation}

The weight of an element limitation is computed as the ratio between required
uptake flux for given other constraints to the available fluxes for
biosynthesis.
  
\begin{subequations}
\label{eq:weightsLim}
\begin{align}
w_{\operatorname{CLim}} &= \left( \frac{\text{required}}{\text{available}}
\right)^\delta 
= \left( \frac{ N_{\operatorname{synBN}} \beta_B }{ C_{\operatorname{synBC}} }
\right)^\delta
\\
w_{\operatorname{NLim}} &= \left( \frac{ C_{\operatorname{synBC}} / \beta_B }{
N_{\operatorname{synBN}} } \right)^\delta
\text{,} 
\end{align}
\end{subequations}
where parameter $\delta$, which was arbitrarily set to 200, controls
the steepness of the transition between the two limitations.
$X_{\operatorname{synBY}}$ denotes the available flux of element X for
biosynthesis and associated respiration given the limitation of element Y
(\ref{eq:synBC}) and \ref{eq:synBN}).


\subsection{Turnover mineralization fluxes}
In addition to mineralization flux due to stoichiometric imbalance, a part of
microbial biomass is mineralized during microbial turnover, e.g. by predation. A
part $(1-\epsilon_{\operatorname{Tvr}})$ of the biomass is used for catabolic
respiration. With assuming that predator biomass elemental ratios do not differ
very much from the one of microbial biomass, a respective proportion of the
nitrogen must be mineralized.
\begin{subequations}
\label{eq:MinTvrB}
\begin{align} 
r_{\operatorname{Tvr}} &= (1-\epsilon_{\operatorname{Tvr}}) \,
\operatorname{tvr}_B
\\
M_{\operatorname{Tvr}} &= (1-\epsilon_{\operatorname{Tvr}}) \,
\operatorname{tvr}_{B} / \beta_B
\end{align}
\end{subequations}

All the non-respired turnover carbon enters the residue pool. In reality, a part
of the microbial turnover probably enters the DOM pool again (e.g. by cell
lysis) and is taken up again by microbial biomass. The increased uptake nearly
cancels with an increased turnover. Hence, SEAM does not explicitly consider this
shortcut loop so that fewer model parameters are required.
Note, however, that turnover, uptake, and carbon use efficiency in the model are
slightly lower than in the real system where this shortcut operates.


%\section{\\ \\ \hspace*{-7mm}  SEAM equations \label{app:fAlphaFix}}    %%
\section{\hspace*{-7mm}  SEAM equations \label{app:fAlphaFix}}    %%
% Appendix B


