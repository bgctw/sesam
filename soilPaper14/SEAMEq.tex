%\section{\\ \\ \hspace*{-7mm}  SEAM equations}
%\section{\hspace*{-7mm}  SEAM equations \label{app:SEAM}}    
\section{SEAM equations \label{app:SEAM}}    
%% Appendix A

For an overview of symbol defintions see tables \ref{tab:modelStates},
\ref{tab:modelParameters}, and \ref{tab:furtherSymbols}.

\begin{table*}[t]
\caption{
%\captionof{table}{
\label{tab:modelParameters}
Model parameters. The two value columns of initial values and parameter values refer to the prototypical
examples and the Laqueuille pasture calibration respectively.}
%\vskip4mm
%\centering
%\begin{tabular}{lp{6.5cm}lllp{4.5cm}}
\begin{tabular}{lp{4.5cm}lllp{4.5cm}}
%{\setstretch{1} \begin{longtable}{lp{4cm}lllp{4cm}}
\tophline
Symbol &  Definition & \multicolumn{2}{c}{Value} & Unit & Rational \\
\middlehline
$\beta_B$ &  C/N ratio of microbial biomass & 11 & 11& \unit{g~g^{-1}} &
\citep{Perveen14} 
\\
$\beta_E$ &  C/N ratio of extracellular enzymes & 3.1 & 3.1 &
\unit{g~g^{-1}} & \citep{Sterner02} \\
$\beta_{\mathrm{input}_L}$ &  C/N ratio of plant litter inputs & 30 & 70 &
\unit{g~g^{-1}} & \citep{Perveen14} ($1/\beta$) \\
$k_R$ &  maximum decomposition rate of $R$ & 1 & 4.39e-2 & \unit{yr^{-1}}
& calibrated \\
$k_L$ &  maximum decomposition rate of $L$ & 5 & 1.95 & \unit{yr^{-1}}
& calibrated \\
$k_E$ &  enzyme turnover rate &  60  & 60 & \unit{yr^{-1}} & \citep{Burns13} \\
$\kappa_E$ & fraction enzyme tvr. entering DOM instead $R$ & 
0.8 & 0.8 & (-) & mostly small proteins \\
$a_{E}$ &  enzyme production per microbial biomass & 0.365 & 0.365 &
\unit{yr^{-1}} & $\approx 6\%$ of biomass synthesis \\ 
$K_{M}$ &  enzyme half saturation constant & 0.05 & 0.05 &
\unit{g~m^{-2}} & magnitude of DOC concentration \\
$\tau$ &  microbial biomass turnover rate & 6.17 & 6.17 & \unit{yr^{-1}} &
\citep{Perveen14} ($s/\epsilon_{\operatorname{tvr}}$) \\
$m$ & specific rate of maintenance respiration & 1.825 & 0 & 
\unit{yr^{-1}} & \citep{Bodegom07}, zero in \citep{Perveen14} \\
$\epsilon$ & anabolic microbial C substrate efficiency & 0.5 & 0.53 &
(-) & calibrated \\ %\citep{Perveen14} ($s/(s+r)$ +2\% enzymes) \\
$\nu$ & aggregated microbial organic N use efficiency & 0.7 &
0.9 & (-) & \citep{Manzoni08} \\
$\epsilon_{\operatorname{tvr}}$ & microbial turnover that is not
mineralized & 0.3 & 0.8 & (-) & part of turnover is consumed by
predators
\\
$i_{B}$ & maximum microbial uptake rate of inorganic N & 25 & 25 &
\unit{yr^{-1}} & larger than simulated immobilization flux \\
$l$ & inorganic N leaching rate & - & 0.959 &
\unit{yr^{-1}} & \citep{Perveen14} ($l)$ \\
\bottomhline
\end{tabular}
%\end{longtable}}
\end{table*}

\begin{table}[t]
\caption{
%\captionof{table}{
\label{tab:furtherSymbols}
Further symbols of quantities derived within the system}
%\vskip4mm
%\centering
\begin{tabular}{lp{4cm}l}
%\begin{longtable}{lp{4cm}lllp{4cm}}
%{\setstretch{1}
\tophline
Symbol &  Definition  & Unit \\
\middlehline
%\\
%\multicolumn{6}{l}{Fluxes and quantities derived within the system}\\
$\alpha$ & proportion of enzyme investments allocated to production of 
$E_R$ & (-) 
\\
$\operatorname{syn}_B $ & C for microbial biomass synthesis 
& \unit{g~m^2yr^{-1}} 
\\
$\operatorname{syn}_{E_S} $ & C synthesis of enzymes degrading $S \in \{L,R\}$
& \unit{g~m^2yr^{-1}} \\
$\operatorname{tvr}_B $ & microbial biomass turnover C 
& \unit{g~m^2yr^{-1}} \\
$\operatorname{tvr}_{E_S} $ & enzyme turnover C 
& \unit{g~m^2yr^{-1}} \\
$\operatorname{dec}_S $ & C in decomposition of resource $S \in \{L,R\}$
& \unit{g~m^2yr^{-1}} \\
$u_C,u_N$ & microbial uptake of C and N  
& \unit{g~m^2yr^{-1}} \\
$\Phi_u, \Phi_B, \Phi_{\operatorname{tvr}}$ & N mineralization with microbial
DOM uptake, stoichiometric imbalance, and turnover (Fig.
\ref{fig:SEAMStructNFluxes}) & \unit{g~m^2yr^{-1}} \\
\bottomhline
\end{tabular}
%} %stretch
%\end{longtable}
\end{table}

\subsection{Carbon fluxes}
\begin{subequations}
\label{eq:SEAM}
\begin{align}
\frac{dB}{dt} &= \operatorname{syn}_B - \operatorname{tvr}_B \\
\frac{dE_L}{dt} &= (1 - \alpha)  \, \operatorname{syn}_E -\operatorname{tvr}_{EL}\\
\frac{dE_R}{dt} &= \alpha \, \operatorname{syn}_E - \operatorname{tvr}_{ER} \\
\frac{dL}{dt} &=  - \operatorname{dec}_L + \operatorname{input}_L
\\
\frac{dR}{dt} &=  - \operatorname{dec}_R +
\epsilon_{\operatorname{tvr}}  \operatorname{tvr}_B + (1 -\kappa_E)
(\operatorname{tvr}_{ER} + \operatorname{tvr}_{EL})
\text{,} 
\end{align}
\end{subequations}
where $\alpha$ is the proportion of total investment into enzymes that is
allocated to the residue pool $R$ (section \ref{sec:AllocStrategies},
$\operatorname{input}_L$ is the litter C input to the system,
$\epsilon_{\operatorname{tvr}})$ is the fraction of microbial turnover C that is
respired by predators, and $\kappa_E$ is the fraction of enzyme turnover that is
transferred to the DOM instead of the $R$ pool.
The specific fluxes are detailed below.

Total enzyme production $\operatorname{syn}_E$, maintenance respiration
$\operatorname{r}_{M}$, and microbial turnover $\operatorname{tvr}_B$ are
modelled as a first-order kinetics of biomass:
\begin{subequations}
\begin{align}
\label{eq:synE} \operatorname{syn}_E &= a_E B \\
\label{eq:rM} \operatorname{r}_{M} &= m B \\
\label{eq:tvrB} \operatorname{tvr}_B &= \tau B 
\end{align}
\end{subequations}
% \begin{equation}
% \end{equation}

Enzyme turnover ($\operatorname{tvr}_{ER}$ and $\operatorname{tvr}_{EL}$) is
modelled as first-order kinetics of enzyme levels.
\begin{equation}
\label{eq:tvrE}
\operatorname{tvr}_{E_S} = k_{E} E_S \text{,}
\end{equation}
where $S$ represents the litter $L$ and residue $R$ substrate pools, respectively.  

% \begin{subequations}
% \begin{align}
% \operatorname{tvr}_{ER} &= k_{NR} E_R \\
% \operatorname{tvr}_{EL} &= k_{NL} E_L
% \end{align}
% \end{subequations}
% 
Substrate depolymerisation is modelled first-order to substrate
availability with a saturating Michaelis-Menten kinetics to enzyme levels:
\begin{subequations}
\label{eq:dec}
\begin{align}
\operatorname{dec}_{S,Pot} &= k_S S
\\ 
\operatorname{dec}_S &= dec_{S,Pot} \, \frac{E_S}{K_{M,S} + E_S}
\end{align}
\end{subequations}
 
The DOM pool is assumed to be in quasi steady state, and hence, the sum of all
influxes to the DOM pool (decomposition + part of the enzyme turnover) is taken
up by microbial community.
\begin{equation}
\label{eq:uC}
u_C = \operatorname{dec}_L + \operatorname{dec}_R +  \kappa_E
(\operatorname{tvr}_{ER} + \operatorname{tvr}_{EL})
\end{equation}

Under C limitation, C available for synthesis of new biomass and
associated catabolic growth respiration, $C_{\operatorname{synBC}}$, is the
difference between C uptake and expenses for enzyme synthesis (eq.
\ref{eq:synE}) and maintenance respiration (eq.
\ref{eq:rM}).

\begin{equation}
\label{eq:synBC} 
C_{\operatorname{synBC}} = u_C -
\operatorname{syn}_E/\epsilon - \operatorname{r}_{M}
\end{equation}

If the C balance for biomass synthesis, $\operatorname{syn}_B$ (eq.
\ref{eq:NCLim}), is positive, only a fraction $\epsilon$, the anabolic
carbon use efficiency (CUE) is used for synthesis of biomass and enzymes,
whereas the rest is used for catabolic growth respiration $r_G$ to support this
synthesis.
The model assumes that requirements for enzyme synthesis and maintenance must be
met. Hence, the microbial C balance can become negative where microbial biomass starves
and declines.

\begin{subequations}
\label{eq:synB}
\begin{align}
\operatorname{syn}_B &= \begin{cases}
  \epsilon \, C_{\operatorname{synB}},  & \text{if~} C_{\operatorname{synB}} >
  0\\
  C_{\operatorname{synB}}, & \text{otherwise}
\end{cases} \\  
\operatorname{r}_G &= \begin{cases}
  (1 - \epsilon) \, C_{\operatorname{synB}},  & \text{if~}
  C_{\operatorname{synB}} > 0\\
  0, & \text{otherwise ,}
\end{cases}  
\end{align}
\end{subequations}
%\noident
where $C_{\operatorname{synB}}$ is the C balance for biomass
synthesis and is given below by eq.
\ref{eq:NCLim}.



\subsection{Nitrogen fluxes}
Nitrogen fluxes and pools are derived by dividing the respective fluxes with the
C/N ratio, $\beta$, of their source.

The C/N ratios $\beta_B$ and $\beta_E$ of the microbial biomass and 
enzymes are assumed to be fixed. However, the C/N ratio of the substrate pools
may change over time and thus the substrate N pools are modelled explicitly.
\begin{subequations}
\label{eq:SEAMN}
\begin{align}
\frac{dL_N}{dt} &=  - \operatorname{dec}_L /\beta_{L} +
\operatorname{input}_L/\beta_{\operatorname{i}} 
\\
\frac{dR_N}{dt} &=  - \operatorname{dec}_R /\beta_{R} +
\epsilon_{\operatorname{tvr}} \operatorname{tvr}_B /\beta_{B} + \notag 
\\
& \qquad (1 -\kappa_E) (\operatorname{tvr}_{ER} +
\operatorname{tvr}_{EL})/\beta_{E} 
\\
\frac{dI}{dt} &= +i_I -k_{IP} -l I +\Phi
\\
\Phi &= \Phi_u +\Phi_B + \Phi_{\operatorname{tvr}}
\\
\Phi_u &= (1 -\nu) u_{N,OM}
\text{,} 
\end{align}
\end{subequations}

\noindent where the balance of the inorganic N pool $I$ sums inorganic
inputs $i_I$, plant uptake $k_{IP}$, leaching $l I$, and the exchange flux with
soil microbial biomass, $\Phi$. The latter is the sum of the
apparent mineralization
due to soil heterogeneity \citep{Manzoni08}, $\Phi_u$, mineralisation-immobilisation
imbalance flux, $\Phi_B$ (\ref{eq:PhiB}), and mineralisation of a part of microbial
turnover ,$\Phi_{\operatorname{tvr}}$ (\ref{eq:PhiTvr}, section \ref{sec:Tvr}).

Organic N uptake, $u_{N,OM}$, was modelled as a parallel scheme (PAR),
where a part of the organic N that is taken up from DON is mineralised
accounting at soil core scale accounting for imbalance flux at sub-scale soil
spots with high N concentration in DOM \citep{Manzoni08}.
Potential N uptake is the sum of organic N uptake and the potential
immobilisation flux ($u_{\operatorname{imm,Pot}}$). Uptake from DOM is assumed
equal to influxes to DOM times the apparent N use efficiency $\nu$.
\begin{subequations}
\label{eq:uN}
\begin{align}
u_N &= \nu u_{N,OM} + u_{\operatorname{imm,Pot}}
\\
u_{N,OM} &= \operatorname{dec}_L/\beta_L + \operatorname{dec}_R/\beta_R +
\kappa_E (\operatorname{tvr}_{ER} + \operatorname{tvr}_{EL})/\beta_E
\\
u_{\operatorname{imm,Pot}} &= i_B I 
\text{,}
\end{align}
\end{subequations}
where C/N ratios $\beta_L$ and $\beta_R$ are calculated based on current C
and N substrate pools: $\beta_L = L/L_N$.  

The N available for biomass synthesis is the difference of microbial N
uptake and expenses for enzyme synthesis. This translates to a N
constraint for the C used for biomass synthesis and its associated
catabolic growth respiration: $C_{\operatorname{synB}} \le
C_{\operatorname{synBN}}$.
\begin{subequations}
\label{eq:synBN}
\begin{align}
N_{\operatorname{synBN}} &= u_N - \operatorname{syn}_E/\beta_E \text{,} \\
C_{\operatorname{synBN}} &= \beta_B \, N_{\operatorname{synBN}}  / \epsilon
\end{align}
\end{subequations}


\subsection{Imbalance fluxes of C versus N limited microbes }
There are constraints of each element on the synthesis of new
biomass and associated growth respiration. The minimum of these fluxes (eq.
\ref{eq:NCLim}) constrains the synthesis of new biomass. 
\begin{equation}
\label{eq:NCLim} 
C_{\operatorname{synB}} =
min(C_{\operatorname{synBC}}, C_{\operatorname{synBN}} )
\end{equation}

The excess elements are lost by imbalance fluxes (eq. \ref{eq:imbalance}).
The excess C is respired by overflow respiration, $r_O$, and the excess
N is mineralised, $M_{\operatorname{Imb}}$, so that the mass
balance is closed.
\begin{subequations}
\label{eq:imbalance}
\begin{align}
r_O &= u_C - (\operatorname{syn}_B + \operatorname{syn}_E /\epsilon  + 
\operatorname{r}_G + r_M )\\
M_{\operatorname{Imb}} &= u_N -
(\operatorname{syn}_{B}/\beta_B + \operatorname{syn}_E/\beta_E )
\\
\label{eq:PhiB}
\Phi_B &= M_{\operatorname{Imb}} - u_{\operatorname{imm,Pot}}  
\end{align} 
\end{subequations}

The actual mineralisation-immobilisation flux $\Phi_B$ is the difference between
the potential immobilisation flux and excess N mineralization.
If microbes are limited by C availability, $\Phi_B$ will be positive, whereas with 
substrate N limitation, $\Phi_B$ will be a negative flux, corresponding to
N immobilisation.
With microbial N limitation, i.e. required immobilisation is larger than
potential immobilisation, $\Phi_B = -u_{\operatorname{imm,Pot}}$ and
stoichiometry must be balanced by overflow respiration.

\subsection{Weight of an element limitation}

The weight of an element limitation is computed as the ratio between required
uptake flux for given other constraints to the available fluxes for
biosynthesis.
  
\begin{subequations}
\label{eq:weightsLim}
\begin{align}
w_{\operatorname{CLim}} &= \left( \frac{\text{required}}{\text{available}}
\right)^\delta 
= \left( \frac{ C_{\operatorname{synBN}} }{ C_{\operatorname{synBC}} }
\right)^\delta
\\
w_{\operatorname{NLim}} &= \left( \frac{ \epsilon \, C_{\operatorname{synBC}} /
\beta_B }{ N_{\operatorname{synBN}} } \right)^\delta
\text{,} 
\end{align}
\end{subequations}
where parameter $\delta$, arbitrarily set to 200, controls
the steepness of the transition between the two limitations.
$X_{\operatorname{synBY}}$ denotes the available flux of element X for
biosynthesis and associated respiration given the limitation of element Y
(\ref{eq:synBC}) and (\ref{eq:synBN}).


\subsection{Turnover mineralization fluxes \label{sec:Tvr}}
In addition to mineralization flux due to stoichiometric imbalance, a part of
microbial biomass is mineralised during microbial turnover, e.g. by grazing. A
part $(1-\epsilon_{\operatorname{tvr}})$ of the biomass is used for catabolic
respiration. With assuming that predator biomass elemental ratios do not differ
very much from the one of microbial biomass, a respective proportion of N must
be mineralized.
\begin{subequations}
\label{eq:MinTvrB}
\begin{align} 
r_{\operatorname{tvr}} &= (1-\epsilon_{\operatorname{tvr}}) \,
\operatorname{tvr}_B
\\
\label{eq:PhiTvr}
\Phi_{\operatorname{tvr}} &=
(1-\epsilon_{\operatorname{tvr}}) \, \operatorname{tvr}_B / \beta_B
\end{align}
\end{subequations}

All the non-respired turnover C enters the residue pool. In reality, a part
of the microbial turnover probably enters the DOM pool again (e.g. by cell
lysis) and is taken up again by microbial biomass. The increased uptake nearly
cancels with an increased turnover. Hence, SEAM does not explicitly consider this
shortcut loop so that fewer model parameters are required.
Note, however, that turnover, uptake, and CUE in the model are
slightly lower than in the real system where this shortcut operates.


