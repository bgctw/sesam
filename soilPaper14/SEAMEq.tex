\section{\\ \\ \hspace*{-7mm}  SEAM equations \label{app:SEAM}}    %% Appendix A

\subsection{Carbon fluxes}
\begin{subequations}
\label{eq:SEAM}
\begin{align}
\frac{dB}{dt} &= \operatorname{syn}_B - \operatorname{tvr}_B \\
\frac{dE_R}{dt} &= \alpha \, \operatorname{syn}_E - \operatorname{tvr}_{ER} \\
\frac{dE_L}{dt} &= (1 - \alpha)  \, \operatorname{syn}_E -\operatorname{tvr}_{EL}\\
\frac{dR}{dt} &=  - \operatorname{dec}_R +
\epsilon_{\operatorname{tvr}}  \operatorname{tvr}_B + (1 -k_{NB})
(\operatorname{tvr}_{ER} + \operatorname{tvr}_{EL})\\
\frac{dL}{dt} &=  - \operatorname{dec}_L + \operatorname{input}_L
\text{,} 
\end{align}
\end{subequations}
where $k_{NB}$ is the fraction of enzyme turnover that is transferred to the
DOM instead of the $R$ pool, and $\operatorname{input}_L$ is the carbon input
to the system. The specific fluxes are detailed below.

Total enzyme production $\operatorname{syn}_E$, maintenance respiration
$\operatorname{r}_{M}$, and microbial turnover $\operatorname{tvr}_B$ are
modelled as a first order kinetics of biomass:
\begin{subequations}
\begin{align}
\label{eq:synE} \operatorname{syn}_E &= a_E B \\
\label{eq:rM} \operatorname{r}_{M} &= m B \\
\label{eq:tvrB} \operatorname{tvr}_B &= \tau B 
\end{align}
\end{subequations}
% \begin{equation}
% \end{equation}

Enzyme turnover ($\operatorname{tvr}_{ER}$ and $\operatorname{tvr}_{EL}$) is
modelled as first order kinetics of enzyme levels.
\begin{equation}
\label{eq:tvrE}
\operatorname{tvr}_{ES} = k_{NS} E_S \text{,}
\end{equation}
where $S$ represents either labile $L$ or residue $R$ substrate pools.  

% \begin{subequations}
% \begin{align}
% \operatorname{tvr}_{ER} &= k_{NR} E_R \\
% \operatorname{tvr}_{EL} &= k_{NL} E_L
% \end{align}
% \end{subequations}
% 
Substrate depolymerization is modelled first order to substrate
availability an with a saturating Michaelis-Menten kinetics to enzyme levels:
\begin{equation}
\label{eq:dec}
dec_S = k_S S \frac{E_S}{k_{m,S} + E_S} 
\end{equation}
 
The DOM pool is assumed to be in fast quasi steady state and all the sum of all
influxes to the DOM pool (decomposition + part of the enzyme turnover) is taken
up by microbial community.
\begin{equation}
\label{eq:uC}
u_C = \operatorname{dec}_L + \operatorname{dec}_R +  k_{NB}
(\operatorname{tvr}_{ER} + \operatorname{tvr}_{EL})
\end{equation}

\subsection{Carbon vs. Nitrogen limited microbes}
Under carbon limitation, carbon available for synthesis of new biomass,
$C_{\operatorname{synBC}}$, is the difference between carbon uptake and expenses
for enzyme synthesis (eq. \ref{eq:synE}) and maintenance respiration (eq.
\ref{eq:rM}). If this balance is positive, only a fraction $\epsilon$, the
intrinsic microbial carbon use efficiency, is used for growth and the rest is
required for catabolic growth respiration $r_G$ to support this synthesis.
The model assumes that requirements for enzyme synthesis and maintenance must be
met. Hence, the balance can become negative and microbial biomass starves and
declines.
\begin{subequations}
\label{eq:synB}
\begin{align}
\label{eq:synBC} C_{\operatorname{synBC}} &= u_C -
\operatorname{syn}_E/\epsilon - \operatorname{r}_{M} \\
\label{eq:NCLim} C_{\operatorname{synB}} &=
min(C_{\operatorname{synBC}},N_{\operatorname{synBN}}/\beta_B)\\
\operatorname{syn}_B &= \begin{cases}
  \epsilon \, C_{\operatorname{synB}},  & \text{if} C_{\operatorname{synB}} >
  0\\
  C_{\operatorname{synB}}, & \text{else}
\end{cases} \\  
\operatorname{r}_G &= \begin{cases}
  (1 - \epsilon) \, C_{\operatorname{synB}},  & \text{if}
  C_{\operatorname{synB}} > 0\\
  0, & \text{else ,}
\end{cases}  
\end{align}
\end{subequations}

where $N_{\operatorname{synBN}}$ is the nitrogen available for biomass synthesis
(eq. \ref{eq:synBN}). If is $N_{\operatorname{synBN}}$ smaller than required for
C-limited biomass synthesis ($N_{\operatorname{synBN}} / \beta_B <
C_{\operatorname{synBC}}$), then microbial biomass is nitrogen limited and less
carbon is used for biomass synthesis than available (eq. \ref{eq:NCLim}).  

\subsection{Nitrogen fluxes}
Nitrogen fluxes and pools are derived by deviding the respective fluxes with the
C/N ratio, $\beta$, of their source.

The C/N ratio $\beta$ of the microbial biomass and of the enzymes,
are fixed. The C/N ratio of the litter pools, however, my change over time and
the N pools are modelled explicitely.
\begin{subequations}
\label{eq:SEAMN}
\begin{align}
\frac{dR_N}{dt} &=  - \operatorname{dec}_R /\beta_{R} +
\epsilon_{\operatorname{tvr}} \operatorname{tvr}_B /\beta_{B} + \notag \\
& \qquad (1 -k_{NB}) (\operatorname{tvr}_{ER} +
\operatorname{tvr}_{EL})/\beta_{E}   \\
\frac{dL_N}{dt} &=  - \operatorname{dec}_L /\beta_{L} +
\operatorname{input}_L/\beta_{\operatorname{input}}
\text{,} 
\end{align}
\end{subequations}

Nitrogen uptake is the sum of influxes to the DOM pool. 
\begin{equation}
\label{eq:uN}
u_N = \operatorname{dec}_L/\beta_L + \operatorname{dec}_R/\beta_R +  k_{NB}
(\operatorname{tvr}_{ER} + \operatorname{tvr}_{EL})/\beta_E
\text{,}
\end{equation}
where C/N ratios $\beta_L$ and $\beta_R$ are calculated based on current carbon
and nitrogen substrate pools.

The nitrogen available for biomass synthesis is the difference of nitrogen
uptake and expenses for enzyme synthesis.
\begin{equation}
\label{eq:synBN}
N_{\operatorname{synBN}} &= u_N - \operatorname{syn}_E/\beta_E \text{,}
\end{equation}

\subsection{Imbalance and turnover mineralization fluxes}
The excess carbon is respired by overflow respiration, $r_O$, and the excess
nitrogen is mineralized, $M_\operatorname{Imb}$, so that the mass
balance is closed.
\begin{subequations}
\label{eq:imbalance}
\begin{align}
r_O &= u_C - (\operatorname{syn}_E/\epsilon  + \operatorname{syn}_B +
\operatorname{r}_G + r_M )\\
M_\operatorname{Imb} &= u_N - (\operatorname{syn}_E/\beta_E +
\operatorname{syn}_B/\beta_B)
\end{align}
\end{subequations}

In addition, a part of microbial biomass is mineralized during microbial
turnover, e.g. by predation. A part $(1-\epsilon_\operatorname{Tvr})$ of the
biomass is used for catabolic respiration. With assuming that predator biomass
elemental ratios do not differ very much from the one of microbial biomass, a
respecitve proportion of the nitrogen must be mineralized. 
\begin{subequations}
\label{eq:tvrB}
\begin{align}
r_\operatorname{Tvr} &= (1-\epsilon_{\operatorname{Tvr}}) \,
\operatorname{tvr}_B
\\
M_\operatorname{Tvr} &= (1-\epsilon_{\operatorname{Tvr}}) \,
\operatorname{tvr}_B / \beta_B
\end{align}
\end{subequations}

A part of the microbial turnover enters the DOM pool again and is taken up again
to microbial biomass. The increased uptake nearly cancels with an increased
turnover. Hence, the shortcut loop is not explicitely considered in the model
so that fewer model parameters are required.
Note, however, that turnover, uptake, and carbon use efficiency in the model
are slightly lower than in the real system where this shortcut operates. 


\section{\\ \\ \hspace*{-7mm}  SEAM equations \label{app:fAlphaFix}}    %%
% Appendix B


