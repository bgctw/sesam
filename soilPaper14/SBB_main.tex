%\documentclass[preprint,12pt,authoryear]{elsarticle}

%% Use the option review to obtain double line spacing
\documentclass[authoryear,preprint,review,12pt]{elsarticle}

%% main text
% twutz: some commands used in copernicus
\newcommand{\introduction}{\section{Introduction}}
\newcommand{\conclusions}{\section{Conclusions}}
\newcommand*\chem[1]{\ensuremath{\mathrm{#1}}}
\newcommand{\unit}[1]{\ensuremath{\mathrm{#1}}}

\usepackage{amsmath}


%% Use the options 1p,twocolumn; 3p; 3p,twocolumn; 5p; or 5p,twocolumn
%% for a journal layout:
%% \documentclass[final,1p,times,authoryear]{elsarticle}
%% \documentclass[final,1p,times,twocolumn,authoryear]{elsarticle}
%% \documentclass[final,3p,times,authoryear]{elsarticle}
%% \documentclass[final,3p,times,twocolumn,authoryear]{elsarticle}
%% \documentclass[final,5p,times,authoryear]{elsarticle}
%% \documentclass[final,5p,times,twocolumn,authoryear]{elsarticle}

%% For including figures, graphicx.sty has been loaded in
%% elsarticle.cls. If you prefer to use the old commands
%% please give \usepackage{epsfig}

%% The amssymb package provides various useful mathematical symbols
\usepackage{amssymb}
%% The amsthm package provides extended theorem environments
%% \usepackage{amsthm}

%% The lineno packages adds line numbers. Start line numbering with
%% \begin{linenumbers}, end it with \end{linenumbers}. Or switch it on
%% for the whole article with \linenumbers.
\usepackage{lineno}

\journal{Soil Biology \& Biochemistry}

\begin{document}

\begin{frontmatter}

%% Title, authors and addresses

%% use the tnoteref command within \title for footnotes;
%% use the tnotetext command for theassociated footnote;
%% use the fnref command within \author or \address for footnotes;
%% use the fntext command for theassociated footnote;
%% use the corref command within \author for corresponding author footnotes;
%% use the cortext command for theassociated footnote;
%% use the ead command for the email address,
%% and the form \ead[url] for the home page:
%% \title{Title\tnoteref{label1}}
%% \tnotetext[label1]{}
%% \author{Name\corref{cor1}\fnref{label2}}
%% \ead{email address}
%% \ead[url]{home page}
%% \fntext[label2]{}
%% \cortext[cor1]{}
%% \address{Address\fnref{label3}}
%% \fntext[label3]{}

\title{Adaptation of soil microbial resource allocation affects long term soil
organic matter and nutrient cycling.}

%% use optional labels to link authors explicitly to addresses:
%% \author[label1,label2]{}
%% \address[label1]{}
%% \address[label2]{}

\author{Thomas Wutzler}

\address{Max Planck Institute for Biogeochemistry, Hans-Kn{\"o}ll-Stra{\ss}e 10,
07745 Jena, Germany }

\begin{abstract}
In order to understand the coupling of global (C) and nitrogen (N) cycles, it is
necessary to understand C and N use efficiencies of soil microbial soil organic
matter (SOM) decomposers. While important controls of those efficiencies by
microbial community adaptations have been shown at pore scale, an abstract
simplified representation of community adaptations is needed at ecosystem scale.
Therefore I developed the SEAM model at soil core scale and modeled several
alternative community adaptation strategies of resource allocation to
extracellular enzymes. Using SEAM, I explored the effects of those alternative
strategy-hypotheses on SOM and inorganic N cycling. Simulations showed that the
revenue enzyme allocation strategy was most competitive, which accounted for
adaptations to both, stoichiometry and amount of different SOM resources.
Predictions of the holistic SEAM model were qualitatively similar to microbial
group explicit models with the ability to represent priming effects, bank
mechanism, and a continuous SOM sequestration with high inorganic N inputs. They
showed that soil enzyme allocation strategies strongly affect long term soil
organic matter cycling and nutrient recycling.
The findings imply that ecosystem scale models should account for adaptation of
C and N use efficiencies to represent and understand C-N couplings. The
combination of stoichiometry and optimality principles is a promising route to
yield simple formulations of such adaptations at community level suitable for
incorporation into land surface models.

\end{abstract}


\begin{keyword}
%% keywords here, in the form: keyword \sep keyword

%% PACS codes here, in the form: \PACS code \sep code

%% MSC codes here, in the form: \MSC code \sep code
%% or \MSC[2008] code \sep code (2000 is the default)

\end{keyword}

\end{frontmatter}

%% \linenumbers



\linenumbers
\introduction
% % \introduction[modified heading if necessary]
The global element cycles of carbon (C) and nitrogen (N) are intrically linked
and cannot be understood without their interactions \citep{Thornton07,
Zaehle11}. The links between nutrient cycles are especially strong in dynamic of
soil organic matter (SOM) because all of the SOM has to be depolymerized and
successively mineralized by a microbial community with a rather strict
homeostatic regulation \citep{Sterner02, Zechmeister15}. 

Faced with imbalances between stoichiometry of OM and their stoichiometric
requirements, decomposers have three options \citep{Mooshammer14}. First,
they can lower their carbon use efficiency (CUE) or nutrient use efficiency
(NUE) \citep{Sinsabaugh13}. The alteration of CUE has shown to have large
consequences on prediction of carbon sequestration in SOM \citep{Allison14a, Wieder13}.
Regulation of NUE has consequences for nutrient recycling and loss of nutrients
from the ecosystem \citep{Mooshammer14a} and soil plant feedback
\citep{Rastetter11}. 
As a second option, decomposer community can adapt their stoichiometric
requirements, although the flexibility is very narrow. At most the community
composition can shift between species with high C/N ratio, such as many fungi,
or species with lower C/N ratio, such as many bacteria \citep{Cleveland07,
Xu13}.
As a third option, decomposers can regulate allocation into production of
extracellular enzymes to preferentially degrade food sources of different
stoichiometry in order to match their requirements \citep{Moorhead12}. 

Representation and consequences of stoichiometry on element cycling differ
between models at different scales. Most soil models at ecosystem models scale
employ the first option and use CUE or NUE to represent stoichiometric controls
on respiration and mineralization fluxes \citep{Manzoni08}. However, modelling
studies at the pore scale that is relevant to microbes show the important effect
of community adaptation and their emerging effects on element cycling
\citep{Allison05, Resat11, Wang13}. Competition among several
microbial groups that differed in expression of different enzymes resulted in a
simulated convergence of CUE across a wider range of litter stoichiometry
\citep{Kaiser14}. Hence, there is a need to capture the effects of community
adaptation also in models at ecosystem scale.

Effects of microbial diversity can be represented at ecosystem scale by
at least two options. First, explicit representation of competition of
several microbial populations can represent stoichiometric effects such as
sustained sequestration of N with high N inputs can be simulated \citep{Perveen14}.
Second, effective properties of the entire microbial community can be
represented in a dynamic adaptive way \citep{Rastetter97, Rastetter11}.

The adaptation of enzyme production, was recently
formalized by the conceptual EEZY model \citep{Moorhead12}. While this model
showed the principle feasibility of this strategy in short term, it did not
implement feedback to resource pools and therefore could no look at
consequences for longer term SOM cycling. 

My task, therefore, was to tackle the need of capturing the effects of adaptive
enzyme production by building on the EEZY model to explore the different
consequences of alternative enzyme allocation schemes on long term SOM dynamics
and nutrient recycling.

This paper first introduces the SEAM model (Section \ref{sec:SEAM}), a
conceptual model of SOM cycling that explicitly represents microbial strategies
(Section \ref{sec:AllocStrategies}) of producing several extracellular enzyme
pools.
Next, the effect of those strategies on SOM cycling is presented by prototypical
examples (Sections \ref{sec:SimScen} and \ref{sec:ResultsProto}. Finally, a
calibration to an intensive pasture cite (Section
\ref{sec:methodsPasture}) demonstrates the usability of the model
(Section \ref{sec:ResultsPasture}) and compares its predictions to predictions
of the microbial-group explicit Symphony model \citep{Perveen14}.

This study shows that different decomposers enzyme allocation strategies
have large consequences on long term SOM dynamics and nutrient recycling. It
proposes a holistic scheme to represent effects of microbial adaptation of
enzyme production on SOM cycle at the ecosystem scale.

\section{Methods}
\subsection{Soil Enzyme Allocation Model (SEAM)
\label{sec:SEAM}}
Development of the conceptual structural dynamic Soil Enzyme Allocation Model
(SEAM) allowed to explore consequences of enzyme
allocation strategies for SOM cycling. The
modelled system are carbon and nitrogen pools in soil organic matter in
a representative elemental volume of soil. This could be soil in laboratory
incubation or a soil layer of about $1 \unit{m^2}$ outside the rhizosphere.
Different pools of carbon and nitrogen are represented as state variables and the mass
fluxes between them govern their change by dynamic equations. The model is
driven by inputs of plant litter (both above ground and rhizodeposition) and
inorganic nitrogen inputs, as well as prescribed uptake of inorganic nitrogen
by roots. It computes output fluxes of heterotrophic respiration and
leaching of inorganic nitrogen.

Key features are first, the representation of several SOM pools that differ by
elemental ratios and second, the representation of specific enzymes
that degrade those SOM pools. The quality spectrum is modelled by two
classes: a nitrogen rich more easily degradable labile litter pool,
$\operatorname{L}$, and a pool mainly consisting of nitrogen rich microbial
residues, $\operatorname{R}$ (Fig. \ref{fig:SEAM}).

\begin{figure}[t] \vspace*{2mm}
\label{fig:SEAM}
\begin{center}
\includegraphics[width=8.3cm]{fig/seam2}
\end{center}
\caption{Model strucutre of SEAM: Two resource pools ($L$ and $R$) which differ
in their elemetal ratios are
depolymerized by respective enzymes ($E_L$ and $E_R$). The simple organic
compounds ($\operatorname{DOM}$) are taken up
by the microbial community  and used for synthesizing
new biomass (${B}$), new enzymes, for catabolic respiration, and an apparent
N mineralization during uptake ($\Phi_u$). Stoichiometric imbalance
between $\operatorname{DOM}$ and ${B}$ causes overflow respiration 
or mineralization of excess nitrogen ($\Phi_B$) to inorganic N ($I$).
During microbial turnover an additional part of the biomass is mineralized
($\Phi_{\operatorname{Tvr}}$).
% and the other part is added to the residue pool, $R$. 
Boxes correspond to pools, black
arrows heads to mass fluxes, white arrows heads to other controls, and disks
represent partitioning of fluxes. Solid lines represent fluxes of both C and N, while
dotted and dashed lines represent separate C or N fluxes respectively.}
\end{figure}

The most important assumptions are describe in the next paragraph, while
the symbols are explained in Tab. \ref{tab:modelParameters} and detailed model equations are
given in Appendix \ref{app:SEAM}.

%\onecolumn
\section{Symbols and Values}
\label{app:symbols}



    


%\twocolumn
 

C/N ratios, $\beta$ of biomass and enzymes are
fixed while those of the resource pools may change over time due to
fluxes in and out of the pools. Imbalances in stoichiometry of uptake and demand
in microbial biomass are compensated by overflow respiration or nitrogen
mineralization.
Total enzyme allocation is a fixed fraction, $a_E$, of the microbial biomass,
$B$, per time. But microbial community can use different strategies
(Section \ref{sec:AllocStrategies}) to adjust their allocation to produce new
enzymes $E_R$ or $E_L$: $\alpha = E_R / (E_R + E_L)$. 
The DOM pool is assumed to be in quasi steady state and all the sum of all
influxes to the DOM pool (decomposition + part of the enzyme turnover) is taken
up by the microbial community. If expenses for maintenance and enzyme production
cannot be met, the biomass starves and declines.

\subsection{ Enzyme allocation strategies 
\label{sec:AllocStrategies}}

Three different strategies of allocating investments among production of
alternative enzymes were explored in this study (Table
\ref{tab:AllocStrategies}). Microbes invest proportion $\alpha$ of total enzyme
investments into production of enzymes targeting the nitrogen rich $R$ resource
and proportion $1 - \alpha$ into production of enzymes targeting the nitrogen
poor but better degradable $L$ resource.

\begin{table}[t]
\caption{Simulation scenarios \label{tab:AllocStrategies}}
\vskip4mm
\centering
\begin{tabular}{ll}
\tophline
Strategy &  Allocation is \\
\middlehline
Fixed & independent, constant \\
Match & adjusted to achieve balanced growth \\
Revenue & proportional to return per investments into enzymes \\
\bottomhline
\end{tabular}
\end{table}


The \textbf{Fixed} strategy assumes that allocation is independent
and not changing with changes in substrat availability.
\begin{equation} 
\label{eq:allocFixed}
\alpha = \operatorname{const.}
\end{equation}
This strategy corresponds to the models where decomposition rate is a function
of microbial biomass \citep{Wutzler08}.
 
The \textbf{Match} strategy assumes that microbes balance their stoichiometric
demands \citep{Moorhead12}. The partitioning coefficient is derived by equating
the C/N ratio of the sum of decomposition and immobilization fluxes and the C/N
ratio of microbial biomass, $\beta_B$. 
\begin{equation} 
\label{eq:allocMatchCN}
\frac{\epsilon (\operatorname{dec}_L + \operatorname{dec}_R - r_M)}{
\nu (\operatorname{dec}_L/\beta_L + \operatorname{dec}_R/\beta_R)  +
\Phi_{B} } =
\beta_B
\text{,}
\end{equation}
where $\operatorname{dec}_L$, and $\operatorname{dec}_R$ are 
depolymerization fluxes of the labile and residue resources
respectively, $r_M$ is maintenance respiration, $\epsilon$ is the 
intrinsic microbial efficiency, and $\beta$ are C/N ratio of
the respective pools $i$, and $\Phi_B$ is the mineralization-immobilization
imbalance flux from mineral N pool to microbes.
When the current sum of enzyme levels, $E$ is assumed to be in quasi steady
state for given amounts of resource and microbial biomass, then equation
\ref{eq:allocMatchCN} can be solved for allocation partitioning, $\alpha$.
\begin{subequations}
\label{eq:allocMatch} 
\begin{align}
\alpha_M &= f_{\operatorname{{\alpha}Fix}}(L,\beta_L,R,\beta_R, E); \,
\alpha &= \begin{cases}
  0,  & \text{if} \alpha_M \le 0 \\
  1,  & \text{if} \alpha_M \ge 1 \\
  \alpha_M, & \text{otherwise}
\end{cases} \\  
\end{align}
\end{subequations}

The bounds to zero and one are necessary because the sole
allocation to decompose the carbon rich resource may not suffice to
cover microbial carbon demands and solving eq. \ref{eq:allocMatchCN} for
$\alpha$ then leads to unreasonable solutions.
Function $f_{\operatorname{{\alpha}Fix}}$ is given in appendix
\ref{app:fAlphaFix} and the sympy script of its 
derivation is given with supplementory material. 

The \textbf{Revenue} strategy assumes that microbial community adatps in a way
so that investment into enzyme production is proportional to their revenue, i.e.
their return per investment regarding the currently limiting element.

The return is the current decomposition flux from respective resource. The
investment needs to be equal to enzyme turnover to keep current enzyme levels,
$E_S^*$.
% : $a_E B = k_{NR} E_R^* + k_{NL} E_L^*$.
% 
\begin{subequations}
\label{eq:allocRev}
\begin{align}
\alpha_C &= \frac{\operatorname{rev}_{RC}}{\operatorname{rev}_{LC} + \operatorname{rev}_{RC}} \\
\alpha_N &= \frac{\operatorname{rev}_{RN}}{\operatorname{rev}_{LN} + \operatorname{rev}_{RN}} \\
\operatorname{rev}_{SC} &= \frac{\text{return}}{\text{investment}} 
= \frac{\operatorname{dec}_{S,Pot} \frac{E_S^*}{k_{m,S} + E_S^*}} {k_{NS}E_S^*} 
= \frac{\operatorname{dec}_{S,Pot}} {k_{NS}(k_{m,S} + E_S^*)} \\ 
\operatorname{rev}_{SN} &= \frac{\operatorname{dec}_{S,Pot}
\frac{E_S^*}{k_{m,S} + E_S^*} / \beta_S} {k_{NS} E_S^* / \beta_E} 
= \frac{\operatorname{dec}_{S,Pot}}{k_{NS} (k_{m,S} + E_S^*)}
\frac{\beta_E}{\beta_S}
\text{,} 
\end{align}
\end{subequations}
where $\operatorname{rev}_S$ is the revenue from given resource $S$ ($S$ is either $L$ or $R$)
under carbon and nitrogen limitation respectively.
$k_{NS}$ is rate of enzyme turnover, $k_{m,S}$ is resource affinity, $a_E$ is
total enzyme allocation coefficient, $\operatorname{dec}_{S,Pot} = k_S S$ is
enzyme saturated decomposition flux, and $\beta$ are C/N ratios of the
respective pools.

There are two partitioning coefficients, $\alpha_C$, with carbon limited
microbial biomass, and $\alpha_N$, with nitrogen limited microbial biomass. In order to
avoid frequent large jumps under near co-limitation, SEAM implements a smooth
transition between these two cases as a weighted average with weights based
on ratio of required to available biomass synthesis fluxes (derived in
appendix \ref{app:SEAM}).

\begin{subequations}
\label{eq:allocRev}
\begin{align}
\alpha &= \frac{w_{\operatorname{CLim}} \alpha_C + w_{\operatorname{NLim}}
\alpha_N}{w_{\operatorname{CLim}}  + w_{\operatorname{NLim}} } 
\\
w_{\operatorname{CLim}} &= \left( \frac{\text{required}}{\text{available}}
\right)^\delta 
= \left( \frac{ N_{\operatorname{synBN}} \beta_B }{ C_{\operatorname{synBC}} }
\right)^\delta
\\
w_{\operatorname{NLim}} &= \left( \frac{ C_{\operatorname{synBC}} / \beta_B }{
N_{\operatorname{synBN}} } \right)^\delta
\text{,} 
\end{align}
\end{subequations}
where $\delta$ controls the steepness of the transition between the two
limitations. It was arbitrarily set to a value of 200.


\subsection{ Simulation scenarios 
\label{sec:SimScen}}

The consequences of different microbial enzyme allocation strategies for SOM
dynamics are explored by several simulation scenarios (Table \ref{tab:SimScen}). 
All scenarios used parameter values are given in Table \ref{tab:modelParameters}, if not
stated specifically otherwise below. I chose a very low minimum turnover time
of the residue SOM pool of 10 years in order to demonstrate and plot changes together
with the faster litter pool.

\begin{table}[t]
\caption{Simulation scenarios \label{tab:SimScen}}
\vskip4mm
\centering
\begin{tabular}{lp{5.3cm}}
\tophline
Scenario & Explored issue\\
\middlehline
VarN-Incubation & Efficieny of using given fixed resource levels that
vary by N content \\
Feedback-Steady & Possibility and size of steady state resource pools\\
Priming & Increased resource decomposition and mineralization after
addition of fresh litter\\
CO2-Fertilization & SOM dynamics after increased carbon inputs\\
\bottomhline
\end{tabular}
\end{table}

The \textbf{VarN-Incubation} scenario explored how resources of given
stoichiometry are used more or less efficiently with different enzyme allocation
scenarios. It used a simplified model where all the inputs and feedback to the
resource pools ($L$ and $R$) were neglected and these pools were kept constant
($dL/dt = dR/dt = 0$), and no constraints on the N-mineralization/immobilization
flux. It allowed microbes and enzyme levels to develop to a steady
state with the given resource supply. Hence, it simulated the end of a short term
incubation. Specifically, it used fixed resource carbon of $L=100\mathrm{gC}$
and $R=400\mathrm{gC}$. It set C/N ratio of the residue pool to $\beta_R=6.8$,
and varied the litter C/N ratio ($\beta_L = [18,..,42]$).

The \textbf{Feedback-Steady} scenario explored the long term trajectories of the
entire system including feedback to the substrate pools, including steady states
and microbial biomass levels. Specifically, it set litter input to
$\operatorname{input}_L = 300\mathrm{gCm}^{-2}\mathrm{yr}^{-1}$ with C/N ratio
$\beta_{\operatorname{input}_L} = 30$.

The \textbf{Priming} scenario explored the effect of rhizosphere priming, i.e
the input of fresh carbon into a bulk subsoil. It looked at the fluxes after
an addition of $50g$ of litter on a soil that otherwise
received a litter input of only $30\mathrm{gCm}^{-2}\mathrm{yr}^{-1}$ for a
decade. It simulated the input of an easily available substrate, specifically
with a maximum turnover of $3650\mathrm{yr}^{-1}$.

The \textbf{CO2-Fertilization} scenario explored the effect of continuous 
increased carbon litter input that is expected with elevated atmospheric CO2.
It started from steady state for a litter input, applied 20\% increased
carbon inputs during years 10 to 60, and initial carbon inputs during the
next 50 years. 

\subsection{Calibration to a fertilized grassland site
\label{sec:methodsPasture}} 

To test the capability of the SEAM model to simulate the carbon sink of a
grassland site, I calibrated the model to data of an intensive pasture. The
model drivers and most of the parametrization was taken from the publication of
\citep{Perveen14}. The site is a temperate permanent grassland
located at an altitude of 1040m in France (Laqueuille, 45\textdegree{38}'N,
2\textdegree{44}'E) and has an annual precipitation and temperature of
1200~mm and 7 \textdegree{C}, respectively. 

In all scenarios, plant uptake of inorganic N were
computed assuming the plants to be in steady state with the litter
production and biomass exports. The N balance of the fertilized grassland is
 characterized by high inorganic N-inputs. Part of this N is sequestered in accumulating SOM, part
is leached and part is exported with plant biomass. Parameters were chosen
corresponding to Table 1 in \citep{Perveen14}. 

The three calibrated parameters were the maximum decomposition rates of
substrate pools, $k_L$ and $k_R$, and the microbial carbon use efficiency,
$\epsilon$. Initial pools were prescribed to observed values where available or
to the long-term state after a preliminary optimization in order to prevent
large initial fluctuations. The optimization used the \textit{optim}
function from R \textit{stats} package \citep{R07} to minimize the differences
between model predictions and observations normalized by the standard deviation
of the observations.
 
\section{Results}

First, the results of several prototypical artificial simulation scenarios
clarify the general behaviour and features of the SEAM model. Next, results of a
parameter calibration demonstrates the model's ability to
simulate the observed C and N dynamics of an intensive pasture.

\subsection{Prototypical simulation scenarios
\label{sec:ResultsProto}}

XX
With the \textbf{VarN-Incubation} scenario, there were differences among
allocation strategies for the dependence of allocation $\alpha$ on the nitrogen
content of the litter resource. These caused marked changes in
biomass and imbalance fluxes (Fig.
\ref{fig:VarNNoFeedback}).
 
\begin{figure}[t]
\vspace*{2mm}
\begin{center}
\includegraphics[width=8.3cm]{fig/VarNNoFeedback}
\end{center}
\caption{Match enzyme allocations strategy yields highest resource
efficiency, i.e. lowest mineralization fluxes with the VarN-scenario. Microbes
with alternative strategies, however, compete better indicated by a higher biomass.
\label{fig:VarNNoFeedback}}
\end{figure}

The Match strategy allowed balanced growth and efficient resource usage, but
yielded less biomass than the other scenarios. When the litter contained enough
N, all resources were invested into litter degrading enzymes. The strategy
allowed a balanced microbial growth without stoichiometric imbalance fluxes,
i.e. mineralization of excess N or overflow respiration of excess C, across a
wide range of litter C/N ratios (20 to 42). While for favourable C/N ratios, it
yielded a higher biomass than the fixed strategy, biomass stocks were lower for
litter CN-ratios below 24. This was because I prescribed smaller stocks of
litter, $L$ than residue resource, $R$. 

With the Revenue strategy enzyme allocation also varied with litter N content.
However, under conditions where L contained enough N, about 30\% of the efforts
were invested into R degrading enzymes. This resulted in excretion of
excess N, but in turn allowed for a higher microbial biomass. Imbalance fluxes,
were smaller than with the Fixed strategy.

Threshold elemental ratio slightly increased with higher litter C/N ratio, as
can be seen in the C/N ratio of the DOM in the match strategy. The reason for this
increase, here, was a lower proportion of uptake flux compared to immobilization
flux with lower biomass.
The Carbon use efficiency (carbon used for biomass synthesis per
carbon taken up) XX

When including feedback to the resource pools within the model with the
\textbf{SimSteady} scenario, resource pools approached steady state with the
Fixed and the Revenue strategies, but there was no reasonable steady state for
the Match strategy (Fig. \ref{fig:SimSteady}).

\begin{figure}[t]
\vspace*{2mm}
\begin{center} 
\includegraphics[width=8.3cm]{fig/SimSteady} 
\end{center}
\caption{Match strategy is not viable when considering feedback to
substrate pools with the SimSteady scenario, where microbes degrade a
stoichiometrically matching but depleted R substrate pool.
\label{fig:SimSteady}} 
\end{figure}

With the match strategy microbes grew mostly on the
nitrogen rich residues pool and only partly on the litter pool. This led to an
accumulation of the litter and a degradation of the residue pool. Together with
diminishing return from residue pool, microbes starved and declined.
Because of the Match strategy was not able to simulate reasonable
stocks when including feedback to resource pools in the model, it was omitted
from the remaining simulation scenarios.

After amending a starved subsoil with a pulse of litter in the
\textbf{Priming} scenario, a clear real priming effect was simulated. The
decomposition of the soil residues pool was strongly enhanced after the
amendment. The priming was stronger with the Revenue strategy than with the
Fixed strategy (Fig. \ref{fig:PrimingDec}). 

\begin{figure}[t]
\vspace*{2mm}
\begin{center}
\includegraphics[width=8.3cm]{fig/PrimingDec}
\end{center}
\caption{Residue depolymerization flux increases strongest with the Revenue
strategy after amending a starved subsoil with a pulse of fresh litter with the
Priming scenario.
\label{fig:PrimingDec}}
\end{figure}

\begin{figure}[t]
\vspace*{2mm}
\begin{center}
\includegraphics[width=8.3cm]{fig/PrimingMin}
\end{center}
\caption{Priming a low-input soil by a fresh litter pulse stimulates N
mineralization most strongly with the Revenue strategy with the Priming scenario.
\label{fig:PrimingMin}}
\end{figure}

This stronger priming was due to a higher microbial biomass with Revenue
strategy. Therefore also the N-mineralization flux due to microbial turnover
was larger with the Revenue strategy. After an initial phase of reduced
N-mineralization due to reduced imbalance, the N in mineral biomass was
mineralized by turnover (Fig. \ref{fig:PrimingMin}).

The most marked difference between enzyme allocation strategies was simulated
with the \textbf{CO2-Fertilization} scenario (Fig. \ref{fig:CO2Increase}).
\begin{figure}[t] \vspace*{2mm}
\begin{center}
\includegraphics[width=8.3cm]{fig/CO2Increase}
\end{center}
\caption{
Revenue strategy leads to depletion of SOM and liberation of nutrients, i.e.
nutrient mining, during increased carbon litter inputs
in years 10 to 60 with the CO2-Fertilization scenario.
\label{fig:CO2Increase}}

\end{figure}
\begin{figure}[t] \vspace*{2mm}
\begin{center}
\includegraphics[width=8.3cm]{fig/CO2IncreaseImb} 
\end{center}
\caption{
Mineralization of microbial turnover with the Revenue strategy
contributed most of the liberation of SOM-N during CO2-Fertilization starting
at year 10.
At the end of the fertilization at year 60, mineralization due to stoichiometric
imbalance was smaller with the Revenue strategy.
\label{fig:CO2IncreaseImb}}
\end{figure}
With both, Fixed and Revenue strategies, the litter stock,$L$ increased. The
residues stock $R$, however, slightly increased with the Fixed strategy but
declined with the Revenue strategy. With the Revenue strategy, in addition, the
overflow respiration during increased carbon inputs and the N mineralization
in the period of decreased carbon input was lower compared to the Fixed
strategy (Fig. \ref{fig:CO2IncreaseImb}).

\subsection{Intensive pasture simulation
\label{sec:ResultsPasture}}

The SEAM model successfully simulated the C and N balance of the Laqueuille
intensive pasture (Figure \ref{fig:pastureFit} left), although only 3 parameters
have been calibrated.
The inputs of inorganic N exceeded the losses inorganic N. They fuel a buildup
of a pool of organic N in the recalcitrant SOM, by two pathways. First,
inorganic N is taken up by the plant and supplied via organic N in litter, and
second, microbial biomass immobilizes inorganic N due to stoichiometric
imbalance of substrate. The microbial biomass was substrate N-limited. However,
it actually was C-limited because immobilization of inorganic nitrogen balanced
the substrate limitation. 

\begin{figure}[t] \vspace*{2mm}
\label{fig:pastureFit}
\begin{center}
\includegraphics[width=8.3cm]{fig/pastureFit} 
\end{center}
\caption{Calibrated Seam model predictions match observed values of the
Laqueuille intensive pasture (dots and errorbars left). On altered C and N
inputs, the predictions show a shift in enzyme allocation ($\alpha$) and SOM
turnover (right).
Increased N substrate limitation either due to elevated CO2 or due to
decreasing inorganic N inputs cause an increase in labile pool, $L$, and a
decrease in mineral N pool, $I$. If the substrate N limitation cannot be balanced by mineral N
input then the change rate of the recalcitrant pool, $dR$, decreases down to
negative values, i.e. losses.
}
\end{figure}   

Altered C and N inputs to the system strongly affected the internal SOM and
nutrient cycling (Figure \ref{fig:pastureFit} right). 

\textit{Increased litter C input} by 50\% together with an increased litter
C/N ratio by 25\% caused a shift in enzyme allocation towards enzymes degrading the
N-richer recalcitrant litter. It subsequently caused an increase in the
labile litter pool and a increased demand for mineral N, both for 
the plant to balance increased biomass production and higher stoichiometric
imbalance of microbial biomass. The resulting decrease in mineral N also
decreased leaching losses. Increased N efficiency within the system was
additionally favoured by a faster recycling of N due to increased
microbial activity. 

\textit{Decreased inorganic N inputs} from 22.9
${\textrm{gm}^{-2}\textrm{yr}^{-1}}$ down to 1
${\textrm{gm}^{-2}\textrm{yr}^{-1}}$ together with an doubling of litter C/N
ratio caused a strong shift in enzyme allocation towards enzymes degrading the
N-richer recalcitrant SOM with similar consequences as with increased C input,
such as an increase in labile SOM. However, here, the decreased N inputs caused
a depletion of the mineral N pool.
As a consequence, the microbial biomass could not use immobilization to
balance substrate stoichiometry and became N-limited.
This caused overflow respiration and a decreasing trend in recalcitrant SOM.

\textit{Increased inorganic N inputs} from 22.9 $gm^{-2}yr^{-1}$ up to 25.6
$gm^{-2}yr^{-1}$ together with decrase of litter C/N by 25\% caused a strong
shift in enzyme allocation towards enzymes degrading the labile litter SOM.
Since, the soil system was C-limited before, it could only use a small part of
the additional N for building up SOM.
Instead, the increased N inputs caused the inorganic carbon pool to increase
with associated increase in leaching losses.

\section{Discussion}
\subsection{Importance of resource amounts}
The amounts of resource pools together with resource stoichiometry are
important for regulating enzyme allocation. The match strategy does not account
for resource amounts despite their importance. It assumes that microbes can
achieve balanced growth under a wide range of resource stoichiometry,
specifically when describing incubation experiments \citep{Moorhead12,
Ballantyne14}. However, microbial biomass stocks declined with the Match
strategy both in the VarN-Incubation scenario (Fig. \ref{fig:VarNNoFeedback})
and in the SimSteady scenarios (Fig. \ref{fig:SimSteady}) of this study. Such
biomass decline was caused by focusing enzyme production on pools that matched
stoichiometry demands but were small compared to the other pools. Microbes
invested enzymes into degrading a small but stoichiometrically
balanced residues pool despite the huge amount of N available in a large but
stoichiometric imbalanced litter pool (Fig.
\ref{fig:SimSteady}). This strategy led to the further accumulation of the
litter pool and the exhaustion of both the residues pool and the microbial
biomass. I argue that such unreasonable behaviour is evidence against the Match
strategy and conclude that an improved strategy must account for resource
amounts.

\subsection{Community adapataion lead to more efficient resource usage} 
Revenue strategy consistently supported higher biomass and had
lower N mineralization fluxes compared to the the Fixed strategy with the
VarN-Incubation scenario (Fig. \ref{fig:VarNNoFeedback}). Similar patterns
appeared with the other scenarios (Fig. \ref{fig:SimSteady}
\ref{fig:CO2IncreaseImb}). With the SimSteady scenarios the revenue strategy
yielded higher biomass (XX vs XX) at steady state and lower XX fluxes (XX vs
XX).
Microbial communities that can adapt use
the available resource more efficiently and sustain more biomass for the same litter inputs and
environmental conditions. They use the food more for productive respiration and
turnover that sustains a food chain rather than for imbalance waste fluxes.
Such better resource usage is in line with results of individual based
small-scale modelling \citep{Kaiser14}.
 The finding implies that mineralization fluxes with imbalanced resources may
be lower than inferred from previous modelling studies that did not account for community
adaptation.

The SEAM model focuses on community adaptation of enzyme production. It predicts
a change in the ratio of enzyme activities of enzymes degrading plant litter,
which degrade mainly carbon compounds, versus enzymes degrading the residual
pool which degrade mainly nitrogen containing compounds.
However, in a spatial context at global scale, low variation in stoichiometry of
enzymatic activity have been observed \citep{Sinsabaugh09}, implying either
issues with measurement of enzymatic stoichiometry or implying further
conservative constraints on the enzymatic flexibility of microbial community.
Such stoichiometric constraints have a large effect on SOM cycling and hence
need further study.

\subsection{Bank mechanism}
The Revenue strategy led to a redistribution of resource types during high
and low carbon inputs with the  CO2-Fertilization scenario (Fig.
\ref{fig:CO2Increase}). When there was excess litter carbon, microbes
preferentially depolymerized the nitrogen rich residue pool in order to make use of available
carbon. When decreasing carbon inputs, excess nitrogen was used to
build the residue pool again instead of releasing N by mineralization.

The bank mechanism also worked when simulating the intensive
pasture (Fig. \ref{fig:pastureFit}). During simulations of high inorganic
N inputs, N was sequestered in SOM at a high rate. With decreasing inorganic N
inputs, the sequestration rate decreased until it became negative, the N of SOM
was mined.
 
Ultimately, on the long term, the inputs to the system have to balance the
outputs of the system. Hence there are constraints on possible alteration of
total respiration and N mineralization. However, changes can be buffered by a 
depositing or mining a reserve of excess nitrogen in the residue pool.
The limit of this bank capacity is the difference in steady states between
inputs levels. In our simulations this was about twice the annual N
litter input. 

This implies an important point in interpreting the feedback and consequences
of global change.  First, the additional storage i XX

\subsection{Nutrient recycling by priming}
The priming effect can help plants to stimulate nutrient release.
Priming effects were simulated for both strategies (Fig. \ref{fig:PrimingDec})
associated with an increased nitrogen mineralization flux.
Hence, the difference in plant resource inputs to the soil could influence the
recycling of nutrients from SOM.
Nutrient recycling is the other side of the bank mechanism. If carbon is
sequestered in SOM, then also nutrients are sequestered and become unavailable
for plants.
With the revenue strategy, however, plant inputs could influence the
partitioning of SOM between a low C/N SOM pool and a high C/N pool (Fig.
\ref{fig:CO2Increase}) and, hence, also influence the distribution of nitrogen
in the ecosystem and nutrient availability for plants. This active role of
plants has already demonstrated in a soil incubation experiment
\citep{Fontaine11} and has been further conceptualized with the SYMPHONY model
\citep{Perveen14}. Our results are in line with these studies, although our
explanation is on a more abstract level (section \ref{sec:Holistic}). It implies
greater potential for plants to avoid progressive nitrogen limitation
\citep{Norby10, Franklin14, Averill15}.

Mineralization during microbial turnover is important for nutrient recycling.
Mineralization by stoichiometric imbalance is the most widely implemented flux
from microbial biomass to the inorganic carbon pool \citep{Manzoni09}. It
constitutes the imbalance flux, $\Phi_B$, also called mineralization
immobilization flux.
Two additional mineralization fluxes are implemented with the SEAM model.
First, there is a fraction of microbial uptake, termed uptake mineralization, is
apparently mineralized that accounts for C-limited locations in heterogeneous
soil \citep{Manzoni08}.
Second, a fraction of microbial turnover is mineralized accounting for grazing.
Grazers respire part of the grazed biomass carbon for their energy, and must
release an equivalent amount of nutrients to match their stoichiometric demands.
This mineralization component, here termed turnover mineralization, has been
formalized in the soil microbial loop hypothesis \citep{Clarholm85, Raynaud06}.
Without such additional mineralization mechanisms, in our simulation experiment
microbes shifted enzyme allocation to degrade the residues pool, but the
Nitrogen was fixed in microbial biomass and not liberated. Hence, our
experiments reinforced the need for representing soil heterogeneity and grazing
for making N available for plants under N limitation.

\begin{figure}[t] \vspace*{2mm}
\begin{center}
%\includegraphics[width=8.3cm]{fig/seam_NFluxes} 
\includegraphics[scale=0.6]{fig/seam_NFluxes} 
\end{center}
\caption{
In addition to the imbalance flux, $\Phi_B$ of microbial biomass, $B$, there are
additional mineralization fluxes feeding the inorganic pool, $I$, due to
mineralization during uptake, $\Phi_u$, and mineralization 
during microbial turnover, $\Phi_{\operatorname{tvr}}$. The depicted N
fluxes of the system depends also on fluxes across the system boundary,
namely input of litter $iL$ to the soil organic matter ($SOM$), input of inorganic N $iI$, leaching, and
plant uptake.
\label{fig:seamNFluxes}}
\end{figure}

A refinement of the term of the term N limitation (Table
\ref{tab:NutrientLimDefs}) is required by the introduction of the additional N
mineralization fluxes.
When microbes can meet their stoichiometric demand only by immobilizing
inorganic N I write Uptake N limitation.
When the immobilization flux cannot meet the stoichiometric requirements, I
write microbial N limitation. Despite the maximum immobilization flux there
might still be a net mineralization due to uptake mineralization and turnover
mineralization.
When there is a net immobilization, i.e. a net transfer from inorganic pool to
the organic pools of SOM and microbial biomass, I write SOM N limitation.

\begin{table}[t]
\caption{Kinds of nitrogen limitation \label{tab:NutrientLimDefs}}
%\vskip4mm
\centering
\begin{tabular}{lp{5.5cm}}
\tophline
Term &  Definition \\
\middlehline
Uptake N lim. & Nitrogen in microbial upatake is less than 
constrained by other elements (${\Phi_B < 0}$).
\\
Microbial N lim. & Nitrogen in microbial uptake plus potential
immobilization flux is less than constrained by other elements (${-\Phi_B <
u_{\operatorname{imm,Pot}}}$).
\\
SOM N lim. & There is a net transfer from the inorganic pool to
the organic pools (${\Phi_B+\Phi_u+\Phi_{\operatorname{tvr}}<0}$).
\\
\bottomhline
\end{tabular}
\end{table}
 


\subsection{How long are enzymes active?}
The turnover time of soil extracellular enzymes governs the timescale of priming
effects to a large extent. SEAM simulations showed a rather long time scale of
the priming effect \ref{fig:PrimingDec}) of several month, despite the $L$
resource pulse was used up within about 50 days. This is in contrast with
incubation studies that observe priming effects within days or weeks that
rapidly declines after the resource has been used up e.g.
\citep{Blagodatskaya14}.
Im this study the timescale was set by simulated enzyme turnover, which I
modelled using first order kinetics with a turnover of about a week
\citep{Burns13}. Moreover, the the positive feedback of increased microbial
biomass producing more enzymes retarded the decline of the priming effect.

One option to model the observed faster timescales, the turnover time of enzymes
could be decreased to a few days.
This change, however, would require an immense effort of producing enzymes by
microbial biomass. More sophisticated models of enzyme turnover
including stabilization of a part on mineral surfaces \citep{Burns13},
hopefully will be able to resolve such contradictions.

An alternative explanations is that enzymes such as peroxidases may need to be
fueled by labile organic matter themselves \citep{Rousk14} with no immediate
relationship to microbial biomass dynamics.
This explanation implies, however, that enzyme activity and decomposition of SOM
becomes decoupled from enzyme production and microbial dynamics to a large
extent in the short term.
This is contrary to the assumption of most models of the priming effect. While
such an explanation would greatly complicate comparison between model and data,
I argue that the model generated patterns of enzyme allocation and SOM turnover
are still robust in the long term and longer term activity is still determined
by enzyme production. However, the disagreement in timescale of the
priming effect does not impair the modelling of longer-term microbial community
controls on SOM cycling.

\subsection{Holisctic view for upscaling
\label{sec:Holistic}} 

The presented SEAM conceptual model takes a holistic view \citep{Panikov10}
of microbial community and their adaptations instead of explicitely
describing microbial diversity.
In this respect it differs from the SYMPHONY model \citep{Perveen14} and similar conceptual
models \citep{Fontaine03}, that explicitly modelled several microbial groups,
specifically SOM builders that grow solely on fresh low nutrient material (here
$L$), and SOM builders that can use both resources (here $L$ and $S$). The
effective behaviour of the presented SEAM model, however, is similar. It assumes
that community composition is driven to a large extent by external drivers
outside the community. The SEAM model describes a modified allocation of
resources into breakdown of different resources with changed resource
availability by assuming that the community composition adapts in a way so that
it will optimize its revenue.
While the merismatic approach of the SYMPHONY model explicitly represents this
optimization by shifts between microbial groups, the holistic approach
represents the effects of this optimization.
While the meristic approach gives more detailed understanding of the mechanism,
I hypothize, that the holistic approach is more suitable for simplification
\citep{Wutzler13} and application in larger-scale models, such as land surface
components of earth system models.
This implies that that the proposed abstraction of microbial competition by the
revenue strategy is a step forward of better representing couplings of soil
carbon and nutrient cycles in earth system models and a step forward to improved
predictions of long term carbon sequestration.

The holistic SEAM model mostly had qualitatively similar predictions as the
merismatic SYMPHONY model with simulating priming, the bank mechanism, and a
continuous SOM sequestration under high inorganic N inputs. It differed from
SYMPHONY in the prediction of the inorganic N pool during low N inputs. It
predicted a decrease in this pool while SYMPHONY predicted an increase in this
pool due to changed competition \citep{Perveen14}. The difference I think that
this difference is caused by different assumptions on how the DOM pool is shared
among groups of the microbial community and resulting different competition
conditions. In the SEAM model, decomposition products become mixed in a shared
DOM pool, while in the SYMPHONY model the decomposition products are not shared
between the microbial groups.
The truth probably in between, in that decomposition products are mainly used by
the group that is producing the extracellular enzymes, while a part of the DOM
diffuses also to other groups.
The difference in prediction implies that the rationality of the simplified
model assumptions can be qualitatively tested against observations. If the
holistic view is adequate for larger-scale modelling, it holds promise for
simplification and upscaling.

\subsection{Testable predictions: Change of SOM C/N ratios}
The SEAM model can be used to predict long term patterns of SOM cycling
after changes in resource stoichiometry. Observations of such pattern 
provide evidence for or against the modelling assumptions.
Specifically, I predicted a change in proportions of the litter pool and the
SOM pool. While these abstract pools are not directly comparable to observations
the overall effect is a change in total SOM C/N ratio. The C/N pool is
predicted to increase quite fast in all scenarios, but only with
the Renveue strategy it should it should still increase after a few decades of
FACE and unchanged N inputs/outputs (Fig. \ref{fig:CO2Increase}).
 
\subsection{Outlook}
The biggest limitation of the SEAM model is its focus on a single process:
enzyme allocation. In order to focus I had to ignore the second microbial
strategies of coping with resource stoichiometric imbalance by adapting their
biomass ratio. Although the potential of this biomass
adaptation is thought to be quite limited \citep{Mooshammer14}, it will be
tested whether these two strategies can be combined within a model.

Second, SEAM will be simplified by assuming quasi-steady state of biomass or
enzyme pools \citep{Wutzler13}. These simplifications will lead to fewer
parameters and improved identifiably in model calibration to observations.

Third, the optimality principle will be extended to determine the
proportion of uptake that is allocated to enzyme production. Presence of cheaters,
i.e. microbes that consume resource but without producing enzymes, effectively
lower the community-level allocation to enzymes \citep{Kaiser14}. Community
development can be assumed to maximise biomass production. Such an assumption can be
used to compute the optimal share of cheaters the allocation and allows
exploring effects on SOM cycling, such as more constrained carbon and nutrient
use efficiencies.

Eventually, influence of environmental factors needs to be modeled in order to
simulate global change effects other than changes in C and N inputs, such as
temperature and moisture \citep{Davidson12}.
 

\conclusions  
%% \conclusions[modified heading if necessary]
 
The allocation of resources into the production of different enzymes is an
important means of the microbial community to adapt to changing resource
stoichiometry. Allocation adaptation strategies helped microbial biomass to grow
larger biomass across a wider range of resource stoichiometry (Fig.
\ref{fig:VarNNoFeedback}). The strategy was particularly successful that took
into account both, the amount of resource and their stoichiometry.
The findings imply that models that want to simulate ecosystem balances and
mineralization fluxes of carbon and nutrients (Figs. \ref{fig:PrimingDec} and
\ref{fig:PrimingMin}) must take into account such adaption strategies.
Accounting for such strategies will be especially important, when simulating
competition of soil microorganism and plants for organic and inorganic nutrient
pools, as SOM may function as a storage to sequester surplus elements and
prevent them from leaving the system (Fig. \ref{fig:CO2Increase} and
\ref{fig:CO2IncreaseImb}).


The SEAM model provides a holistic description of community adaptations. It
yields qualitatively similar predictions such as microbial group explicit models
with the ability to represent priming effects, bank mechanism, and
a continuous sequestration with high inorganic N inputs (Fig. \ref{fig:pastureFit}).

Hence, this study provides an important step on providing an abstract
description of microbial community effects and adaptations, with the long-term
goal of including the important mechanisms into abstract earth system models.
Global modelers could already use the presented model to describe resource
stoichiometry effects. However, I hope to further abstract the model for larger
scales by a steady state assumption of the enzyme pools or the microbial
biomass, and by implementing optimality principles to determine the
overall amount of resource invested into enzyme production.




%% The Appendices part is started with the command \appendix;
%% appendix sections are then done as normal sections
%% \appendix

%% \section{}
%% \label{}

%% If you have bibdatabase file and want bibtex to generate the
%% bibitems, please use
%%
\bibliographystyle{elsarticle-harv} 
\bibliography{twutz_txt}

%% else use the following coding to input the bibitems directly in the
%% TeX file.
%\begin{thebibliography}{00}

%% \bibitem[Author(year)]{label}
%% Text of bibliographic item
%\bibitem[ ()]{}
%\end{thebibliography}
\end{document}
\endinput
%%
%% End of file `elsarticle-template-harv.tex'.
