%\documentclass[preprint,12pt,authoryear]{elsarticle}

%% Use the option review to obtain double line spacing
\documentclass[authoryear,preprint,review,12pt]{elsarticle}

%% main text
% twutz: some commands used in copernicus
\newcommand{\introduction}{\section{Introduction}}
\newcommand{\conclusions}{\section{Conclusions}}
\newcommand*\chem[1]{\ensuremath{\mathrm{#1}}}
\newcommand{\unit}[1]{\ensuremath{\mathrm{#1}}}

\usepackage{amsmath}


%% Use the options 1p,twocolumn; 3p; 3p,twocolumn; 5p; or 5p,twocolumn
%% for a journal layout:
%% \documentclass[final,1p,times,authoryear]{elsarticle}
%% \documentclass[final,1p,times,twocolumn,authoryear]{elsarticle}
%% \documentclass[final,3p,times,authoryear]{elsarticle}
%% \documentclass[final,3p,times,twocolumn,authoryear]{elsarticle}
%% \documentclass[final,5p,times,authoryear]{elsarticle}
%% \documentclass[final,5p,times,twocolumn,authoryear]{elsarticle}

%% For including figures, graphicx.sty has been loaded in
%% elsarticle.cls. If you prefer to use the old commands
%% please give \usepackage{epsfig}

%% The amssymb package provides various useful mathematical symbols
\usepackage{amssymb}
%% The amsthm package provides extended theorem environments
%% \usepackage{amsthm}

%% The lineno packages adds line numbers. Start line numbering with
%% \begin{linenumbers}, end it with \end{linenumbers}. Or switch it on
%% for the whole article with \linenumbers.
\usepackage{lineno}

\journal{Soil Biology & Biochemistry}

\begin{document}

\begin{frontmatter}

%% Title, authors and addresses

%% use the tnoteref command within \title for footnotes;
%% use the tnotetext command for theassociated footnote;
%% use the fnref command within \author or \address for footnotes;
%% use the fntext command for theassociated footnote;
%% use the corref command within \author for corresponding author footnotes;
%% use the cortext command for theassociated footnote;
%% use the ead command for the email address,
%% and the form \ead[url] for the home page:
%% \title{Title\tnoteref{label1}}
%% \tnotetext[label1]{}
%% \author{Name\corref{cor1}\fnref{label2}}
%% \ead{email address}
%% \ead[url]{home page}
%% \fntext[label2]{}
%% \cortext[cor1]{}
%% \address{Address\fnref{label3}}
%% \fntext[label3]{}

\title{Adaptation of soil microbial resource allocation affects long term soil
organic matter and nutrient cycling.}

%% use optional labels to link authors explicitly to addresses:
%% \author[label1,label2]{}
%% \address[label1]{}
%% \address[label2]{}

\author{Thomas Wutzler}

\address{Max Planck Institute for Biogeochemistry, Hans-Kn{\"o}ll-Stra{\ss}e 10,
07745 Jena, Germany }

\begin{abstract}
% In order to understand the coupling of global (C) and nitrogen (N) cycles, it
% is
necessary to understand C and N use efficiencies of microbial soil organic
matter (SOM) decomposers. While important controls of those efficiencies by
microbial community adaptations have been shown at pore scale, an abstract
simplified representation of community adaptations is needed at ecosystem scale.
Therefore we developed the soil extracellular enzyme allocation model (SEAM)
that described C and N dynamics at ecosystem scale at weeks to decades. We
explicitely modeled several alternative community adaptation strategies of
resource allocation to extracellular enzymes and enzyme limitations on SOM
decomposition. Using SEAM, we explored the effects of those alternative
strategy-hypotheses on SOM and inorganic N cycling.
Simulations showed that the revenue enzyme allocation strategy was most
competitive, which accounted for adaptations to both, stoichiometry and amount
of different SOM resources.
Predictions of the holistic SEAM model were qualitatively similar to microbial
group explicit models with the ability to represent priming effects, and the
buildup or decline of SOM pools depending on inorganic N inputs.
We found that soil enzyme allocation strategies strongly affect long term soil
organic matter cycling and nutrient recycling.
The findings imply that ecosystem scale models should account for adaptation of
C and N use efficiencies to represent and understand C-N couplings. The
combination of stoichiometry and optimality principles is a promising route to
yield simple formulations of such adaptations at community level suitable for
incorporation into land surface models.

\end{abstract}


\begin{keyword}
%% keywords here, in the form: keyword \sep keyword

%% PACS codes here, in the form: \PACS code \sep code

%% MSC codes here, in the form: \MSC code \sep code
%% or \MSC[2008] code \sep code (2000 is the default)

\end{keyword}

\end{frontmatter}

%% \linenumbers



\linenumbers

\introduction 
% % \introduction[modified heading if necessary] 
The global element cycles of carbon (C) and nitrogen (N) are intrically linked
and cannot be understood without their interactions 
\citep{Thornton07,Janssens10, Zaehle11}%TODO uncomment but latex levels fail
. 
The links between nutrient cycles are especially strong
in the dynamics of soil organic matter (SOM) because all of the SOM has to be
depolymerized and successively mineralized by a microbial community with a
rather strict homeostatic regulation \citep{Sterner02, Zechmeister15}.

Faced with imbalances between stoichiometry of SOM and stoichiometric
requirements of decomposers, decomposers have three options
\citep{Mooshammer14}.
First, they can lower their carbon use efficiency (CUE) or nutrient use
efficiency (NUE) \citep{Sinsabaugh13}. The alteration of CUE has shown to have
large consequences on prediction of carbon sequestration in SOM
\citep{Allison14a, Wieder13}.
Regulation of NUE has consequences for nutrient recycling and loss of nutrients
from the ecosystem \citep{Mooshammer14a} and soil plant feedback
\citep{Rastetter11}.
As a second option, decomposer community can adapt their stoichiometric
requirements. Community composition can shift between species with high C/N
ratio, such as many fungi, or species with lower C/N ratio, such as many
bacteria \citep{Cleveland07, Xu13}, although the flexibility is very narrow.
As a third option, decomposers can adapt their allocation of resources into
production of extracellular enzymes to preferentially degrade fractions of SOM
that differ by their stoichiometry \citep{Moorhead12}.


Representation and consequences of stoichiometry on element cycling differ
between models at different scales. Most soil models at ecosystem scale
employ the first option and use CUE or NUE to represent stoichiometric controls
on respiration and mineralization fluxes \citep{Manzoni08}. However, modelling
studies at the pore scale, which is relevant to microbes, show the
important effect of community adaptation and their emerging effects on element cycling
\citep{Allison05, Resat11, Wang13}. Competition among several
microbial groups that differed in expression of different enzymes resulted in a
simulated approximately equally high CUE across a wider range of litter
stoichiometry \citep{Kaiser14}. Hence, there is a need to capture the effects of community
adaptation also in models at ecosystem scale.

Effects of microbial diversity can be represented at ecosystem scale by
at least two options. First, explicit representation of competition of
several microbial populations can represent stoichiometric effects such as
sustained sequestration of N with high N inputs can be simulated \citep{Perveen14}.
Second, effective properties of the entire microbial community can be
represented in a dynamic adaptive way \citep{Rastetter97, Rastetter11}.

The adaptation of enzyme production, was recently
formalized by the conceptual EEZY model \citep{Moorhead12}. While this model
showed the principle feasibility of this strategy in short term, it did not
implement feedback to resource pools and therefore could no look at
consequences for longer term SOM cycling. 

My task, therefore, was to tackle the need of capturing the effects of adaptive
enzyme production by building on the EEZY model to explore the different
consequences of alternative enzyme allocation schemes on long term SOM dynamics
and nutrient recycling.

This paper first introduces the SEAM model (Section \ref{sec:SEAM}), a
conceptual model of SOM cycling that explicitly represents microbial strategies
(Section \ref{sec:AllocStrategies}) of producing several extracellular enzyme
pools. Next, the effect of those strategies on SOM cycling is presented by
prototypical examples (Sections \ref{sec:SimScen} and \ref{sec:ResultsProto}.
Finally, a calibration to an intensive pasture cite (Section
\ref{sec:methodsPasture}) demonstrates the usability of the model (Section
\ref{sec:ResultsPasture}) and compares its predictions to predictions of the
microbial-group explicit Symphony model \citep{Perveen14}.

This study shows that different decomposers enzyme allocation strategies
have large consequences on long term SOM dynamics and nutrient recycling. It
proposes a holistic scheme to represent effects of microbial adaptation of
enzyme production on SOM cycle at the ecosystem scale.


\section{Methods}
\subsection{Soil Enzyme Allocation Model (SEAM)}
\label{sec:SEAM}

Development of the conceptual structural dynamic Soil Enzyme Allocation Model
(SEAM) allowed to explore consequences of enzyme allocation strategies for SOM
cycling at the soil core scale. The modelled system are C and N
pools in SOM in a representative elemental volume of soil. The
system could be soil of a laboratory incubation or a soil layer.
The model represents different pools of C and N as state variables
and specifies dynamic equations for the mass fluxes. It is driven by inputs of
plant litter (both above ground and rhizodeposition) and inorganic N
inputs, as well as prescribed uptake of inorganic N by roots.  It
computes output fluxes of heterotrophic respiration and leaching of inorganic
N.

Key features are first, the representation of several SOM pools that differ by
their stoichiometry, i.e. elemental ratios, and second, the representation of
specific enzymes that degrade those SOM pools. The quality spectrum is modelled by two
classes: a C rich more easily degradable labile litter pool,
$\operatorname{L}$, and a N rich more recalcitrant pool that consists
mainly of microbial residues, $\operatorname{R}$ (Fig. \ref{fig:SEAMStruct}).

\begin{figure}[t] \vspace*{2mm}
\begin{center}
\includegraphics[width=8.3cm]{fig/seam2}
\end{center}
\caption{
Model structure of SEAM: Two resource pools ($L$ and $R$) which differ in their
elemental ratios are depolymerized by respective enzymes ($E_L$ and $E_R$). The
simple organic compounds ($\operatorname{DOM}$) are taken up by the microbial
community  and used for synthesizing new biomass (${B}$), new enzymes, for
catabolic respiration, and an apparent N mineralization during uptake
($\Phi_u$). Stoichiometric imbalance between $\operatorname{DOM}$ and ${B}$
causes overflow respiration or mineralization of excess N ($\Phi_B$) to
inorganic N ($I$).
During microbial turnover an additional part of the biomass is mineralized
($\Phi_{\operatorname{Tvr}}$).
Boxes correspond to pools, black arrow heads to mass fluxes, white arrow heads
to other controls, and disks represent partitioning of fluxes. Solid lines
represent fluxes of both C and N, while dotted and dashed lines represent
separate C or N fluxes respectively.
\label{fig:SEAMStruct}
}
\end{figure}

The most important assumptions are described in the following paragraphs, while
the symbols are explained in Tab. \ref{tab:modelParameters} and detailed model
equations are given in Appendix \ref{app:SEAM}. 

% \section{\\ \\ \hspace*{-7mm}  Symbols and Values \label{app:symbols}}    %%
% % Appendix
% 
% Table \ref{tab:modelParameters} lists symbols used in the model together with
% their values as applied in the model calibration. It also reports initial
% values of the state variables for the grassland calibration. 

\begin{table}[t]
\caption{
\label{tab:modelParameters}
Model parameters and drivers. The first value column refers to the
prototypical example while the subsequent column reports values that
differ for the grassland calibration (TODO adjust). }
\vskip4mm
\centering
\begin{tabular}{lp{6cm}lllp{5.5cm}}
\tophline
Symbol &  Definition & \multicolumn{2}{c}{Value} & Unit & Rational \\
\middlehline
\multicolumn{6}{l}{State variables}  \\
$L$ &  C in litter & & 571 & \unit{gm^{-2}} & quasi steady state 
\\
$L_N$ &  N in litter & & 815 & \unit{gm^{-2}} & \citep{Perveen14}
(by their N/C ratio $\beta$)
\\
$R$ &  C in residue substrate & & 10500 & \unit{gm^{-2}} &
\citep{Allard07} (total stocks - L - dR)
\\
$R_N$ &  N in residue substrate & & 968 & \unit{gm^{-2}} & by C/N
ratio in \citep{Perveen14} \\
$E_L$ &  C in enzymes targeting $L$ & & 0.34 & \unit{gm^{-2}} & 
quasi steady state \\
$E_R$ &  C in enzymes targeting $R$ & & 0.21 & \unit{gm^{-2}} & 
quasi steady state \\
$B$ & microbial biomass C & & 89.2 & \unit{gm^{-2}} &  quasi steady
state \\
$I$ & inorganic N & & 2.05 & \unit{gm^{-2}} & \citep{Perveen14} \\
\\
\multicolumn{6}{l}{Model parameters}  \\
$\beta_B$ &  C/N ratio of microbial biomass & 11 & & \unit{gg^{-1}} &
Perveen 2014 
\\
$\beta_E$ &  C/N ratio of extracellular enzymes & 3.1 & &
\unit{gg^{-1}} & Sterner 2002 \\
$\beta_{\mathrm{input}_L}$ &  C/N ratio of plant litter inputs & 30 & 70 &
\unit{gg^{-1}} & \citep{Perveen14} ($1/\beta$) \\
$k_R$ &  maximum decomposition rate of $R$ & 1 & 4.39e-2 & \unit{yr^{-1}}
& calibrated \\
$k_L$ &  maximum decomposition rate of $L$ & 5 & 1.95 & \unit{yr^{-1}}
& calibrated \\
$k_N$ &  enzyme turnover rate &  60  & & \unit{yr^{-1}} & \citep{Burns13} \\
$k_{NB}$ & enzyme turnover entering DOM rather than $R$ & 
0.8 & & (-) & mostly small proteins \\
$a_{E}$ &  enzyme production per microbial biomass & 0.365 & &
\unit{yr^{-1}} & $\approx 6\%$ of biomass synthesis \\ 
$K_{M}$ &  enzyme half saturation constant & 0.05 & &
\unit{gm^{-2}} & magnitude of DOC concentration \\
$\tau$ &  microbial biomass turnover rate & 6.17 & & \unit{yr^{-1}} &
\citep{Perveen14} ($s/\epsilon_{\operatorname{Tvr}}$) \\
$m$ & specific rate of maintenance respiration & 1.825 & 0 & 
\unit{yr^{-1}} & \citep{Bodegom07}, zero in \citep{Perveen14} \\
$\epsilon$ & anabolic microbial C substrate efficiency & 0.5 & 0.53 &
(-) & calibrated \\ %\citep{Perveen14} ($s/(s+r)$ +2\% enzymes) \\
$\nu$ & aggregated microbial organic N use efficiency & 0.7 &
0.9 & (-) & Manzoni 2008 \\
$\epsilon_{\operatorname{Tvr}}$ & microbial turnover that is not
mineralized & 0.3 & 0.8 & (-) & part of turnover is consumed by
predators
\\
$i_{B}$ & maximum microbial uptake rate of inorganic N & 25 & & \unit{yr^{-1}} 
& larger than simulated immobilization flux \\
$l$ & inorganic N leaching rate & 0 & 0.959 &
\unit{yr^{-1}} & \citep{Perveen14} ($l)$ \\
\\
\multicolumn{6}{l}{Model drivers, i.e. fluxes across system boundary}  \\ 
$\mathrm{input}_{L}$ & litter C input & & 969.16 &
\unit{gm^2yr^{-1}} & \citep{Perveen14} \, ($m_p C^{obs}_p$)\\
$i_{I}$ & inorganic N input & & 22.91 & \unit{gm^2yr^{-1}} 
& \citep{Perveen14} \\
$k_{IP}$ & inorganic plant N uptake & & 16.04 & 
\unit{gm^2yr^{-1}} & \citep{Perveen14} (assuming plant
steady state: plant N export + litter N input)\\
\\
\multicolumn{6}{l}{Fluxes and quantities derived within the system}
\\
$\alpha$ & proportion of enzyme investments allocated to production of 
$E_R$ & & & (-) &
\\
$\operatorname{syn}_B $ & C for microbial biomass synthesis &  &
& \unit{gm^2yr^{-1}} &
\\
$\operatorname{syn}_E $ & C for enzyme synthesis &  &
& \unit{gm^2yr^{-1}} & \\
$\operatorname{tvr}_B $ & microbial biomass turnover C &  &
& \unit{gm^2yr^{-1}} & \\
$\operatorname{tvr}_E $ & enzyme turnover C &  &
& \unit{gm^2yr^{-1}} & \\
$\operatorname{dec}_X $ & C in decomposition of resource X ($X$ is $L$ or $R$)
& & & \unit{gm^2yr^{-1}} & \\
$u_C,u_N$ & microbial uptake of C and N  & &
& \unit{gm^2yr^{-1}} & \\
$\Phi_u, \Phi_B, \Phi_{\operatorname{tvr}}$ & N mineralization with microbial
DOM uptake, stoichiometric imbalance, and turnover & &
& \unit{gm^2yr^{-1}} & see Fig. \ref{fig:SEAMStructNFluxes}
\\
\bottomhline
\end{tabular}
\end{table}


 

C/N ratios, $\beta$, of biomass and enzymes are
fixed while those of the resource pools may change over time due to
fluxes in and out of the pools. Imbalances in stoichiometry of uptake and
microbial requirements are compensated by overflow respiration or N
mineralization.
Total enzyme allocation is a fixed fraction, $a_E$, of the microbial biomass,
$B$, per time. But microbial community can use different strategies
(Section \ref{sec:AllocStrategies}) to adjust their allocation to produce new
enzymes.  
The DOM pool is assumed to be in quasi steady state and all the sum of all
influxes to the DOM pool (decomposition + part of the enzyme turnover) is taken
up by the microbial community. If expenses for maintenance and enzyme production
cannot be met, the biomass starves and declines.

\subsection{ Enzyme allocation strategies} 
\label{sec:AllocStrategies}

Three different strategies of allocating investments among production of
alternative enzymes were explored in this study (Table
\ref{tab:AllocStrategies}). Microbes allocate a proportion $\alpha$ of total
enzyme investments, $a_e, B$, to the production of enzymes targeting the N rich
$R$ resource and proportion $1 - \alpha$ to the production of enzymes targeting
the N poor but better degradable $L$ resource: $\operatorname{alloc}_{E_R} /
(\operatorname{alloc}_{E_R} + \operatorname{alloc}_{E_L}) = \alpha$.

\begin{table}[t]
\caption{Microbial enzyme allocation strategies \label{tab:AllocStrategies}}
\vskip4mm
\centering
\begin{tabular}{ll}
\tophline
Strategy &  Allocation is \\
\middlehline
Fixed & independent, constant \\
Match & adjusted to achieve balanced growth \\
Revenue & proportional to return per investments into enzymes \\
\bottomhline
\end{tabular}
\end{table}


The \textbf{Fixed} strategy assumes that allocation is independent
and not changing with changes in substrat availability.
\begin{equation} 
\label{eq:allocFixed}
\alpha = \operatorname{const.}
\end{equation}
This strategy corresponds to the models where decomposition rate is a function
of microbial biomass \citep{Wutzler08}.
 
The \textbf{Match} strategy assumes that microbes regulate enzymes production in
a way that decomposition products balance their stoichiometric demands
\citep{Moorhead12}.
The partitioning coefficient is derived by equating the C/N ratio of the sum of
uptake fluxes after other expenses to the C/N ratio of microbial biomass,
$\beta_B$.

\begin{equation} 
\label{eq:allocMatchCN}
\frac{\epsilon (\operatorname{dec}_L + \operatorname{dec}_R - r_M)}{
\operatorname{dec}_L/\beta_L + \operatorname{dec}_R/\beta_R  - \Phi_M } =
\beta_B
\text{,}
\end{equation}
where $\operatorname{dec}_L$, and $\operatorname{dec}_R$ are depolymerization
fluxes of the labile and residue resources respectively, $r_M$ is maintenance
respiration, $\epsilon$ is the anabolic microbial efficiency (\ref{eq:synB}),
and $\beta_i$ are C/N ratios of the respective pools $i$, and $\Phi_M$ is the
net flux of N from living microbes to the mineral N pool.
Here, I assume that microbes use the maximal immobilization of inorganic N,
$u_{\operatorname{imm,Pot}}$ (\ref{eq:uN}), to meet their stoichiometric
requirements with the Match strategy. Hence, the net N flux is a difference
between mineralization during uptake and the immobilization: $\Phi_M = \Phi_u -
u_{\operatorname{imm,Pot}}$. With microbial N limitation,
(\ref{eq:allocMatchCN})  has no solution, enzymes are allocated entirely to
the N-rich resource ($\alpha=1$), and excess carbon is
respired by overflow respiration.

When the current sum of enzyme levels, $E$ is assumed to be in quasi
steady state for given amounts of resource and microbial biomass, then equation
\ref{eq:allocMatchCN} can be solved for allocation partitioning, $\alpha$.
\begin{subequations}
\label{eq:allocMatch} 
\begin{align}
\alpha_M &= f_{\operatorname{{\alpha}Fix}}(L,\beta_L,R,\beta_R, E_L + E_R,
\Phi_M)
\\
\alpha &= \begin{cases}
  0,  & \text{if} \alpha_M \le 0 \\
  1,  & \text{if} \alpha_M \ge 1 \\
  \alpha_M, & \text{otherwise}
\end{cases}   
\end{align}
\end{subequations} 

\noindent
The bound to one is necessary to handle the case of microbial N
limitation, and the bound to zero corresponds to the theoretical case where the
C rich resource may not suffice to cover microbial C demands. 
Function $f_{\operatorname{{\alpha}Fix}}$ is given in appendix
\ref{app:fAlphaFix} and the SYMPY script of its 
derivation is given with supplementary material. 

The \textbf{Revenue} strategy assumes that microbial community adapts in a way
so that investment into enzyme production is proportional to their revenue, i.e.
their return per investment regarding the currently limiting element: 
\begin{subequations}
\label{eq:allocRev}
\begin{align}
\alpha_C &= \frac{\operatorname{rev}_{RC}}{\operatorname{rev}_{LC} + \operatorname{rev}_{RC}} 
\\
\alpha_N &= \frac{\operatorname{rev}_{RN}}{\operatorname{rev}_{LN} + \operatorname{rev}_{RN}} 
\text{,} 
\end{align}
\end{subequations}
where $\operatorname{rev}_S$ is the revenue from given resource $S$ ($S$ is
either $L$ or $R$) under C and N limitation respectively. The return is the
current decomposition flux from the resource degraded by the respecitve enzyme, and the investment is assumed to be
equal to enzyme turnover to keep current enzyme levels, $E_S^*$.
% : $a_E B = k_{NR} E_R^* + k_{NL} E_L^*$.
\begin{subequations}
\label{eq:allocRev2}
\begin{align}
\operatorname{rev}_{SC} &= \frac{\text{return}}{\text{investment}} 
= \frac{\operatorname{dec}_{S,Pot} \frac{E_S^*}{k_{m,S} + E_S^*}} {k_{NS}E_S^*} 
= \frac{\operatorname{dec}_{S,Pot}} {k_{NS}(k_{m,S} + E_S^*)} \\ 
\operatorname{rev}_{SN} &= \frac{\operatorname{dec}_{S,Pot}
\frac{E_S^*}{k_{m,S} + E_S^*} / \beta_S} {k_{NS} E_S^* / \beta_E} 
= \frac{\operatorname{dec}_{S,Pot}}{k_{NS} (k_{m,S} + E_S^*)} 
\frac{\beta_E}{\beta_S}
\text{,} 
\end{align}
\end{subequations}
where $k_{NS}$ is rate of enzyme turnover, $k_{m,S}$ is enzyme's resource
affinity, $a_E$ is total enzyme allocation coefficient, $\operatorname{dec}_{S,Pot}$ is
enzyme saturated decomposition flux (\ref{eq:dec}), and $\beta$ are C/N ratios
of the respective pools.

There are two partitioning coefficients, $\alpha_C$ and $\alpha_N$ with C or N
limited microbial biomass repectively. In order to avoid frequent large jumps
under near co-limitation, SEAM implements a smooth transition between these two
cases as a weighted average.

\begin{equation}
\label{eq:allocRev3}
%\begin{align}
\alpha = \frac{w_{\operatorname{CLim}} \alpha_C + w_{\operatorname{NLim}}
\alpha_N}{w_{\operatorname{CLim}}  + w_{\operatorname{NLim}} } 
\text{,} 
%\end{align}
\end{equation}
where $w$ are strength of the limitation of the respective element, specifically
the ratio of required to available biomass synthesis fluxes
(\ref{eq:weightsLim}).


\subsection{ Prototypical simulation scenarios} 
\label{sec:SimScen}

Several prototypical simulation scenarios (Table \ref{tab:SimScen}) were used to
explore consequences of different microbial enzyme allocation strategies for SOM
dynamics. All scenarios used parameter values given in Table
\ref{tab:modelParameters}, if not stated specifically otherwise. The
scenarios use a low minimum turnover time of the residue SOM pool of 10 years in
order to demonstrate and total changes together with changes of the faster
litter pool. Inorganic N pool was kept steady at $I=0.4 \mathrm{gN}$, here,
while a inorganic N feedback is explored in section \ref{sec:methodsPasture}.

\begin{table}[t]
\caption{Prototypical simulation scenarios \label{tab:SimScen}}
\vskip4mm
\centering
\begin{tabular}{lp{5.3cm}}
\tophline
Scenario & Explored issue\\
\middlehline
VarN-Incubation & Efficieny of using given fixed resource levels that
vary by N content \\
Feedback-Steady & Possibility and size of steady state resource pools\\
Priming & Increased resource decomposition and mineralization after
addition of fresh litter\\
CO2-Fertilization & SOM dynamics after increased carbon inputs\\
\bottomhline
\end{tabular}
\end{table}

The \textbf{VarN-Incubation} scenario explored how resources of given
stoichiometry are used more or less efficiently with different enzyme allocation
scenarios. It used a simplified model where all the inputs and feedback to the
resource pools ($L$ and $R$) and to the inorganic N pool ($I$) were neglected
and these pools were kept constant ($dL/dt = dR/dt = 0$). It allowed microbes
and enzyme levels to develop to a quasi steady state with the given resource
supply. Hence, it simulated the end of a short term incubation. Specifically, it
used fixed resource carbon of $L=100 \mathrm{gCm}^{-2}$, $R=400 \mathrm{gCm}^{-2}$.
It set C/N ratio of the residue pool to $\beta_R=7$, and varied the litter
C/N ratio ($\beta_L = [18,..,42]$).

The \textbf{Feedback-Steady} scenario explored the long term trajectories of the
entire system including feedback to the substrate pools. Specifically, it set
litter input to $\operatorname{input}_L = 400\mathrm{gCm}^{-2}\mathrm{yr}^{-1}$
with C/N ratio $\beta_{\operatorname{input}_L} = 30$.

The \textbf{Priming} scenario explored the effect of rhizosphere priming, i.e
the input of fresh carbon into a bulk subsoil. It looked at the fluxes after
an addition of $50g$ of litter on a soil that otherwise
received a litter input of only $30\mathrm{gCm}^{-2}\mathrm{yr}^{-1}$ for a
decade. It simulated the amendment of a very available substrate, specifically
with a maximum turnover of $k_L = 10\mathrm{day}^{-1}$.

The \textbf{CO2-Fertilization} scenario explored the effect of continuous 
increased carbon litter input that is expected with elevated atmospheric CO2.
It started from steady state for a litter input, applied 20\% increased
carbon inputs during years 10 to 60, and applied original carbon inputs again
during the next 50 years.

\subsection{Calibration to a fertilized grassland site}
\label{sec:methodsPasture}

To test the capability of the SEAM model to simulate the carbon sink of a
grassland site, I calibrated the model using the revenue strategy to data of an
intensive pasture.
The model drivers and most of the parametrization was taken from the publication
of \citep{Perveen14}. The site is a temperate permanent grassland located at an
altitude of 1040m in France (Laqueuille, 45\textdegree{38}'N,
2\textdegree{44}'E) and has an annual precipitation and temperature of 1200~mm
and 7 \textdegree{C}, respectively.

The N balance of the fertilized grassland is characterized by high inorganic
N-inputs. Part of this N is sequestered in accumulating SOM, part is leached and
part is exported with plant biomass. In all scenarios, plant uptake of inorganic N was
computed assuming the plants to be in steady state with the litter
production and biomass exports.
 
Parameters were chosen corresponding to Table 1 in \citep{Perveen14}. Three
parameters were calibrated: the maximum decomposition rates of substrate pools,
$k_L$ and $k_R$, and the anabolic carbon use efficiency, $\epsilon$. Initial
pools were prescribed to observed values. Intial pools of microbial biomass and
enzymes were set to the long-term state after a preliminary optimization in
order to prevent large initial fluctuations. The calibrations used the
\textit{optim} function from R \textit{stats} package \citep{R07} to minimize
the differences between model predictions and observations normalized by the
standard deviation of the observations. It used observations of the labile
OM, the inorganic N, leaching, and rate of change of the recalictrant pool.
 
\section{Results}

First, the results of several prototypical artificial simulation scenarios
clarify the general behaviour and features of the SEAM model. Next, results of a
parameter calibration demonstrates the model's ability to
simulate the observed C and N dynamics of an intensive pasture.

\subsection{Prototypical simulation scenarios}
\label{sec:ResultsProto}

With the \textbf{VarN-Incubation} scenario, there were differences among
allocation strategies for the dependence of allocation $\alpha$ on the N
content of the litter resource. They caused marked changes in biomass and
imbalance fluxes (Fig. \ref{fig:VarNNoFeedback}).
 
\begin{figure}[t] \vspace*{2mm}
\begin{center}
\includegraphics[width=8.3cm]{fig/VarNNoFeedback}
\end{center}
\caption{
Match enzyme allocations strategy yielded highest resource efficiency, i.e.
lowest mineralization fluxes with the VarN-scenario. Microbes with alternative
strategies, however, competed better indicated by a higher biomass. Carbon use
efficiency (CUE) and C/N ratio of the decomposition flux (cnDOM) helped to
explain the different patterns.
\label{fig:VarNNoFeedback}}
\end{figure}

The Match strategy allowed balanced growth and efficient resource usage, but
yielded less biomass than the other scenarios. When the litter contained enough
N, microbes invested all ressources into litter degrading enzymes.
Across a wide range of litter C/N ratios (22 to 42) microbes did not need
stoichiometric imbalance fluxes, i.e. mineralization of excess N
or overflow respiration of excess C.  

With the Revenue strategy enzyme allocation varied with litter N content too.
With litter containing enough N (low C/N ratio), still about 5\% of the enzyme C
expenditures were allocated into R degrading enzymes. This resulted in higher
mineralization of excess N, but in turn allowed for a higher microbial biomass.
With litter lacking enough N (higher C/N ratios), investment into R-degrading enzymes
increased to about 30\%, a value that was much lower than with the the Match
strategy. Hence, the Revenue strategy yielded higher C overflow respiration
associated with low carbon use efficiency. However, at the same time, it
yielded higher levels of microbial biomass.

The Fixed strategy yielded higher N-mineralization at low C/N ratios. At high
C/N ratios its allocation was intermediate between the other strategies leading
to intermediate values of all the other outputs.

% Threshold elemental ratio slightly increased with higher litter C/N ratio, as
% can be seen in the C/N ratio of the DOM in the match strategy (Fig.
% \ref{fig:VarNNoFeedback}). The reason for this increase, here, was a lower
% proportion of uptake flux compared to immobilization flux with lower biomass.

With the \textbf{SimSteady} scenario, which included feedback to the
resource pools, both Fixed and the Revenue strategies caused
resource pools to approach a steady state.
However, the microbes with Match strategy solely degraded the
stoichiometrically better matching high-N residue pool, $R$ (Fig.
\ref{fig:SimSteady}). Hence, they declined together with the R residues
pool despite the large amoung of N accumulating in the stoichiometrically less
unfavourable litter pool (Fig. \ref{fig:SimSteady}).

\begin{figure}[t]
\vspace*{2mm}
\begin{center} 
\includegraphics[width=8.3cm]{fig/SimSteady} 
\end{center}
\caption{Match strategy was not viable when considering feedback to
substrate pools with the SimSteady scenario, where microbes degraded a
stoichiometrically matching but depleted R substrate pool.
\label{fig:SimSteady}} 
\end{figure}

Because of the Match strategy was not able to simulate reasonable
stocks when including feedback to resource pools in the model, it was omitted
in the following simulation scenarios.

With the \textbf{Priming} scenario, where a starved soil was amended with a
pulse of litter, a clear real priming effect was simulated. The priming effect
is by a strongly enhanced decomposition of the soil residues pool (Fig.
\ref{fig:PrimingDec}). It was stronger with the Revenue strategy than with the
Fixed strategy. This stronger priming was mostly due to a higher microbial
biomass with the Revenue strategy. Therefore also the N-mineralization flux due
to microbial turnover was larger with the Revenue strategy (Fig.
\ref{fig:PrimingMin}).

\begin{figure}[t]
\vspace*{2mm}
\begin{center}
\includegraphics[width=8.3cm]{fig/PrimingDec}
\end{center}
\caption{Residue depolymerization flux increased strongest with the Revenue
strategy after amending a starved subsoil with a pulse of fresh litter with the
Priming scenario.
\label{fig:PrimingDec}}
\end{figure}

\begin{figure}[t]
\vspace*{2mm}
\begin{center}
\includegraphics[width=8.3cm]{fig/PrimingMin}
\end{center}
\caption{Priming a low-input soil by a fresh litter pulse stimulated N
mineralization most strongly with the Revenue strategy with the Priming scenario.
\label{fig:PrimingMin}}
\end{figure}

With the \textbf{CO2-Fertilization} scenario, enzyme allocation strategies
yielded most marked differences between strategies in both SOM stocks (Fig.
\ref{fig:CO2Increase}) and nutrient recycling (\ref{fig:CO2IncreaseImb}). While
litter stocks $L$ increased with both scenarios, the residues stock $R$
slightly increased with the Fixed strategy but declined with the Revenue strategy. In
addition N mineralization was much stronger with the Revenue scenario during
enhanced CO2 period, with largest contribution from mineralization by microbial
turnover.

\begin{figure}[t] \vspace*{2mm}
\begin{center}
\includegraphics[width=8.3cm]{fig/CO2Increase}
\end{center}
\caption{
Revenue strategy led to a depletion of SOM and liberation of nutrients, i.e.
nutrient mining, during increased carbon litter inputs
in years 10 to 60 with the CO2-Fertilization scenario.
\label{fig:CO2Increase}}

\end{figure}
\begin{figure}[t] \vspace*{2mm}
\begin{center}
\includegraphics[width=8.3cm]{fig/CO2IncreaseImb} 
\end{center}
\caption{
Mineralization of microbial turnover with the Revenue strategy
contributed most of the liberation of SOM-N during CO2-Fertilization starting
at year 10.
At the end of the fertilization at year 60, mineralization due to stoichiometric
imbalance was smaller with the Revenue strategy.
\label{fig:CO2IncreaseImb}}
\end{figure}

\subsection{Intensive pasture simulation}
\label{sec:ResultsPasture}

The SEAM model successfully simulated the C and N balance of the Laqueuille
intensive pasture (Figure \ref{fig:pastureFit} left). The observed continuous
buildup of a pool of organic N in the recalcitrant SOM was driven by the systems
positive N balance. It was simulated with SEAM by two pathways. First, inorganic
N was taken up by the plant and supplied via organic N in litter, and second,
microbial biomass immobilized inorganic N due to stoichiometric imbalance of
substrate. The microbial biomass was N-limited when looking at the organic
substrate only. However, it was C-limited when taking into account
immobilization of inorganic N.

\begin{figure}[t] \vspace*{2mm}
\begin{center}
\includegraphics[width=8.3cm]{fig/pastureFit} 
\end{center}
\caption{
Calibrated Seam model predictions matched observed values of the Laqueuille
intensive pasture (dots and errorbars left). Altered C and N inputs shifted
enzyme allocation ($\alpha$) and affected SOM turnover (right).
Increased N substrate limitation, either due to elevated CO2 or due to
decreasing inorganic N inputs, caused an increase in labile pool, $L$, and a
decrease in mineral N pool, $I$. If the substrate N limitation could not be
balanced by inorganic N input, then the change rate of the recalcitrant pool,
$dR$, decreased down to negative values, i.e. losses.
\label{fig:pastureFit}
}
\end{figure}   

Altered C and N inputs to the system strongly affected the internal SOM and
nutrient cycling with several simulation scenarios (Figure \ref{fig:pastureFit}
right).

\textit{Increased litter C input} by 50\% together with an increased litter C/N
ratio by 25\% caused a shift in enzyme allocation towards enzymes degrading the
N-rich recalcitrant litter and an increase of the labile litter pool. It also
increased the demand for mineral N, both for the plant to balance increased
biomass production and for microbial biomass with higher stoichiometric
imbalance. The resulting decrease in mineral N also decreased leaching losses.
Increased N efficiency within the system was additionally favoured by a faster
recycling of N due to increased microbial activity.

\textit{Decreased inorganic N inputs} from 22.9
${\textrm{gm}^{-2}\textrm{yr}^{-1}}$ down to 1
${\textrm{gm}^{-2}\textrm{yr}^{-1}}$ together with an doubling of litter C/N
ratio caused a strong shift in enzyme allocation towards enzymes degrading the
N-rich recalcitrant SOM with similar consequences as with increased C input,
such as an increase in labile SOM. However, here, the decreased N inputs caused
a depletion of the mineral N pool.
As a consequence, the microbial biomass could not use immobilization to
balance substrate stoichiometry and became N-limited.
This caused overflow respiration and a decreasing trend in recalcitrant SOM.

\textit{Increased inorganic N inputs} from 22.9 $gm^{-2}yr^{-1}$ up to 25.6
$gm^{-2}yr^{-1}$ together with decrase of litter C/N by 25\% did not
much affect the system behaviour because the soil system was already C-limited
before.
The microbes could only immobilize a small part of the additional N for building
up SOM. Instead, N accumulated in the inorganic pool with associated
increased leaching losses.

\section{Discussion}
Microbial adaptation of enzyme production benefited the community so that higher
biomass levels could be sustained on a wider range of substrate stoichiometry.
The different prototypic simulation scenarios and the simulation of the
intensive pasture led to similar conclusions on comparing the different enzyme
allocation adaptation strategies.

\subsection{Amounts of resources matter}
Both, the amounts of resource pools and resource stoichiometry are important for
regulating enzyme allocation. The match strategy failed to account for resource
amounts with assuming that microbes can achieve balanced growth under a wide
range of resource stoichiometry \citep{Moorhead12, Ballantyne14}. The strategy
was not competetive, as incidated by declining microbial biomass stocks both in
the VarN-Incubation scenario (Fig. \ref{fig:VarNNoFeedback}) and in the
SimSteady scenarios (Fig. \ref{fig:SimSteady}). Microbes focused on degrading a
stoichiometrically balanced but declining residues pool despite the huge amount
of N available in a large but stoichiometrically less favourable litter pool
(Fig. \ref{fig:SimSteady}). This finding implies that microbial enzyme
allocation strategies must account for resource amounts.

\subsection{Community adaptation leads to more efficient resource usage}
The adaptive Revenue strategy consistently supported higher biomass and had
lower N mineralization fluxes at steady state compared to the the non-adaptive
Fixed strategy with the VarN-Incubation scenario (Fig.
\ref{fig:VarNNoFeedback}). Similar patterns appeared with the other scenarios
(Figs. \ref{fig:SimSteady} and \ref{fig:CO2IncreaseImb}). Such better resource
usage is in line with results of individual based small-scale modelling
\citep{Kaiser14}.
The finding implies that N mineralization fluxes with imbalanced resources may
be lower than inferred from previous modelling studies that did not account for
community adaptation.

The SEAM model focuses on community adaptation of enzyme production. It predicts
a change in the ratio of enzyme activities of enzymes degrading C-rich plant
litter versus enzymes degrading the N-rich recalcitrant SOM.
However, in a spatial context at global scale, low variation in stoichiometry
of N-degrading versus C-degrading enzymatic activity have been observed
\citep{Sinsabaugh09}, implying either issues with measurement of enzymatic
stoichiometry or implying further conservative constraints on the enzymatic
flexibility of microbial community.
Such stoichiometric constraints have a large effect on SOM cycling and hence
need further study.

\subsection{SOM can store and release N}
Nitrogen was stored in recalcitrant SOM during periods of high N inputs and
released during periods of low N inputs relative C inputs in simulations (Fig.
\ref{fig:CO2Increase}). When there was excess litter carbon, microbial community
preferentially depolymerized or mined the N rich residue pool and also
made it available for plants by mineralization.
Contrary, when there were low carbon inputs, microbes less strongly degraded the
recalictrant pool but continued to build it via microbial turnover.
Hence, microbes kept N in the system instead of releasing it by mineralization.
This bank mechanism \citep{Perveen14} also worked when simulating the intensive
pasture (Fig. \ref{fig:pastureFit}). During simulations of high inorganic N
inputs, N was sequestered in SOM at a high rate. With decreasing inorganic N
inputs, the sequestration rate decreased until it became negative, i.e. the N
in recalcitrat SOM was mined. Ultimately, on the long term, the inputs to the
system have to balance the outputs of the system. In the intensive pasture simulation,
inorganic N pools and N leading increased with the increase of SOM with the SEAM
model. This finding has consequences on feedbacks of global change, especially
on the projected C land uptake \citep{Friedlingstein14}. When accounting for the
bank mechanisms land carbon sink and plant nutrition will not be as strongly
N-limited.

\subsection{Priming effects recycle N}
Priming effects, i.e. the altered decomposition of SOM after soil amendmends
\citep{Kuzyakov00}, can help plants to stimulate N release for plant nutrition.
Priming effects and associated increased N mineralization were simulated for all
strategies (Fig. \ref{fig:PrimingDec}). With adaptive microbial enzyme
allocation, plant litter inputs influenced the partitioning of SOM between a low
and high C/N pools (Fig. \ref{fig:CO2Increase}) and, hence,
also influenced the distribution of N in the ecosystem and nutrient
availability for plants (Fig. \ref{fig:CO2IncreaseImb}). This active role of
plant inputs has been demonstrated in a soil incubation experiment
\citep{Fontaine11} and has been further conceptualized with the SYMPHONY model
\citep{Perveen14}. Our results are in line with these studies, although our
explanation is on a more abstract level (section \ref{sec:Holistic}). The
conservation or release of N by the bank mechanism implies greater potential
for ecosystems to avoid progressive N limitation \citep{Norby10, Franklin14, Averill15}.

Mineralization during microbial turnover is important for nutrient recycling.
While mineralization by stoichiometric imbalance, $\Phi_B$, is the most widely
implemented flux from microbial biomass to the inorganic carbon pool
\citep{Manzoni09}, it is not sufficient to recycle N if microbial biomass is
itself uptake N limited. Therefore, two additional mineralization fluxes are
implemented with the SEAM model. First, a fraction of microbial DON uptake,
termed uptake mineralization $\Phi_u$, is apparently mineralized that accounts for
C-limited locations in heterogeneous soil \citep{Manzoni08}.
Second, a fraction of microbial turnover is mineralized that accounts for
grazing. Grazers respire part of the grazed biomass carbon for their energy, and
must release an equivalent amount of nutrients to match their stoichiometric
demands. This mineralization component, here termed turnover mineralization
$\Phi_{\operatorname{Tvr}}$, has been formalized in the soil microbial loop
hypothesis \citep{Clarholm85, Raynaud06}. Without such additional mineralization
mechanisms in our simulation experiment, microbes shifted enzyme allocation to
degrade the residues pool, but the N was fixed in microbial biomass and was not
mineralized to inorganic N. Hence, our experiments reinforced the need for
representing soil heterogeneity and grazing for making N available for plants
under N limitation.

\begin{figure}[t] \vspace*{2mm}
\begin{center}
%\includegraphics[width=8.3cm]{fig/seam_NFluxes} 
\includegraphics[scale=0.6]{fig/seam_NFluxes} 
\end{center}
\caption{
In addition to the maybe negative imbalance flux, $\Phi_B$ of microbial
biomass, $B$, there are additional mineralization fluxes feeding the inorganic
pool, $I$, due to mineralization during uptake, $\Phi_u$, and mineralization
during microbial turnover, $\Phi_{\operatorname{tvr}}$. The N dynamics depends
also on fluxes across the system boundary, namely input of litter $iL$ to the
soil organic matter ($SOM$), input of inorganic N $iI$, leaching, and plant
uptake of inorganic N.
\label{fig:SEAMStructNFluxes}}
\end{figure}

A refinement of the term of the term N limitation (Table
\ref{tab:NutrientLimDefs}) is required by the introduction of the additional N
mineralization fluxes.
When microbes can meet their stoichiometric demand only by immobilizing
inorganic N, I suggest the term uptake N limitation.
When the immobilization flux cannot meet microbe's stoichiometric requirements,
I suggest the term microbial N limitation. Despite the maximum immobilization flux
there might still be a net mineralization due to uptake mineralization and turnover
mineralization.
When there is a net immobilization, i.e. a net transfer from inorganic pool to
the organic pools of SOM and microbial biomass, I suggest the term SOM N
limitation.

\begin{table}[t]
\caption{Increasing levels of N limitation \label{tab:NutrientLimDefs}}
%\vskip4mm
\centering
\begin{tabular}{lp{5.5cm}}
\tophline
Term &  Definition \\
\middlehline
Uptake N lim. & N in microbial upatake is less than 
constrained by other elements (${\Phi_B < 0}$).
\\
Microbial N lim. & Maximum immobilization flux is not enough to satisfy
microbial N requirements (${-\Phi_B =
u_{\operatorname{imm,Pot}}}$).
\\
SOM N lim. & There is a net transfer from the inorganic pool to
the organic pools (${\Phi_B+\Phi_u+\Phi_{\operatorname{tvr}}<0}$).
\\
\bottomhline
\end{tabular}
\end{table}
 


\subsection{Mismatch in time scale of priming effects}
The unrealistically long time scale of priming effects in SEAM resulted from
both, the turnover time of enzymes and the positive feedback between amounts of
microbial biomass and enzymes. Time scale of priming effects of several month in
SEAM simulations (Fig. \ref{fig:PrimingDec}) was in contrast with incubation
studies that observe priming effects within days or weeks that rapidly
declined after the amendmend has been used up \citep{Blagodatskaya14}. Here, we
discuss hypotheses of what processes could be reponsible for this mismatch and
need further study.
Priming timescale in SEAM was longer than the duration of the uptake pulse of
the $L$ amendment that only lasted a few days. It was controlled by simulated
enzyme turnover, which SEAM described as first order kinetics with a turnover of
about a week. Moreover, priming timescale was prolonged by the
positive feedback of increased microbial biomass producing more enzymes that
again fueled biomass.

One option to decrease priming timescale in SEAM is a different representation
of enzyme turnover. Prescibing a shorter turnover time of enzymes, however,
would require an increased effort of producing enzymes by microbial biomass.
More sophisticated models of different enzyme turnover kinetics including
stabilization of a part on mineral surfaces \citep{Burns13}, hopefully, will be
able to resolve such contradictions. Testing this hypothesis requires data on
the fraction of uptake allocated to enzyme production and data on age distribution
of enzymes in soil.

An alternative option to decrease priming timescale, is to describe the priming
effect in a different manner. Some enzymes such as peroxidases need to be
fueled by labile organic matter themselves \citep{Rousk14} with no immediate relationship
to microbial biomass dynamics. This explanation, however, implies that enzyme
activity and decomposition of SOM becomes decoupled from enzyme production and
microbial dynamics to a large extent in the short term.
This option is contrary to the assumption of most current models that simulate
the priming effect. 

Another option to decrease the priming timescale, is to diminish the sustaining
positive feedback between enzymes and microbial biomass. Currently, grazing is
an implicit part of a first order microbial turnover. With increasing microbial
biomass, grazers become more efficient \citep{Clarholm81}. With implementing a
time-lagged stronger increase in microbial turnover rate with microbial biomass,
biomass levels would decrease faster to pre-treatment levels and help to shorten
the time-scale of the priming effect. Testing this hypothesis requires data on
grazing during priming effects.

Overall, the unrealistically simulated priming time scale hints to gaps in
understanding short-term SOM turnover. However, it does not impair the simulated
longer-term microbial community controls on SOM cycling. We argue that the
simulated long-term patterns are robust, because they are more strongly
controlled by the proportions in enzyme production than by the time scale of
priming effects.

\subsection{A holisctic view for upscaling}
\label{sec:Holistic}

The presented SEAM conceptual model takes a holistic view \citep{Panikov10} of
microbial community and their adaptations instead of explicitely describing
microbial diversity.
In this respect, it differs from the SYMPHONY model \citep{Perveen14} and
similar conceptual models \citep{Fontaine03}, that explicitly modelled several
microbial groups.
% specifically SOM builders, that grow solely on fresh low nutrient material
% (here $L$), and SOM builders that can use all the SOM.
The effective behaviour of the presented SEAM model, however, is similar.
It assumes that community composition is driven to a large extent by external
drivers outside the community. The SEAM model describes an adaptive allocation
of resources into breakdown of different resources by assuming that the
community composition adapts to changed resource availability in a way to
optimize microbe's revenue.
While the mechanistic approach of the SYMPHONY model explicitly represents this
optimization by shifts between microbial groups, the holistic approach
represents the effects of this optimization at community level.
While the mechanistic approach gives more detailed understanding of the mechanism,
I hypothesize, that the holistic approach is more suitable for upscaling
\citep{Wutzler13}.
Hence, the proposed abstraction of microbial competition by the revenue strategy
is a step forward of better representing couplings of soil carbon and nutrient
cycles in earth system models and a step forward to improved predictions of long
term carbon sequestration.

The holistic SEAM model yielded qualitatively similar predictions as the
mechanistic SYMPHONY model with simulating priming, the bank mechanism, and a
continuous SOM sequestration under high inorganic N inputs. It differed from
SYMPHONY in the prediction of the inorganic N pool during low N inputs.
Specifically, it predicted a decrease in this pool, while SYMPHONY predicted an
increase in this pool due to changed competition \citep{Perveen14}. The
difference is probably caused by different assumptions on how the DOM pool is
shared among groups of the microbial community and resulting different
competition conditions. In the SEAM model, decomposition products become mixed
in a shared DOM pool, while in the SYMPHONY model the decomposition products are
not shared between the microbial groups.
The truth at pore scale is in between, in that decomposition products are mainly
used by the group that is producing the extracellular enzymes, while a part of
the DOM diffuses also to other groups \citep{Kaiser14}. At larger scales, such
details cannot be measured or reolved. The difference in model prediction
implies that the rationality of the simplified model assumptions of a mixed
DOM pool can be qualitatively tested against observations. 

\subsection{Testable predictions of change of SOM C/N ratios}
The SEAM model can be used to predict long term patterns of SOM cycling after
changes in resource stoichiometry. Observations of such pattern provide evidence
for or against the modelling assumptions.
Specifically, SEAM predicted a change in proportions of the litter pool and the
SOM pool (Fig. \ref{fig:CO2Increase}). While these abstract pools are not
directly comparable to observations, a measureable consequence is the associated
change of total SOM C/N ratio at the time scale of turnover of the recalcitrant
pool. Specifically, SEAM predicted a decline in SOM stocks and an increase of
SOM C/N with Face experiments at formerly C-limited systems over time scales of
several decades.
 
\subsection{Outlook} 
The biggest limitation of the SEAM model is its focus on a single process:
community adaptation of enzyme allocation. In order to focus, I had to ignore
several other important processes. One such process is the second microbial
community strategy of handling resource stoichiometric imbalance,
the adaption of stoichiometry of microbial biomass. Although the potential of this biomass
adaptation is thought to be quite limited \citep{Mooshammer14}, it will be
tested whether these two strategies can be combined within a model.

Next, the optimality principle will be extended to determine the proportion of
uptake that is allocated to enzyme production. Presence of cheaters, i.e.
microbes that consume resource but without producing enzymes, effectively lower
the community-level allocation to enzymes \citep{Kaiser14}. Community
development can be assumed to maximise biomass production. Such an assumption
can be used to compute the optimal community enzyme production and allows
exploring effects on SOM cycling, such as more constrained carbon and nutrient
use efficiencies.

Moreover, SEAM will be simplified by assuming quasi-steady state of biomass or
enzyme pools \citep{Wutzler13}. These simplifications will lead to fewer
parameters and improved identifiably in model calibration to observations.
Togehter with implementing the influence of environmental factors such as
temperature and moisture \citep{Davidson12}, these changes will make SEAM more
suitable to be used as a component within larger scale land surface models.
 

\conclusions   
%% \conclusions[modified heading if necessary]
 
The adaptation of allocating resources into the production of different enzymes
is an important means of the microbial community to react to changing resource
stoichiometry. Allocation adaptation strategies helped microbial biomass in SEAM
(Fig. \ref{fig:SEAMStruct}) simulations to grow larger biomass across a wider
range of resource stoichiometry (Fig. \ref{fig:VarNNoFeedback}). Among the
tested strategies, the revenue strategy was particularly successful, which took
into account both, the amount of resource and their stoichiometry.
The findings imply that models that want to simulate soil carbon and nutrients
dynamics (Figs. \ref{fig:PrimingDec} and \ref{fig:PrimingMin}) must account for
adaptations in carbon and nutrient strategies. Accounting for adaptations will
be especially important, when studying the competition for nutrients between
soil microorganism and plants, because SOM can function as a storage to
sequester surplus elements and prevent them from leaving the system (Fig.
\ref{fig:CO2Increase} and \ref{fig:CO2IncreaseImb}).

The SEAM model provides a holistic description of community adaptations. It
yields qualitatively similar predictions as microbial-group-explicit models
with the ability to represent priming effects, bank mechanism, and
a continuous SOM sequestration with high inorganic N inputs (Fig.
\ref{fig:pastureFit}).

Hence, this study provides an important step for providing an abstract
description of microbial community effects and adaptations, with the long-term
goal of including the important mechanisms into earth system models.





%% The Appendices part is started with the command \appendix;
%% appendix sections are then done as normal sections
%% \appendix

%% \section{}
%% \label{}

%% If you have bibdatabase file and want bibtex to generate the
%% bibitems, please use
%%
\bibliographystyle{elsarticle-harv} 
\bibliography{twutz_txt}

%% else use the following coding to input the bibitems directly in the
%% TeX file.
%\begin{thebibliography}{00}

%% \bibitem[Author(year)]{label}
%% Text of bibliographic item
%\bibitem[ ()]{}
%\end{thebibliography}
\end{document}
\endinput
%%
%% End of file `elsarticle-template-harv.tex'.
