\introduction  %% \introduction[modified heading if necessary] 
The global element cycles of carbon and nitrogen are strongly linked and cannot
be understood without their interactions (XXCStorebyN). The links between nutrient
cycles are especially strong in dynamic of soil organic matter (SOM) because all
of the SOM has to be depolymerized and successively mineralized by a microbial
community with a rather strict homeostatic regulation. Faced with stoichiometric
imbalances between OM substrates and synthesized biomass, decomposers have to
lower their carbon use efficiency (CUE) or nutrient use efficiency (NUE)
(XXSterner, Mooshammer). The regulation of CUE has shown to have large
consequences on prediction of carbon sequestration in SOM (XX Allison follows,
Manzoni). Regulation of NUE has consequences for nutrient recycling and loss of
nutrients from the ecosystem (XXMooshammer) and soil plant feedbacks
(XXRastetter 97). In addition, however, decomposers could also regulate
allocation into production of extracellular enzymes to preferentially
depolymerize food sources of different stoichiometry in order to match their
requirements (XXSterner, Mooshammer). This hypothesis was recently formalized by
XXMoorehead with the conceptual EEZY model. While this model showed the
principle feasibility of this strategy in short term, it did implement feedbacks
to substrate pools and therefore could no look at consequences for longer term
SOM cycling. The associated change in CUE and NUE, however, can have severe
consequences on the long term SOM cycling and our understanding of soil carbon
sequestration and plant- soil- atmosphere elemental feedbacks.

The aim of this study, therefore, is to further develop the conceptual model of
enzyme allocation including feedbacks of turnover to the substrate pools and
explore the different consequences of alternative enzyme allocation schemes on
long term SOM dynamics and nutrient recycling.

To this end we developed a conceptual model of SOM cycling that explicitly
represented several extracellular enzyme pools involved in the depolymerization
of specific substrates which differed in their elemental ratios. Several
variants were developed that differed by the strategy of enzyme allocation:
either fixed or flexible, and either based on stoichiometry only or on pool
sizes too. Simulations scenarios were devised to study effects on mineralization
fluxes and buildup of SOM. Specifically, we first studied under which ranges of
substrate stoichiometry decomposers with flexible allocation could use the
substrate more efficiently and decrease mineralization fluxes (including
overflow respiration as carbon mineralization). Second, we tested with which
strategies sustainable microbial communities could develop under constant litter
input, by simulating a steady state. Third, we were interested in different
responses of subsoil SOM mineralization and nutrient recycling after amendment
of fresh litter (priming). And fourth, we studied consequences for development
of SOM pools in a scenario of CO2 fertilization.

We show that different decomposers enzyme allocation strategies have large
consequences on long term SOM dynamics and nutrient recycling and suggest
further ways to study these strategies by both experimental, modelling and
combined studies.

\section{Methods}
\subsection{Soil Enzyme Allocation Model (SEAM)}
We used a conceptual structural dynamic model to explore consequences of enzyme
allocation strategies for SOM cycling: Soil Enzyme Allocation Model (SEAM). The
modelled system are the carbon and nitrogen pools in soil organic matter in a
representative elemental volume of soil. This could be soil in laboratory
incubation or a soil layer of about $1 \unit{m}^2$ outside roots. Different
pools of carbon and nitrogen are represented as state variables and the mass
fluxes between them govern their change by dynamic equations. We assume that the
single input to the system is by plant litter (both above ground and
rhizodeposition) and the sole outputs are respired carbon dioxide and
mineralized nitrogen. 

Key features are the representation of different SOM pools that differ by
elemental ratios and the explicit representation of specific enzymes that
depolymerize those SOM pools. The quality spectrum is modelled by two
classes: a nitrogen rich more easily degradable labile litter pool,
$\operatorname{L}$, and a pool mainly consisting of nitrogen rich microbial
residues, $\operatorname{R}$ (Fig. \ref{fig:SEAM}).

\begin{figure}[t] \vspace*{2mm}
\label{fig:SEAM}
\begin{center}
\includegraphics[width=8.3cm]{fig/eezy5}
\end{center}
\caption{Model strucutre of SEAM: Two substrate pools ($L$ and
$R$) are depolymerized by respective enzymes
($E_L$ and $E_R$). The simple organic
compounds ($\operatorname{DOM}$) are transported by soil solution and taken up
by the microbial community  and used for synthesizing
new biomass ($\operatorname{B}$), new enzymes, and for catabolic respiration. 
Due to stoichiometric imbalance between $\operatorname{DOM}$ and $\operatorname{B}$
there will be overflow respiration or mineralization of excess nitrogen. During
microbial turnover an additional part of the biomass is mineralized and the
other part is added to the residue pool, $R$. Boxes represent pools, black
arrows mass fluxes, white arrows othr controls, and disks represent
partitioning of fluxes.}
\end{figure}

Appendix \ref{app:SEAM} reports the model equations and details. Here we
describe the most important assumptions.

C/N ratios, $\beta$ of biomass and enzymes are
fixed while those of the substrate pools may change over time due to
fluxes in and out of the pools. Imbalances in stoichiometry of uptake and demand
in microbial biomass are compensated by overflow respiration or nitrogen
mineralization.

Total enzyme allocation is a fixed fraction,
$a_E$, of the microbial biomass, $B$, per time. But microbial community can
adjust their allocation to produce new enzymes $E_R$ or $E_L$: $\alpha = E_R / (E_R + E_L)$. The different strategies
of enzyme allocation are explained in section \ref{sec:AllocStrategies}.

The DOM pool is assumed to be in quasi steady state and all the sum of all
influxes to the DOM pool (decomposition + part of the enzyme turnover) is taken
up by microbial community.

If expenses for maintenance and enzyme production cannot be met, the biomass
starves and declines.
 
\subsection{ Enzyme allocation strategies 
\label{sec:AllocStrategies}}

Three different strategies of allocating investments among production of
alternative enzymes were explored in this study (Table
\ref{tab:AllocStrategies}). Microbes invest porportion $\alpha$ of total enzyme
investments into production of enzymes targeting the nitrogen rich $R$ substrate
and proportion $1 - \alpha$ into production of enzymes targeting the nitrogen
poor but betteer degradable $L$ substrate.

The \textbf{fixed} strategy assumes that allocation is independent
and not changing with changes in substrat availability. 
\begin{equation}
\label{eq:allocFixed}
\alpha = \operatorname{const.}
\end{equation}
This strategy corresponds to the models where decomposition rate is a function
of microbial biomass \citep{Wutzler08}.
 
The \textbf{match} strategy assumes that microbes balance their stoichiometric
demands \citep{Moorhead12}. The partitioning coefficient is derived by equating
the C/N ratio of the sum of decomposition fluxes and the C/N ratio of microbial
biomass and solving for $\alpha$.
\begin{equation}
\label{eq:allocMatchCN}
\beta_B &= \frac{\epsilon (\operatorname{dec}_L + \operatorname{dec}_R - r_M)}{
\operatorname{dec}_L/\beta_L + \operatorname{dec}_R/\beta_R } \text{,}\\
\end{equation}
where $\operatorname{dec}_L$, and $\operatorname{dec}_R$ are
depolymerization fluxes of the labile and residue substrates
respectively, $r_M$ is maintenance respiration, $\epsilon$ is the intrinsic microbial efficiency, and $\beta$ are C/N ratio of
the respective pools i.
In the derivation, the enzyme levels have been assumed in steady state for given
amounts of substrate and microbial biomass.
\begin{equation}
\label{eq:allocMatch}
\alpha_M &= f_{\operatorname{{\alpha}Fix}}(L,\beta_L,R,\beta_R,B); \,
\alpha &= \begin{cases}
  0,  & \text{if} \alpha_M \le 0 \\
  1,  & \text{if} \alpha_M \ge 1 \\
  \alpha_M, & \text{otherwise}
\end{cases} \\  
\end{equation}
The bounds to zero and one are necessary because the sole
allocation to decompose the carbon rich substrate may not suffice to
cover microbial carbon demands. Then nitrogen mineralization
has to balance the stoichiometric imbalance. 
Function $f_{\operatorname{{\alpha}Fix}}$ is given in appendix
\ref{app:fAlphaFix} and the sympy script of its 
derivation is given with supplementory material. 

The \textbf{revenue} strategy assumes that microbes invest into enzyme
production porportional to their revenue, i.e. their return per investment
regarding the currently limiting element.

The return is the current decomposition flux from respective substrate. The
investment needs to be equal to enzyme turnover to keep current enzyme levels.
% : $a_E B = k_{NR} E_R^* + k_{NL} E_L^*$.
% 
\begin{subequations}
\label{eq:allocRev}
\begin{align}
\alpha_C &= \frac{e_{RC}}{e_{LC} + e_{RC}} \\
e_{SC} &= \frac{\text{return}}{\text{investment}} 
= \frac{\operatorname{dec}_{S,Pot} \frac{E_S^*}{k_{m,S} + E_S^*}} {k_{NS}E_S^*} 
= \frac{\operatorname{dec}_{S,Pot}} {k_{NS}(k_{m,S} + E_S^*)} \\ 
e_{SN} &= \frac{\operatorname{dec}_{S,Pot}
\frac{E_S^*}{k_{m,S} + E_S^*} / \beta_S} {k_{NS} E_S^* / \beta_E} 
= \frac{\operatorname{dec}_{S,Pot}}{k_{NS} (k_{m,S} + E_S^*)}
\frac{\beta_E}{\beta_S}
\text{,} 
\end{align}
\end{subequations}
where $e_S$ is the revenue from given substrate $S$ ($S$ is either $L$ or $R$)
under carbon and nitrogen limitation respectively.
$k_{NS}$ is rate of enzyme turnover, $k_{m,S}$ is substrate affinity, $a_E$ is
total enzyme allocation coefficient, $\operatorname{dec}_{S,Pot} = k_S S$ is
enzyme saturated decomposition flux, and $\beta$ are C/N ratios of the
respective pools.

There are two partitionings, $\alpha_C$, with carbon limitated microbial
biomass, and $\alpha_N$, with nitrogen limited microbial biomass. In order to
avoid frequent large jumps under near co-limitation, SEAM implements a smooth
transition between these two cases as a weighted average with weights based
on ratio of required to available biomass synthesis fluxes (derived in
appendix \ref{app:SEAM}).

\begin{subequations}
\label{eq:allocRev}
\begin{align}
\alpha &= \frac{w_{\operatorname{CLim}} \alpha_C + w_{\operatorname{NLim}}
\alpha_N}{w_{\operatorname{CLim}}  + w_{\operatorname{NLim}} } \\
w_\operatorname{CLim} &= \left( \frac{\text{required}}{\text{available}}
\right)^\delta 
= \left( \frac{ N_{\operatorname{synBN}} \beta_B }{ C_{\operatorname{synBC}} }
\right)^\delta
\\
w_\operatorname{NLim} &= \left( \frac{ C_{\operatorname{synBC}} / \beta_B }{
N_{\operatorname{synBN}} } \right)^\delta
\text{,} 
\end{align}
\end{subequations}
where $\delta$ controls the steepness of the transition between the two
limitations. It was arbitrarily set to a value of 200.















\subsection{ Simulation scenarios 
\label{sec:SimScen}}


\section{Results}

\section{Discussion}
Differences to Moorhead12 approach

Both litter pools of two elements

maintenance respiration

allocation part of biomass instead of uptake, decoupled

prediction changing C/N ratios

holistic view on microbial community, upscaling




\conclusions  %% \conclusions[modified heading if necessary]
TEXT

