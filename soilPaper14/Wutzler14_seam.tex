
\introduction 
% % \introduction[modified heading if necessary] 
The global element cycles of carbon (C) and nitrogen (N) are strongly linked
and cannot be understood without their intricate interactions 
\citep{Thornton07,Janssens10, Zaehle11}. 
The ties between nutrient cycles are especially strong
in the dynamics of soil organic matter (SOM), because the depolymerisation and mineralisation of SOM relies on a microbial community with a rather strict homeostatic regulation of their stoichiometry, i.e. their elemental ratio of C/N \citep{Sterner02, Zechmeister15}.

%%somehow misses an introduction saying that this paper is about the "correct" way to deal with this in ecosystem models. 

Decomposers can - in principle - adjust in three different ways when faced with imbalances between the stoichiometry of the organic material (OM), i.e. the litter and SOM they
feed on, and their own stoichiometric requirements \citep{Mooshammer14}.
%are they not two separate things? % is the concept of these efficiencies clear to everyone?
First, individual microbes can adapt their carbon-use efficiency (CUE), or their nutrient-use
efficiency (NUE) \citep{Sinsabaugh13}. The alteration of CUE has shown to have
large consequences on prediction of carbon sequestration in SOM
\citep{Allison14a, Wieder13}.
Regulation of NUE has consequences for nutrient recycling and loss of nutrients
from the ecosystem \citep{Mooshammer14a} and soil plant feedback
\citep{Rastetter11}.
Second, decomposer communities can adapt their stoichiometric
requirements. Community composition can shift between species with high C/N
ratio, such as many fungi, or species with lower C/N ratio, such as many
bacteria \citep{Cleveland07, Xu13}, although the flexibility is very narrow.
Third, decomposers can adapt their allocation of resources into
synthesis of different extracellular enzymes to preferentially degrade
fractions of SOM that differ by their stoichiometry \citep{Moorhead12}.

Representation and consequences of stoichiometry on element cycling differ
between models at different scales. Most models at ecosystem scale employ the
first option, and use changes in CUE or NUE to represent stoichiometric controls
on respiration and mineralization fluxes \citep{Manzoni08}. However, modelling
studies at the pore scale have demonstrated the important effect of community
adaptation and their emerging effects on element cycling \citep{Allison05,
Resat11, Wang13}. Explite representaion of competition among several microbial
groups that differ in their expression of different enzymes resulted in a
comparable simulated CUE across a wide range of litter stoichiometry
\citep{Kaiser14}. Likely, therefore, there is a need to capture the effects of
community adaptation also in models at ecosystem scale.

At least two alternative exists to represent the effects of microbial diversity
at the ecosystem scale. First, competition of several microbial populations can
be explicitly modelled to represent stoichiometric effects such as sustained
sequestration of N with high N inputs \citep{Perveen14}.
Second, adaptation of effective properties of the entire microbial community,
such as concepts called uptake efforts \remarkSZ{}{(not clear what this is)}
\citep{Rastetter97, Rastetter11}, can represent the emerging effects in an
abstract but dynamic and adaptive way. The adaptation of enzyme allocation was
recently formalised using the second strategy by the conceptual EEZY model
\citep{Moorhead12}.
While this model shows the strong effects on nutrient cycling such as N
mineralization of this strategy in time scale of days to months, it does not
represent feedback mechanisms to the size and stoichiometry of the SOM pools,
and therefore cannot study the consequences for long-term SOM dynamics.

In this paper, we adopt the second alternative as working hypothesis and propose
a holistic scheme to represent effects of microbial adaptation of enzyme
synthesis on SOM cycle at the ecosystem scale. Our aim was to tackle the need
of capturing the decadal time scale effects of adaptive enzyme synthesis on 
SOM dynamics and nutrient recycling. We therefore extended the EEZY model to
explore different consequences of alternative enzyme allocation schemes.

This paper first introduces the SEAM model (Section \ref{sec:SEAM}), a dynamical
model of SOM cycling that explicitly represents microbial enzystrategies of
producing several extracellular enzyme pools (Section
\ref{sec:AllocStrategies}). Next, the effects of those strategies on SOM cycling
are presented by prototypical examples (Sections \ref{sec:SimScen} and
\ref{sec:ResultsProto}).
Finally, a calibration to an intensive pasture site (Section
\ref{sec:methodsPasture}) demonstrates the usability of the model (Section
\ref{sec:ResultsPasture}) and compares its predictions to the ones of the
Symphony model \citep{Perveen14}, which explicitly models several
microbial-groups. 

\section{Methods}
\subsection{Soil Enzyme Allocation Model (SEAM)}
\label{sec:SEAM}

The dynamic Soil Enzyme Allocation Model (SEAM) allows to explore consequences
of enzyme allocation strategies for SOM cycling at the soil core to ecosystem
scale. The modelled system are C and N pools in SOM in a representative
elemental volume of soil. The system could be soil of a laboratory incubation or
a layer of a soil profile, e.g. its upper 20~\unit{cm}.
The model represents different SOM pools containing C and N as state variables
and specifies differential equations for the mass fluxes. It is driven by C and
N inputs of plant litter (both above-ground and rhizodeposition), inorganic N
inputs from deposition and fertilisers, as well as prescribed
uptake of inorganic N by roots. SEAM computes output fluxes of heterotrophic
respiration and leaching of inorganic N at monthly to decadal time scale.

Key features are: first, the representation of several SOM pools that differ by
their stoichiometry, and second, the representation of
specific enzymes that degrade those SOM pools. The quality spectrum is modelled by two
classes: a C rich litter pool,
$\operatorname{L}$, and a N rich pool that consists
mainly of microbial residues, $\operatorname{R}$ (Fig. \ref{fig:SEAMStruct}). The most important assumptions are described in the following paragraphs, while
the symbols are explained in Tab. \ref{tab:modelParameters} and detailed model
equations are given in Appendix \ref{app:SEAM}. 

\begin{figure}[t] \vspace*{2mm}
\begin{center}
\includegraphics[width=8.3cm]{fig/seam2}
\end{center}
\caption{
Model structure of SEAM: Two substrate pools ($L$ and $R$) which differ in their
elemental ratios are depolymerized by respective enzymes ($E_L$ and $E_R$). The
simple organic compounds ($\operatorname{DOM}$) are taken up by the microbial
community  and used for synthesizing new biomass (${B}$), new enzymes, or for
catabolic respiration. Stoichiometric imbalance between $\operatorname{DOM}$ and
${B}$ causes overflow respiration or mineralization/immobilization ($\Phi_B$) of
inorganic N ($I$) (further detailed in Fig.
\ref{fig:SEAMStructNFluxes}).
Boxes correspond to pools, black arrow heads to mass fluxes, white arrow heads to
other controls, and disks represent partitioning of fluxes. Solid lines
represent fluxes of both C and N, while dotted and dashed lines represent
separate C or N fluxes respectively.
\label{fig:SEAMStruct}
}
\end{figure}



% \section{\\ \\ \hspace*{-7mm}  Symbols and Values \label{app:symbols}}    %%
% % Appendix
% 
% Table \ref{tab:modelParameters} lists symbols used in the model together with
% their values as applied in the model calibration. It also reports initial
% values of the state variables for the grassland calibration. 

\begin{table}[t]
\caption{
\label{tab:modelParameters}
Model parameters and drivers. The first value column refers to the
prototypical example while the subsequent column reports values that
differ for the grassland calibration (TODO adjust). }
\vskip4mm
\centering
\begin{tabular}{lp{6cm}lllp{5.5cm}}
\tophline
Symbol &  Definition & \multicolumn{2}{c}{Value} & Unit & Rational \\
\middlehline
\multicolumn{6}{l}{State variables}  \\
$L$ &  C in litter & & 571 & \unit{gm^{-2}} & quasi steady state 
\\
$L_N$ &  N in litter & & 815 & \unit{gm^{-2}} & \citep{Perveen14}
(by their N/C ratio $\beta$)
\\
$R$ &  C in residue substrate & & 10500 & \unit{gm^{-2}} &
\citep{Allard07} (total stocks - L - dR)
\\
$R_N$ &  N in residue substrate & & 968 & \unit{gm^{-2}} & by C/N
ratio in \citep{Perveen14} \\
$E_L$ &  C in enzymes targeting $L$ & & 0.34 & \unit{gm^{-2}} & 
quasi steady state \\
$E_R$ &  C in enzymes targeting $R$ & & 0.21 & \unit{gm^{-2}} & 
quasi steady state \\
$B$ & microbial biomass C & & 89.2 & \unit{gm^{-2}} &  quasi steady
state \\
$I$ & inorganic N & & 2.05 & \unit{gm^{-2}} & \citep{Perveen14} \\
\\
\multicolumn{6}{l}{Model parameters}  \\
$\beta_B$ &  C/N ratio of microbial biomass & 11 & & \unit{gg^{-1}} &
Perveen 2014 
\\
$\beta_E$ &  C/N ratio of extracellular enzymes & 3.1 & &
\unit{gg^{-1}} & Sterner 2002 \\
$\beta_{\mathrm{input}_L}$ &  C/N ratio of plant litter inputs & 30 & 70 &
\unit{gg^{-1}} & \citep{Perveen14} ($1/\beta$) \\
$k_R$ &  maximum decomposition rate of $R$ & 1 & 4.39e-2 & \unit{yr^{-1}}
& calibrated \\
$k_L$ &  maximum decomposition rate of $L$ & 5 & 1.95 & \unit{yr^{-1}}
& calibrated \\
$k_N$ &  enzyme turnover rate &  60  & & \unit{yr^{-1}} & \citep{Burns13} \\
$k_{NB}$ & enzyme turnover entering DOM rather than $R$ & 
0.8 & & (-) & mostly small proteins \\
$a_{E}$ &  enzyme production per microbial biomass & 0.365 & &
\unit{yr^{-1}} & $\approx 6\%$ of biomass synthesis \\ 
$K_{M}$ &  enzyme half saturation constant & 0.05 & &
\unit{gm^{-2}} & magnitude of DOC concentration \\
$\tau$ &  microbial biomass turnover rate & 6.17 & & \unit{yr^{-1}} &
\citep{Perveen14} ($s/\epsilon_{\operatorname{Tvr}}$) \\
$m$ & specific rate of maintenance respiration & 1.825 & 0 & 
\unit{yr^{-1}} & \citep{Bodegom07}, zero in \citep{Perveen14} \\
$\epsilon$ & anabolic microbial C substrate efficiency & 0.5 & 0.53 &
(-) & calibrated \\ %\citep{Perveen14} ($s/(s+r)$ +2\% enzymes) \\
$\nu$ & aggregated microbial organic N use efficiency & 0.7 &
0.9 & (-) & Manzoni 2008 \\
$\epsilon_{\operatorname{Tvr}}$ & microbial turnover that is not
mineralized & 0.3 & 0.8 & (-) & part of turnover is consumed by
predators
\\
$i_{B}$ & maximum microbial uptake rate of inorganic N & 25 & & \unit{yr^{-1}} 
& larger than simulated immobilization flux \\
$l$ & inorganic N leaching rate & 0 & 0.959 &
\unit{yr^{-1}} & \citep{Perveen14} ($l)$ \\
\\
\multicolumn{6}{l}{Model drivers, i.e. fluxes across system boundary}  \\ 
$\mathrm{input}_{L}$ & litter C input & & 969.16 &
\unit{gm^2yr^{-1}} & \citep{Perveen14} \, ($m_p C^{obs}_p$)\\
$i_{I}$ & inorganic N input & & 22.91 & \unit{gm^2yr^{-1}} 
& \citep{Perveen14} \\
$k_{IP}$ & inorganic plant N uptake & & 16.04 & 
\unit{gm^2yr^{-1}} & \citep{Perveen14} (assuming plant
steady state: plant N export + litter N input)\\
\\
\multicolumn{6}{l}{Fluxes and quantities derived within the system}
\\
$\alpha$ & proportion of enzyme investments allocated to production of 
$E_R$ & & & (-) &
\\
$\operatorname{syn}_B $ & C for microbial biomass synthesis &  &
& \unit{gm^2yr^{-1}} &
\\
$\operatorname{syn}_E $ & C for enzyme synthesis &  &
& \unit{gm^2yr^{-1}} & \\
$\operatorname{tvr}_B $ & microbial biomass turnover C &  &
& \unit{gm^2yr^{-1}} & \\
$\operatorname{tvr}_E $ & enzyme turnover C &  &
& \unit{gm^2yr^{-1}} & \\
$\operatorname{dec}_X $ & C in decomposition of resource X ($X$ is $L$ or $R$)
& & & \unit{gm^2yr^{-1}} & \\
$u_C,u_N$ & microbial uptake of C and N  & &
& \unit{gm^2yr^{-1}} & \\
$\Phi_u, \Phi_B, \Phi_{\operatorname{tvr}}$ & N mineralization with microbial
DOM uptake, stoichiometric imbalance, and turnover & &
& \unit{gm^2yr^{-1}} & see Fig. \ref{fig:SEAMStructNFluxes}
\\
\bottomhline
\end{tabular}
\end{table}


 

Decomposition of litter and residue pool follows an inverse Michaelis-Menten
kinetics \citep{Schimel03}, which is first-order to the amount of OM, and
saturates with the amount of the respective enzyme. C/N ratios, $\beta$, of the
decomposition flux are equal to the C/N ratios of the decomposed pool. The C/N
ratios of biomass and enzymes are assumed to be fixed, while those of the
substrate pools may change over time due to changing C/N ratio of total
influxes to these pools. Imbalances in stoichiometry of
uptake and microbial requirements are compensated by overflow respiration or N
mineralization.
Total enzyme allocation is a fixed fraction, $a_E$, of the microbial biomass,
$B$, per time. However, the microbial community can use different strategies to
adjust their allocation to synthesis of alternative kinds of new enzymes
(Section \ref{sec:AllocStrategies}).
The DOM pool is assumed to be in quasi steady state. The sum of all influxes to
the DOM pool, i.e. decomposition plus part of the enzyme turnover, is taken up
by the microbial community. If expenses for maintenance and enzyme synthesis
cannot be met, the microbial community starves and declines in biomass.

\subsection{Exchange with inorganic N pools}

%%it would have been easier to follow this section if equations were provided

The imbalance flux, $\Phi_B$ (\ref{eq:PhiB}), lets microbes mineralise excess N,
or immobilise required N up to a maximum rate, $u_{\operatorname{imm,Pot}}$. The
latter is assumed to increase linearly with the inorganic N pool.
While this stoichiometric imbalance flux is the most widely implemented flux
mechanism between microbial biomass and the inorganic carbon pool in SOM models
\citep{Manzoni09}, it is not sufficient to recycle N to the inorganic pool if
microbial biomass is itself N limited.
Therefore, two additional mineralisation fluxes are implemented in the SEAM
model (Fig. \ref{fig:SEAMStructNFluxes}). First, a fraction of microbial DON
uptake, $\Phi_u$ (termed uptake mineralisation), is  mineralised to
represent imbalance flux at C-limited spots in heterogeneous soils at
the scale of a soil core that is in bulk not C-limited \remTW{understandable?}
\citep{Manzoni08}.
Second, a fraction of microbial turnover is mineralised that accounts for
grazing. Grazers respire a fraction of the grazed biomass C to meet their energy
demand, and - assuming invariant grazer stoichiometry  - must release an
equivalent amount of nutrients to match their stoichiometric demands.
This mineralization component, here termed turnover mineralization
$\Phi_{\operatorname{tvr}}$, has been formalised in the soil microbial loop
hypothesis \citep{Clarholm85, Raynaud06}.

\begin{figure}[t] \vspace*{2mm}
\begin{center}
%\includegraphics[width=8.3cm]{fig/seam_NFluxes} 
\includegraphics[scale=0.6]{fig/seam_NFluxes} 
\end{center}
\caption{
In addition to the maybe negative imbalance flux, $\Phi_B$ of microbial biomass,
$B$, there are additional mineralization fluxes feeding the inorganic pool, $I$,
due to mineralization during uptake, $\Phi_u$, and mineralization during
microbial turnover, $\Phi_{\operatorname{tvr}}$. The N dynamics depends also on
fluxes across the system boundary, namely input of organic N with litter, input
of inorganic N $iI$, leaching, and plant uptake of inorganic N.
\label{fig:SEAMStructNFluxes}}
\end{figure}

In the light of the introduction of these additional N mineralisation fluxes, a
refinement of the term N-limitation (Table
\ref{tab:NutrientLimDefs}) is required.
When microbes cannot meet their stoichiometric demand by DOM uptake but can
meet ther demand by immobilising N, we suggest the term \textit{uptake
N-limitation}.
When the immobilisation flux cannot meet the stoichiometric requirement of the
microbial community, we suggest the term microbial N-limitation. Despite the
maximum immobilisation flux \remarkSZ{there might still be a net mineralization
due to uptake mineralization and turnover mineralization.}{(I don't follow here.
If there is net N mineralisation through these fluxes, then the N liberated by
these processes should be immediately used by the microbes to increase the
immobilisation flux. Unless you mean gross mineralisation, which would only
reduce the N deficit. What am I not getting?)} When there is a net
immobilizsation, i.e. a net transfer from inorganic pool to the organic pools of
SOM and microbial biomass, we suggest the term \remarkTW{ SOM N
limitation.}{still searching for a better term here}.

\begin{table}[t]
\caption{Increasing levels of N limitation \label{tab:NutrientLimDefs}}
%\vskip4mm
\centering
\begin{tabular}{lp{5.5cm}}
\hline
Term &  Definition \\
\hline
Uptake N lim. & N in microbial uptake is less than 
constrained by other elements (${\Phi_B < 0}$).
\\
Microbial N lim. & Maximum immobilisation flux is not enough to satisfy
microbial N requirements (${-\Phi_B \ge
u_{\operatorname{imm,Pot}}}$).
\\
SOM N lim. & There is a net transfer from the inorganic pool to
the organic pools (${\Phi_B+\Phi_u+\Phi_{\operatorname{tvr}}<0}$).
\\
\hline
\end{tabular}
\end{table}
 



\subsection{ Enzyme allocation strategies} 
\label{sec:AllocStrategies}

Microbes allocate a proportion $\alpha$ of total enzyme investments, $a_e\,B$,
to the synthesis of enzymes targeting the N-rich $R$ substrate and proportion $1
- \alpha$ to the synthesis of enzymes targeting the N-poor, but better
degradable $L$ substrate (\ref{eq:alpha}). Three different strategies of
allocating investments among synthesis of alternative enzymes were explored in this study (Table
\ref{tab:AllocStrategies}).
\begin{equation} 
\label{eq:alpha}
\operatorname{syn}_{E_R} /
(\operatorname{syn}_{E_R} + \operatorname{syn}_{E_L}) \equiv \alpha
\end{equation}

\begin{table}[t]
\caption{Microbial enzyme allocation strategies \label{tab:AllocStrategies}}
\vskip4mm
\centering
\begin{tabular}{lp{6.5cm}}
\hline
Strategy &  Allocation is \\
\hline
Fixed & independent, constant \\
Match & adjusted to achieve balanced growth, i.e. $\beta_{DOM}$ matches
microbial demands  \\
Revenue & proportional to return per investments into enzymes \\
\hline
\end{tabular}
\end{table}


The \textbf{Fixed} strategy assumes that allocation is independent of,
and not changing with changes in substrate availability.
\begin{equation} 
\label{eq:allocFixed}
\alpha = \operatorname{const.} = 1/2
\end{equation}
This strategy corresponds to the models where decomposition rate is a function
of microbial biomass \citep{Wutzler08}.
 
The \textbf{Match} strategy assumes that microbes regulate enzyme synthesis in a
way that the decomposition products balance their stoichiometric demands
\citep{Moorhead12}. The partitioning coefficient $\alpha$ (\ref{eq:alpha}) is
derived by equating the C/N ratio of the sum of uptake fluxes after
other expenses, such as growth and maintenance respiration, to the
C/N ratio of microbial biomass, $\beta_B$.

\begin{equation}
\label{eq:allocMatchCN}
\beta_B = \frac{\epsilon (\operatorname{dec}_L + \operatorname{dec}_R - r_M)}{
\operatorname{dec}_L/\beta_L + \operatorname{dec}_R/\beta_R  - \Phi_M } 
\text{,}
\end{equation}
where $\operatorname{dec}_L$, and $\operatorname{dec}_R$ are depolymerisation
fluxes of the litter and residue pools, respectively, which both are a function
of $\alpha$ (Appendix \ref{app:fAlphaFix}). $r_M$ is maintenance respiration,
$\epsilon$ is the anabolic microbial efficiency accounting for growth
respiration (\ref{eq:synB}), $\beta_i$ are C/N ratios of the respective pools
$i$, and $\Phi_M$ is the net flux of N from living microbes to the mineral N
pool. Equation \ref{eq:allocMatchCN} for simplicity neglects the small inputs of
enzymes to DOM. Here, we assume that microbes use the maximal immobilisation of
inorganic N, $u_{\operatorname{imm,Pot}}$ (\ref{eq:uN}) to meet their
stoichiometric requirements with the Match strategy. Hence, the net N
imbalance flux is the difference between mineralization during uptake and the
immobilisation:
$\Phi_M = \Phi_u - u_{\operatorname{imm,Pot}}$. With microbial N-limitation,
(\ref{eq:allocMatchCN}) has no solution. In this case, the enzyme effort is
allocated entirely to the N-rich substrate ($\alpha=1$), and excess carbon uptake
is respired by overflow respiration.

If current enzyme pools $E_S$, are assumed to be in quasi steady-state with
their respective substrate $S \in \{L,R\}$ and microbial biomass, equation
\ref{eq:allocMatchCN} can be solved for partitioning coefficient, $\alpha$.

\begin{subequations}
\label{eq:allocMatch} 
\begin{align}
\alpha_M &= f_{\operatorname{{\alpha}Fix}}(L,\beta_L,R,\beta_R, E_L, E_R, r_M,
\Phi_M)
\\
\alpha &= \begin{cases}
  0,  & \text{if} \alpha_M \le 0 \\
  1,  & \text{if} \alpha_M \ge 1 \\
  \alpha_M, & \text{otherwise}
\end{cases}   
\end{align}
\end{subequations} 

\noindent where function $f_{\operatorname{{\alpha}Fix}}$ is given in Appendix
\ref{app:fAlphaFix} and the SYMPY script of its derivation is given with
supplementary material. The bound to one is necessary to handle the case of
microbial N-limitation. The bound to zero corresponds to the theoretical case
where the C-rich substrate may not suffice to cover microbial C demands relative
to N demands.

The \textbf{Revenue} strategy assumes that the microbial community adapts in a
way to ensure that the investment into enzyme synthesis is proportional to its
revenue, i.e.
the \remarkSZ{marginal}{(?)} return per investment regarding the currently
limiting element:
\begin{subequations}
\label{eq:allocRev}
\begin{align}
\alpha_C &= \frac{\operatorname{rev}_{RC}}{\operatorname{rev}_{LC} + \operatorname{rev}_{RC}} 
\\
\alpha_N &= \frac{\operatorname{rev}_{RN}}{\operatorname{rev}_{LN} + \operatorname{rev}_{RN}} 
\text{,} 
\end{align}
\end{subequations}
where $\operatorname{rev}_S$ is the revenue from given substrate $S \in \{L,R\}$
under C and N-limitation respectively. The return is the current decomposition
flux from the substrate degraded by the respective enzyme, and the investment is
assumed to be equal to enzyme turnover to keep current enzyme levels, $E_S^*$.
% : $a_E B = k_{NR} E_R^* + k_{NL} E_L^*$.
\begin{subequations}
\label{eq:allocRev2}
\begin{align}
\operatorname{rev}_{SC} &= \frac{\text{return}}{\text{investment}} 
= \frac{\operatorname{dec}_{S,Pot} \frac{E_S^*}{K_{M,S} + E_S^*}} {k_{NS}E_S^*} 
= \frac{\operatorname{dec}_{S,Pot}} {k_{NS}(K_{M,S} + E_S^*)} 
\\ 
\operatorname{rev}_{SN} &= \frac{\operatorname{dec}_{S,Pot}
\frac{E_S^*}{K_{M,S} + E_S^*} / \beta_S} {k_{NS} E_S^* / \beta_E} 
= \operatorname{rev}_{SC} \frac{\beta_E}{\beta_S}
% \\
% = \frac{\operatorname{dec}_{S,Pot}}{k_{NS} (K_{M,S} + E_S^*)} 
% \frac{\beta_E}{\beta_S}
\text{,} 
\end{align}
\end{subequations}
where $k_{NS}$ is rate of enzyme turnover, $K_{M,S}$ is enzyme's substrate
affinity, $\operatorname{dec}_{S,Pot}$ is
enzyme saturated decomposition flux (\ref{eq:dec}), and $\beta$ are C/N ratios
of the respective pools.

There are two resulting partitioning coefficients, $\alpha_C$ and $\alpha_N$
with C or N-limited microbial biomass, respectively. In order to avoid frequent large jumps
under near co-limitation, SEAM implements a smooth transition between these two
cases as a weighted average.

\begin{equation}
\label{eq:allocRev3}
%\begin{align}
\alpha = \frac{w_{\operatorname{CLim}} \alpha_C + w_{\operatorname{NLim}}
\alpha_N}{w_{\operatorname{CLim}}  + w_{\operatorname{NLim}} } 
\text{,} 
%\end{align}
\end{equation}
where $w$ is the strength of the limitation of the respective element,
specifically the ratio of required to available biomass synthesis fluxes
(\ref{eq:weightsLim}).


\subsection{Prototypical simulation scenarios} 
\label{sec:SimScen}

Several prototypical simulation \remarkSZ{scenarios}{(I've got problems with the
word scenarios here, because at least the first one isn't really a scenario but
a type of simulation experiment (short-term/limited scope). Would simply
removing scenarios also work?} \remTW{relace all occurences of scenario by
experiment?} (Table \ref{tab:SimScen}) were used to explore the consequences of
the different microbial enzyme allocation strategies (\ref{sec:AllocStrategies})
for the simulated SOM dynamics. They increase in complexity from a
soil incubation experiment to a long-term \chem{CO_2} manipulation treatment.
All scenarios used parameter values given in Table \ref{tab:modelParameters} unless
stated otherwise. For the prototypical scenarios, the inorganic N pool was
kept constant at $I=0.4~\unit{gN}$, while inorganic N feedback were considered
in Section \ref{sec:methodsPasture}.

\begin{table}[t]
\caption{Prototypical simulation scenarios \label{tab:SimScen}}
\vskip4mm
\centering
\begin{tabular}{lp{5.3cm}}
\tophline
Scenario & Explored issue\\
\middlehline
VarN-Incubation & Efficieny of using given fixed substrate levels that
vary by N content \\
Feedback-Steady & Possibility and size of steady state substrate pools\\
Priming & Increased substrate decomposition and mineralization after
a pulse addition of fresh litter\\
\chem{CO_2}-Fertilization & Contiued increase of litter C inputs but
constant litter N inputs
\\
\bottomhline
\end{tabular} 
\end{table}

The \textbf{VarN-Incubation} scenario explored to which efficiency substrates of
given a stoichiometry are used for microbial biomass growth under the different
enzyme allocation scenarios. A simplified model version was used in this
scenario, where all the inputs and feedback to the substrate pools ($L$ and $R$)
were neglected, and in which these pools were
kept constant ($dL/dt = dR/dt = 0$). This simplification led to a quasi
steady state of microbial biomass and enzyme levels for the given substrate supply. This
scenario mimics a short-term incubation experiment, where changes in litter and
residue pools are negligible small. The assumed boundary conditions for this
scenario were fixed substrate carbon of $L=100 \unit{gCm^2}$, $R=400
\unit{gCm^2}$. The C/N ratio of the residue pool was assumed constant at
$\beta_R=7$, whereas litter C/N ratio varied between 18 and 42 ($\beta_L =
[18,..,42]$).

The \remarkSZ{\textbf{Feedback-Steady}}{(this name does not work for me...)}
scenario explored the long-term trajectories of the entire system including
feedback to the substrate pools. Litter input was assumed
constant at a rate of $\operatorname{input}_L = 400~\unit{gCm^2yr^{-1}}$ with a
C/N ratio of $\beta_{\operatorname{input}_L} = 30$.

The \textbf{Priming} scenario explored the effect of rhizosphere priming, i.e
the input of fresh litter into a bulk subsoil. Specifically, the simulations
evaluated the fluxes after an addition of 50~\unit{gC} C and a respective amount
of N (C/N ratio $\beta_{\operatorname{input}_L} = 30$) on a soil that otherwise
received a litter input of only 30 \unit{gCm^2yr^{-1}} (and respective N with
$\beta_{\operatorname{input}_L} = 30$) for a decade. The assumption is made that
the litter input was very easily degradable litter, specifically with a maximum
turnover of $k_L = 10~\unit{day^{-1}}$.

The \textbf{\chem{CO_2}-Fertilization} scenario explored the effect of
continuous increased litter C input, which is expected with elevated atmospheric
\chem{CO_2}.
The simulations started from steady state corresponding to previous litter C
input, applied 20\% increased C inputs during years 10 to 60, and applied original carbon inputs
again during the next 50 years. The litter N inputs assumed to be constant over
time, implying an increase in the litter C/N ratio of 20\%.

\subsection{Calibration to a fertilised grassland site}
\label{sec:methodsPasture}

To test the capacity of the SEAM model to simulate the net carbon storage of a
grassland site including feedback of the inorganic N pool, we calibrated the
model using the revenue strategy to data of an intensive pasture. The intensive
pasture calibration was tackled only with the Revenue scenario, because the
Match strategy had already been proven invalid with the prototypical Feedback
scenario and the control case of the Fixed strategy did not allow for adaptation
of microbial enzyme allocation.

The model drivers and most of the parametrisation were taken
from \citet{Perveen14}. The site is a temperate permanent grassland located at
an altitude of 1040m a.s.l. in France (Laqueuille, 45\textdegree{38}'N,
2\textdegree{44}'E), receives an annual precipitation of 1200~mm and has an
annual mean temperature of  7 \textdegree{C}.

The N-balance of the fertilised grassland is characterised by high inorganic
N-inputs. A fraction of this N is sequestered in accumulating SOM, a fraction is
lost to leaching, while the remainder is exported with plant biomass harvest.
Plant uptake of inorganic N was computed as the sum of plant litter production
and plant biomass exports, keeping the plant N pool constant.
 
Model parameters were chosen corresponding to Table 1 in \citet{Perveen14}, and
initial litter and SOM pools were prescribed to observed values.
Three parameters were calibrated independently: the maximum decomposition rates
of substrate pools, $k_L$ and $k_R$, and the anabolic carbon-use efficiency,
$\epsilon$. Initial pools of microbial biomass and enzymes were set to the
long-term steady state after in order to prevent large transient initial
fluctuations in model pools. The calibration used the \textit{optim} function
from R \textit{stats} package \citep{R07} and minimised the differences between model
predictions and observations normalised by the standard deviation of the
observations. The calibration used observations of the litter OM, the inorganic
N, leaching, and rate of change of the total SOM pool ($\approx dR/dt$ if $L$ is
near quasi steady state). 

Subsequently, the calibrated parameters were used to generate
predictions for several scenarios of changed inputs to the system.

The R-code to generate the results and figures of this paper is available upon
request. 
 
\section{Results}

First, the results of several prototypical artificial simulation scenarios
clarify the general behaviour and features of the SEAM model. Next, results of a
parameter calibration demonstrate the model's ability to
simulate the observed C and N dynamics of an intensive pasture and explore
feedbacks with the dynamics of the inorganic N pool.

\subsection{Prototypical simulation scenarios}
\label{sec:ResultsProto}

Under the \textbf{VarN-Incubation} scenario, in which the substrate pools were fixed,
there were marked differences in the effect of allocation strategies on simulated biomass and the imbalance flux (Fig.
\ref{fig:VarNNoFeedback}).
 
\begin{figure}[t] \vspace*{2mm}
\begin{center}
\includegraphics[width=8.3cm]{fig/VarNNoFeedback}
\end{center}
\caption{
Match enzyme allocations strategy yielded highest resource efficiency, i.e.
lowest mineralization fluxes (N mineralization and C overflow respiration) at
steady state with the VarN-scenario.
Microbes with alternative strategies, however, competed better indicated by a higher
biomass. The patterns are caused by different adatption of resource
allocation ($alpha$) effecting carbon use efficiency (CUE) and C/N ratio of the
decomposition flux (cnDOM).
\label{fig:VarNNoFeedback}}
\end{figure}

The Match strategy allowed balanced growth, and yielded the highest substrate
efficiency and lowest mineralization fluxes among the enzyme allocation
strategies. Across a range of litter C/N ratios of 22 to 42 microbes did not
need stoichiometric imbalance fluxes, i.e. mineralization of excess N or
overflow respiration of excess C. However, it also yielded lowest biomass among
the strategies. When the litter contained enough N, microbes invested all
resources into litter degrading enzymes. Producing less biomass means to loose
competition with other microbes that are able to produce more biomass from given
substrates.

With the Revenue strategy, enzyme allocation also varied with litter N content,
but to a lesser extent.
With litter containing enough N (low C/N ratio), still about 5\% of the enzyme
synthesis C expenditures were allocated into R degrading enzymes. This resulted
in higher mineralization of excess N, but in turn allowed for a higher microbial
biomass.
With high C/N litter, investment into R-degrading enzymes increased to about
30\%, much less than with the Match strategy. Hence, the Revenue strategy
yielded higher overflow respiration associated with a low carbon-use
efficiency (CUE), because of a larger composition flux of the limiting element
N.

The Fixed strategy yielded higher N-mineralization due to stoichiometric
imbalance at low C/N ratios.
At high C/N ratios its constant allocation coefficient was intermediate between
the other strategies leading to intermediate values of all the other outputs.

\remTW{ Threshold elemental ratio slightly increased with higher litter C/N
ratio, as can be seen in the C/N ratio of the DOM in the match strategy. The
reason for this increase, here, was a lower proportion of uptake flux compared
to immobilization flux with lower biomass.
}

When the substrate pools were allowed to be refuelled by microbial and enzyme
turnover with the \textbf{Feedback-Steady} scenario, both Fixed and the Revenue
strategies caused substrate pools to approach a steady state.
However, the microbes with Match strategy solely degraded the stoichiometrically
better matching high-N residue pool, $R$. Hence, they declined together with the
R residues pool despite the large amount of N accumulating in the
stoichiometrically less favourable litter pool (Fig. \ref{fig:SimSteady}).
Because of the Match strategy was not able to simulate reasonable stocks when
including feedback to substrate pools in the model, it was omitted in the
following simulation scenarios.

\begin{figure}[t]
\vspace*{2mm}
\begin{center} 
\includegraphics[width=8.3cm]{fig/SimSteady} 
\end{center}
\caption{
Match strategy was not viable when considering feedback to
substrate pools with the SimSteady scenario. Microbes with
Match-strategy degraded a stoichiometrically matching but depleted R substrate
pool and their biomass, B, declined.
\label{fig:SimSteady}} 
\end{figure}

When the soil was amended with a pulse of litter with the \textbf{Priming
scenario}, a clear true priming effect, i.e. an increased decomposition of the
exisitng SOM, was simulated with the Fixed and Revenue strategy.
The priming effect occurred due a strong enhancement of residue decomposition
(Fig. \ref{fig:PrimingMinDec}). This
enhancement was stronger with the Revenue strategy than with the Fixed strategy,
primarily because of a higher simulated microbial biomass with the Revenue
strategy. In consequence, also the N-mineralization flux due to microbial
turnover was larger with the Revenue strategy (Fig. \ref{fig:PrimingMinDec}).
Note, that the time scale of the simulated priming effect of more than 100 days
was longer than observed in priming experiments.

\begin{figure}[t] \vspace*{2mm}
\begin{center}
\includegraphics[width=8.3cm]{fig/PrimingMinDec}
\end{center}
\caption{
Both depolymerisation of the residue substrate pool and N
mineralization were stimulated most strongly with the Revenue strategy after a
subsoil has been amended with a pulse of fresh litter (Priming scenario)
compared to a control with no amendment (straight horizontal lines).
\label{fig:PrimingMinDecDec}}
\end{figure}

When the continuous litter C input was assumed to be higher for 50 years with
the \textbf{\chem{CO_2}-fertilisation scenario}, enzyme allocation strategies
yielded marked difference in SOM stocks (Fig. \ref{fig:CO2Increase}) and
nutrient recycling (Fig. \ref{fig:CO2IncreaseImb}) between both strategies.
While litter stocks, $L$, increased with both scenarios following the increased
input, the residues stock, $R$, slightly increased with the Fixed strategy, but
declined strongly with the Revenue strategy. This was the consequence of an
increased mining of the $R$ pool with the Revenue strategy. Accordingly, N
mineralization was much stronger with the Revenue scenario during elevated
\chem{CO_2} period, with largest contribution from mineralization by microbial
turnover.

\remarkSZ{to me it seems important to mention that these effects were completely reversible over time! Also, what does the imbalance = 0 mean. total N sequestration or N liberalisation...? =)> important for the effect on vegetation}{}

\begin{figure}[t] \vspace*{2mm}
\begin{center}
\includegraphics[width=8.3cm]{fig/CO2Increase}
\end{center}
\caption{
Revenue strategy led to a depletion of residue substrate pool, R, that was
stronger than the increase in litter substrate pool, L, during increased carbon
litter inputs in years 10 to 60 with the \chem{CO_2}-Fertilization scenario.
\label{fig:CO2Increase}}

\end{figure}
\begin{figure}[t] \vspace*{2mm}
\begin{center}
\includegraphics[width=8.3cm]{fig/CO2IncreaseImb} 
\end{center}
\caption{
Mineralization of N associated with microbial turnover 
contributed most of the liberation of SOM-N with the Revenue strategy during
\chem{CO_2}-Fertilisation, which started at year 10. 
After the end of the fertilisation at year 60, microbies with the Revenue
strategy continued to more strongly immobilize N (negative flux $\Phi_b$).
\label{fig:CO2IncreaseImb}}
\end{figure}

\subsection{Intensive pasture simulation}
\label{sec:ResultsPasture}

The calibrated SEAM model successfully simulated the
observed C and N balance of the Laqueuille intensive pasture (Figure
\ref{fig:pastureFitMatch} left \remarkSZ{This Figure would be best split in too -
it's quite heavy on information}{}). In contrast to the prototypical simulation
experiments, here, the feedback of the inorganic N pool was included, the model
was driven and compared to observed values, and only the Revenue strategy has
been considered.

The observed continuous build-up of an organic N pool in the residue SOM was
driven by the system's positive N balance. Two pathways caused the model
behaviour in SEAM. First, inorganic N was taken up by the plant and returned to
the soil via organic N in litter. Second, microbial biomass immobilised
inorganic N due to its stoichiometric imbalance with the substrate. The
microbial biomass was N-limited when looking at the organic substrate only.
However, it was C-limited when taking into account immobilisation of inorganic
N.

\begin{figure}[t] \vspace*{2mm}
\begin{center}
\includegraphics[width=8.3cm]{fig/pastureFitMatch} 
\end{center}
\caption{
Calibrated SEAM predictions (lines)  matched observations from the
Laqueuille intensive pasture site (dots and errorbars) of litter pool, $L$, change of
SOM pools, $dR$, inorganic N, $I$, and N leaching rate.
\label{fig:pastureFitMatch}}
\end{figure}

\begin{figure}[t] \vspace*{2mm}
\begin{center}
\includegraphics[width=8.3cm]{fig/pastureFitInputScenarios} 
\end{center}
\caption{
Prescribed alteration of C and N inputs led to subsequent shifts in enzyme
allocation ($\alpha$) and affected development of soil pools.
Increased N substrate limitation, either due to elevated \chem{CO_2} or due to
decreasing inorganic N inputs, caused an increase in litter pool, $L$, and a
decrease in mineral N pool, $I$. If the substrate N limitation could not be
balanced by inorganic N input, then the change rate of the residue pool, $dR$,
decreased down to negative values, i.e. decreasing SOM pools, and a
positive N flux, $\Phi$, from SOM to the inorganic N pool.
\label{fig:pastureFitScen}}
\end{figure}   
   
Simulated alteration of C and N inputs to the system strongly affected the
internal SOM and nutrient cycling. Effects were shown by several
simulation scenarios that started from the calibrated state but applied a step change in
inputs of litter or inorganic N (Figure \ref{fig:pastureFitScen} right) as detailed
in following paragraphs.

Increased litter C input by 50\% together with an increased litter C/N
ratio by 25\% (elevated \chem{CO_2} scenario) caused a shift in enzyme
allocation towards enzymes degrading the N-rich residue pool and an increase of
the litter pool. The increased input also increased the mineral N demand of both the plant
to balance increased biomass synthesis and the microbial biomass with its higher
stoichiometric imbalance. The resulting decrease in mineral N also decreased
leaching losses, implying that the ecosystem available N was re-used more often, because of a higher
turnover flux of N in increased microbial biomass.

Decreased inorganic N inputs from 22.9
\unit{gm^{-2}yr^{-1}} down to 1
\unit{gm^{-2}yr^{-1}} together with a doubling of litter C/N
ratio caused a strong shift in enzyme allocation towards enzymes degrading the
N-rich residue SOM with similar consequences as with increased C input,
such as an increase in litter OM. However, in this scenario, the decreased N inputs caused
a depletion of the mineral N pool.
As a consequence, the microbial biomass could not use immobilisation to
balance substrate stoichiometry and became N-limited.
This caused overflow respiration and a decreasing trend in residue SOM.

Increased inorganic N inputs from 22.9 \unit{gm^{-2}yr^{-1}} up to 25.6
\unit{gm^{-2}yr^{-1}} together with a decrease of litter C/N by 25\% did not
much affect the system behaviour, because the soil system was already C-limited
at the start. The microbial biomass could only immobilise a small fraction of the additional N to build 
up new SOM. Instead, N accumulated in the inorganic pool with associated
increased losses to leaching.

\section{Discussion}
Microbial adaptation of enzyme synthesis to substrate availability and
stoichiometry benefited the community so that higher microbial biomass levels
could be sustained on a wider range of substrate stoichiometry.
The different prototypic simulation scenarios and the simulation of the
intensive pasture led to similar conclusions on the effects of adaptation of
enzyme allocation.

\subsection{Amounts of substrates matter}
The amount of substrate and the substrate stoichiometry are both important for
regulating enzyme allocation. The Match strategy failed to account for substrate
amount, assuming that microbes can achieve balanced growth under a wide range of
substrate stoichiometry \citep{Moorhead12, Ballantyne14}. This strategy yielded
lower microbial biomass both in the VarN-Incubation (Fig.
\ref{fig:VarNNoFeedback}) and in the Feedbacky scenarios (Fig.
\ref{fig:SimSteady}). Hence it would be outcompeted by other strategies.
Match-strategy microbes focused on degrading a stoichiometrically balanced, but
declining residues pool, leaving the large amount of N available in a
stoichiometrically less favourable litter pool untouched (Fig.
\ref{fig:SimSteady}). This finding implies that microbial enzyme allocation
strategies must account for substrate amounts.

\subsection{Community adaptation leads to a more efficient substrate usage}
The adaptive Revenue strategy consistently supported higher biomass and had
lower N mineralization fluxes at steady state compared to the the non-adaptive
Fixed strategy with the VarN-Incubation scenario (Fig.
\ref{fig:VarNNoFeedback}). Similar patterns appeared with the other scenarios
(Figs. \ref{fig:SimSteady} and \ref{fig:CO2IncreaseImb}). Such better substrate
usage is in line with results of individual based small-scale modelling
\citep{Kaiser14}.
The finding implies that N mineralization fluxes with imbalanced substrates may
be lower than inferred from previous modelling studies that did not account for
community adaptation.

\subsection{Comparison to observed changes in enzyme stoichiometry}

The SEAM model focuses on community adaptation of enzyme synthesis. It predicts
a change in the ratio of enzyme activities of enzymes degrading C-rich plant
litter versus enzymes degrading the N-rich residue SOM when changing inputs of
inorganic N to the soil.
While only low variation in stoichiometry of N-degrading versus C-degrading
enzymatic activity is observed across biomes \citep{Sinsabaugh09}, microcosm
studies detect short-term changes of enzyme activities with N fertilization
\citep{Kumar16}, but their observations differ between different kinds of
N-degrading enzymes.  Hence, the assumptions of the SEAM model are not validated
by observations yet. 

\subsection{SOM can store and release N}
\remarkSZ{don't like this heading - too unspecific}{}
Nitrogen was stored in residue SOM during periods of high N inputs and
released during periods of low N inputs relative to C inputs in simulations
(Fig.
\ref{fig:CO2Increase}). When there was excess litter carbon, microbial community
preferentially depolymerised, or mined, the N-rich residue pool, and thereby
made the N available for plants. When carbon inputs were low, microbes degraded the
residue pool to a lesser extent, but continued to build new residue via microbial turnover.
Hence, under low C conditions, the microbial kept N in the microbial system instead of releasing it through mineralisation. 

This 'bank' mechanism \citep[sensu][]{Perveen14} also worked when simulating the
intensive pasture (Fig. \ref{fig:pastureFitScen}). During simulations of high
inorganic N inputs, N was sequestered in SOM at a high rate. With decreased
inorganic N inputs, the sequestration rate decreased until it became negative,
that is the N in recalcitrant SOM was mined. In the long-term, i.e. centuries,
the inputs to the system have to balance the outputs of the system. In the
intensive pasture simulation, inorganic N pools and N leaching increased with
the increase of SOM with the SEAM model. The conservation or release of N by the
bank mechanism implies greater potential for ecosystems to avoid progressive N
limitation \citep{Norby10, Franklin14, Averill15}. This finding
potentially has consequences on feedbacks of global change, especially on the
projected C land uptake \citep{Friedlingstein14}.

\subsection{Priming effects recycle N}
Priming effects, i.e. the altered decomposition of SOM after soil amendments
\citep{Kuzyakov00}, are a potential mechanism to help plants stimulate N release
from the SOM for plant nutrition.
Priming effects and associated increased N mineralization were simulated for
both, the Fixed and Revenue strategies (Fig. \ref{fig:PrimingMinDec}). With
adaptive microbial enzyme allocation (Revenue strategy), \remarkSZ{changes
in plant litter inputs changed the preference for decomposing OM pools
of either low or high C/N ratios}{(not sure I follow here)} (Fig.
\ref{fig:CO2Increase}).
This in turn influenced the distribution of N in the ecosystem, and N
availability for plants (Fig.
\ref{fig:CO2IncreaseImb}). This active role of plant inputs has been
demonstrated in a soil incubation experiment \citep{Fontaine11} and has been
further conceptualised with the SYMPHONY model \citep{Perveen14}. Our results
are in line with these studies, although our explanation is on a more abstract
level (see Section \ref{sec:Holistic}). 

Mineralization during microbial turnover is important for nutrient recycling.
Without the additional mineralization mechanisms of uptake mineralization
\citep{Manzoni08} and turnover mineralization \citep{Clarholm85, Raynaud06} in
our simulation experiment, microbes shifted enzyme allocation to degrade the
residues pool, but the N was then sequestered in microbial biomass and not
mineralised to inorganic N. Hence, our simulation experiments reinforced the
need for representing soil heterogeneity and grazing for making N available for
plants under N limitation.

\subsection{Mismatch in time scale of priming effects}
The unrealistically long time-scale of the priming effect of several month in
SEAM (Fig. \ref{fig:PrimingMinDec}) resulted from both, the long turnover time of
enzymes, and the sustaining positive feedback between amounts of microbial
biomass and enzymes. It was in contrast with incubation studies that observe
priming effects within days or weeks that rapidly declined after the amendment
has been used up \citep{Blagodatskaya14}.
Priming timescale in SEAM was longer than the duration of the uptake pulse of
the $L$ amendment that only lasted a few days. It was controlled by simulated
lifetime of enzymes and enzyme turnover, which SEAM described as first order
kinetics with a turnover of about a week. Moreover, priming timescale was
prolonged by the positive feedback of increased microbial biomass producing more
enzymes that again fuelled micorbial biomass. 

\remarkSZ{I would present the following more as possible causes rather than as options to fix the model.}{}

One possible cause for a shorter priming time-scale is a different dynamics of
enzyme turnover. However, prescribing a shorter turnover time of enzymes would
require an unrealistically large effort of producing enzymes by microbial
biomass. More sophisticated models of different enzyme turnover kinetics
including stabilisation of a part of the enzymes on mineral surfaces
\citep{Burns13} may be able to resolve such contradictions. Testing this
hypothesis would require observations of the fraction of C uptake allocated to
enzyme synthesis and on age distribution of enzymes in the soil. \remarkSZ{which
are easy|impossible to do??}{} \remTW{I have no clever suggestion - But with
pulse labelling approaches might be expensive but feasable.}

An alternative cause for a shorter priming time-scale may be an important
control of enzyme activity that is not strongly coupled to microbial biomass
dynamics. Some enzymes such as peroxidase need to be fuelled by labile OM
themselves \citep{Rousk14} with no immediate relationship to microbial biomass
dynamics. This explanation, however, implies that enzyme activity and
decomposition of SOM become largely decoupled from enzyme synthesis and
microbial dynamics in the short-term. This option is contrary to the assumption
of most current models that simulate the priming effect. Such a fundamental
change of model assumption would affect most implications gained from
modelling studies that involving soil microbes.

Another cause for a shorter priming time-scale, is a diminished sustaining
positive feedback between enzymes and microbial biomass. Currently, grazing is
modelled as an implicit part of a first-order microbial turnover. With
increasing microbial biomass, grazers become more efficient \citep{Clarholm81}.
With implementing a time-lagged stronger increase in microbial turnover rate
with microbial biomass, biomass levels would decrease faster to pre-treatment
levels and help to shorten the time-scale of the priming effect. Testing this
hypothesis requires data on grazing during priming effects.

Overall, the mismatch in the time scale of priming between simulations and
observations hints to gaps in understanding short-term SOM turnover as
represented by SEAM. However, this model limitation does not impair the
simulated longer-term microbial community controls on SOM cycling both in the
prototypic simulation and at the pasture site. We argue therefore that the
simulated long-term patterns are robust, because they are more strongly
controlled by the proportions in enzyme synthesis than by the time scale of
priming effects. \remarkSZ{Argumentum e contrario, it is not a prerequisite of
large-scale models to capture small scale incubation experiment in order to
simulate the long-term largescale consequences of altered litter inputs on SOM
turnover and stabilisation.}{}

\subsection{A holistic view for upscaling}
\label{sec:Holistic}

The presented SEAM model takes a holistic view \citep{Panikov10} of microbial
community and their adaptations instead of explicitly describing microbial
diversity.
In this respect, it differs from the SYMPHONY model \citep{Perveen14} and
similar models \citep{Fontaine03}, which explicitly model several microbial
groups.
% specifically SOM builders, that grow solely on fresh low nutrient material
% (here $L$), and SOM builders that can use all the SOM.
However, the effective behaviour of the presented SEAM model is similar to these
models.
SEAM assumes that community composition is to a large extent driven by external
drivers. Specifically, SEAM describes an adaptive allocation of resources into
breakdown of different substrates by assuming that the community composition
adapts to changed substrate availability in a way to \remarkSZ{optimise}{(truly
optimal? by which criteria)} \remTW{maximise?, increase} microbe's revenue of
the currently limiting element.
While the mechanistic approach of the SYMPHONY model explicitly represents this
optimisation by shifts between microbial groups, the holistic approach
represents the effects of this optimisation at community level. The proposed
abstraction of microbial competition by the revenue strategy is a step forward
of better representing couplings of soil carbon and nutrient cycles in
large-scale ecosystem models, as it obviates the need to correctly parameterise
the underlying mechanisms with the more mechanistic approaches\remarkTW{, which
nota bene give more detailed understanding of the mechanism}{delete?}.
% \citep{Wutzler13}.

The holistic SEAM model yielded qualitatively similar predictions as the
mechanistic SYMPHONY model with simulating priming, the bank mechanism, and a
continuous SOM sequestration under high inorganic N inputs. SEAM differed from
SYMPHONY in the prediction of the inorganic N pool during low N inputs.
Specifically, SEAM predicted a decrease in this pool, while SYMPHONY predicted an
increase in this pool due to changed competition \citep{Perveen14}. The
difference is probably caused by different assumptions on how the DOM pool is
shared among groups of the microbial community and resulting different
competition conditions. In SEAM, decomposition products become mixed
in a shared DOM pool, while in the SYMPHONY model the decomposition products are
not shared between the microbial groups.
The truth at pore scale is in between, in that decomposition products are mainly
used by the group that is producing the extracellular enzymes, while a part of
the DOM diffuses also to other groups \citep{Kaiser14}. At larger scales, such
details cannot be measured or resolved. The difference in model prediction
implies that the rationality of the simplified model assumptions of a mixed
DOM pool can be qualitatively tested against observations. 

\subsection{Testable predictions of change of SOM C/N ratios}
The SEAM model can be used to predict long-term patterns of SOM cycling following
changes in substrate stoichiometry. Observations of such patterns provide evidence
for or against the modelling assumptions.
Specifically, SEAM predicted a change in proportions of the litter pool and the
SOM pool (Fig. \ref{fig:CO2Increase}). While these abstract pools are not
directly comparable to observations, a measurable consequence is the associated
change of total SOM C/N ratio at the time scale of turnover of the residue
pool. Specifically, SEAM predicted a decline in SOM stocks and an increase of
SOM C/N with FACE experiments at formerly C-limited systems over time scales of
several decades.\remarkSZ{somewhat calls for a reference -> see Duke Forest literature}{}
 
\subsection{Outlook} 
The biggest limitation of the SEAM model is its focus on a single process:
community adaptation of enzyme allocation. In order to focus, we had to ignore
several other important processes. One such process is the second microbial
community strategy of handling substrate stoichiometric imbalance,
the adaption of stoichiometry of microbial biomass. Although the potential of this biomass
adaptation is thought to be quite limited \citep{Mooshammer14}, it will need to be
tested whether these two strategies can be combined within a model.

Next, the optimality principle will be extended to also determine the proportion
of uptake that is allocated to enzyme synthesis. Presence of cheaters, i.e.
microbes that consume substrate but without producing enzymes, effectively lower
the community-level allocation to enzymes \citep{Kaiser14}. Community
development can be assumed to maximise biomass production. Such an assumption
can be used to compute the optimal community enzyme synthesis and allows
exploring effects on SOM cycling, such as more constrained carbon and nutrient
use efficiencies.

Moreover, SEAM will be simplified by assuming quasi-steady state of biomass or
enzyme pools \citep{Wutzler13}. These simplifications will lead to fewer
parameters and improved identifiably in model calibration to observations.
Together with implementing the influence of environmental factors such as
temperature and moisture \citep{Davidson12}, these changes will make SEAM more
suitable to be used as a component within larger scale land surface models.
 
\conclusions   
%% \conclusions[modified heading if necessary]
 
The SEAM model provides a holistic description of community adaptations. It
yields qualitatively similar predictions as microbial-group-explicit models
with the ability to represent priming effects, bank mechanism, and
a continuous SOM sequestration with high inorganic N inputs (Fig.
\ref{fig:pastureFitScen}). Hence, this study provides an important step for providing an abstract
description of microbial community effects and adaptations, with the long-term
goal of including the important mechanisms into earth system models.

Adapting the allocation of resources into the synthesis of different
enzymes can be an effective means of the microbial community to react to
changing substrate stoichiometry. Allocation adaptation strategies helped the simulated microbial biomass in SEAM
(Fig. \ref{fig:SEAMStruct}) to grow larger across a wider
range of substrate stoichiometry (Fig. \ref{fig:VarNNoFeedback}). Among the
tested strategies, the Revenue strategy, which takes account of the amount of substrate pools and their stoichiometry, was particularly successful.
These findings imply that models simulating soil carbon and nutrients
dynamics (Fig \ref{fig:PrimingMinDec}) need to account for
adaptations in carbon and nutrient strategies. Accounting for adaptations will
be especially important when studying the competition for nutrients between
soil microorganism and plants, because SOM can function as a storage to
sequester surplus nutrients and prevent them being lost from the system (Fig.
\ref{fig:CO2Increase} and \ref{fig:CO2IncreaseImb}).




