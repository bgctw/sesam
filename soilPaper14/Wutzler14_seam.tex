\introduction  %% \introduction[modified heading if necessary] 
The global element cycles of carbon and nitrogen are strongly linked and cannot
be understood without their interactions (XXCStorebyN). The links between nutrient
cycles are especially strong in dynamic of soil organic matter (SOM) because all
of the SOM has to be depolymerized and successively mineralized by a microbial
community with a rather strict homeostatic regulation. Faced with stoichiometric
imbalances between OM ressources and synthesized biomass, decomposers have to
lower their carbon use efficiency (CUE) or nutrient use efficiency (NUE) 
(XXSterner, Mooshammer). The regulation of CUE has shown to have large
consequences on prediction of carbon sequestration in SOM (XX Allison follows,
Manzoni). Regulation of NUE has consequences for nutrient recycling and loss of
nutrients from the ecosystem (XXMooshammer) and soil plant feedbacks
(XXRastetter 97). In addition, however, decomposers could also regulate
allocation into production of extracellular enzymes to preferentially
depolymerize food sources of different stoichiometry in order to match their
requirements (XXSterner, Mooshammer). This hypothesis was recently formalized by
XXMoorehead with the conceptual EEZY model. While this model showed the
principle feasibility of this strategy in short term, it did implement feedbacks
to ressource pools and therefore could no look at consequences for longer term
SOM cycling. The associated change in CUE and NUE, however, can have severe
consequences on the long term SOM cycling and our understanding of soil carbon
sequestration and plant- soil- atmosphere elemental feedbacks.

The aim of this study, therefore, is to further develop the conceptual model of
enzyme allocation including feedbacks of turnover to the ressource pools and
explore the different consequences of alternative enzyme allocation schemes on
long term SOM dynamics and nutrient recycling.

To this end we developed a conceptual model of SOM cycling that explicitly
represented several extracellular enzyme pools involved in the depolymerization
of specific ressources which differed in their elemental ratios. Several
variants were developed that differed by the strategy of enzyme allocation:
either fixed or flexible, and either based on stoichiometry only or on pool
sizes too. Simulations scenarios were devised to study effects on mineralization
fluxes and buildup of SOM. Specifically, we first studied under which ranges of
ressource stoichiometry decomposers with flexible allocation could use the
ressource more efficiently and decrease mineralization fluxes (including
overflow respiration as carbon mineralization). Second, we tested with which
strategies sustainable microbial communities could develop under constant litter
input, by simulating a steady state. Third, we were interested in different
responses of subsoil SOM mineralization and nutrient recycling after amendment
of fresh litter (priming). And fourth, we studied consequences for development
of SOM pools in a scenario of CO2 fertilization.

We show that different decomposers enzyme allocation strategies have large
consequences on long term SOM dynamics and nutrient recycling and suggest
further ways to study these strategies by both experimental, modelling and
combined studies.

\section{Methods}
\subsection{Soil Enzyme Allocation Model (SEAM)}
We used a conceptual structural dynamic model to explore consequences of enzyme
allocation strategies for SOM cycling: Soil Enzyme Allocation Model (SEAM). The
modelled system are the carbon and nitrogen pools in soil organic matter in a
representative elemental volume of soil. This could be soil in laboratory
incubation or a soil layer of about $1 \unit{m}^2$ outside roots. Different
pools of carbon and nitrogen are represented as state variables and the mass
fluxes between them govern their change by dynamic equations. We assume that the
single input to the system is by plant litter (both above ground and
rhizodeposition) and the sole outputs are respired carbon dioxide and
mineralized nitrogen. 

Key features are the representation of different SOM pools that differ by
elemental ratios and the explicit representation of specific enzymes that
depolymerize those SOM pools. The quality spectrum is modelled by two
classes: a nitrogen rich more easily degradable labile litter pool,
$\operatorname{L}$, and a pool mainly consisting of nitrogen rich microbial
residues, $\operatorname{R}$ (Fig. \ref{fig:SEAM}).

\begin{figure}[t] \vspace*{2mm}
\label{fig:SEAM}
\begin{center}
\includegraphics[width=8.3cm]{fig/seam2}
\end{center}
\caption{Model strucutre of SEAM: Two ressource pools ($L$ and
$R$) are depolymerized by respective enzymes
($E_L$ and $E_R$). The simple organic
compounds ($\operatorname{DOM}$) are transported by soil solution and taken up
by the microbial community  and used for synthesizing
new biomass (${B}$), new enzymes, and for catabolic respiration. 
Due to stoichiometric imbalance between $\operatorname{DOM}$ and ${B}$
there will be overflow respiration or mineralization of excess nitrogen to
inorganic N pool ($I$).
During microbial turnover an additional part of the biomass is mineralized and the
other part is added to the residue pool, $R$. Boxes represent pools, black
arrows mass fluxes, white arrows other controls, and disks represent
partitioning of fluxes. Solid lines represent fluxes of both C and N, while
dotted and dashed lines represent separate C or N fluxes respectively.}
\end{figure}

Appendix \ref{app:SEAM} reports the model equations and details. Here we
describe the most important assumptions.

C/N ratios, $\beta$ of biomass and enzymes are
fixed while those of the ressource pools may change over time due to
fluxes in and out of the pools. Imbalances in stoichiometry of uptake and demand
in microbial biomass are compensated by overflow respiration or nitrogen
mineralization.

Total enzyme allocation is a fixed fraction, $a_E$, of the microbial biomass,
$B$, per time. But microbial community can adjust their allocation to produce
new enzymes $E_R$ or $E_L$: $\alpha = E_R / (E_R + E_L)$. The different
strategies of enzyme allocation are explained in section
\ref{sec:AllocStrategies}.

The DOM pool is assumed to be in quasi steady state and all the sum of all
influxes to the DOM pool (decomposition + part of the enzyme turnover) is taken
up by microbial community.

If expenses for maintenance and enzyme production cannot be met, the biomass
starves and declines.
 
\subsection{ Enzyme allocation strategies 
\label{sec:AllocStrategies}}

Three different strategies of allocating investments among production of
alternative enzymes were explored in this study (Table
\ref{tab:AllocStrategies}). Microbes invest porportion $\alpha$ of total enzyme
investments into production of enzymes targeting the nitrogen rich $R$ ressource
and proportion $1 - \alpha$ into production of enzymes targeting the nitrogen
poor but better degradable $L$ ressource.

\begin{table}[t]
\caption{Simulation scenarios \label{tab:AllocStrategies}}
\vskip4mm
\centering
\begin{tabular}{ll}
\tophline
Strategy &  Allocation is \\
\middlehline
Fixed & independent, constant \\
Match & adjusted to achieve balanced growth \\
Revenue & proportional to return per investments into enzymes \\
\bottomhline
\end{tabular}
\end{table}


The \textbf{Fixed} strategy assumes that allocation is independent
and not changing with changes in substrat availability. 
\begin{equation}
\label{eq:allocFixed}
\alpha = \operatorname{const.}
\end{equation}
This strategy corresponds to the models where decomposition rate is a function
of microbial biomass \citep{Wutzler08}.
 
The \textbf{Match} strategy assumes that microbes balance their stoichiometric
demands \citep{Moorhead12}. The partitioning coefficient is derived by equating
the C/N ratio of the sum of decomposition and immobilization fluxes and the C/N
ratio of microbial biomass, $\beta_B$.
\begin{equation}
\label{eq:allocMatchCN}
\frac{\epsilon (\operatorname{dec}_L + \operatorname{dec}_R - r_M)}{
\operatorname{dec}_L/\beta_L + \operatorname{dec}_R/\beta_R  +
\phi_{\operatorname{imm}} } =
\beta_B
\text{,}
\end{equation}
where $\operatorname{dec}_L$, and $\operatorname{dec}_R$ are
depolymerization fluxes of the labile and residue ressources
respectively, $r_M$ is maintenance respiration, $\epsilon$ is the 
intrinsic microbial efficiency, and $\beta$ are C/N ratio of
the respective pools $i$, and $\phi_{\operatorname{imm}}$ is the immobilization
flux from mineral N pool to microbes.
When the current sum of enzyme levels, $E$ is assumed to be in quasi steady
state for given amounts of ressource and microbial biomass, then equation
\ref{eq:allocMatchCN} can be solved for allocation partitioning, $\alpha$.
\begin{subequations}
\label{eq:allocMatch}
\begin{align}
\alpha_M &= f_{\operatorname{{\alpha}Fix}}(L,\beta_L,R,\beta_R, E); \,
\alpha &= \begin{cases}
  0,  & \text{if} \alpha_M \le 0 \\
  1,  & \text{if} \alpha_M \ge 1 \\
  \alpha_M, & \text{otherwise}
\end{cases} \\  
\end{align}
\end{subequations}

The bounds to zero and one are necessary because the sole
allocation to decompose the carbon rich ressource may not suffice to
cover microbial carbon demands and solving eq. \ref{eq:allocMatchCN} for
$\alpha$ then leads to unreasonable solutions.
Function $f_{\operatorname{{\alpha}Fix}}$ is given in appendix
\ref{app:fAlphaFix} and the sympy script of its 
derivation is given with supplementory material. 

The \textbf{Revenue} strategy assumes that microbial community adatps in a way
so that investment into enzyme production is proportional to their revenue, i.e.
their return per investment regarding the currently limiting element.

The return is the current decomposition flux from respective ressource. The
investment needs to be equal to enzyme turnover to keep current enzyme levels,
$E_S^*$.
% : $a_E B = k_{NR} E_R^* + k_{NL} E_L^*$.
% 
\begin{subequations}
\label{eq:allocRev}
\begin{align}
\alpha_C &= \frac{\operatorname{rev}_{RC}}{\operatorname{rev}_{LC} + \operatorname{rev}_{RC}} \\
\alpha_N &= \frac{\operatorname{rev}_{RN}}{\operatorname{rev}_{LN} + \operatorname{rev}_{RN}} \\
\operatorname{rev}_{SC} &= \frac{\text{return}}{\text{investment}} 
= \frac{\operatorname{dec}_{S,Pot} \frac{E_S^*}{k_{m,S} + E_S^*}} {k_{NS}E_S^*} 
= \frac{\operatorname{dec}_{S,Pot}} {k_{NS}(k_{m,S} + E_S^*)} \\ 
\operatorname{rev}_{SN} &= \frac{\operatorname{dec}_{S,Pot}
\frac{E_S^*}{k_{m,S} + E_S^*} / \beta_S} {k_{NS} E_S^* / \beta_E} 
= \frac{\operatorname{dec}_{S,Pot}}{k_{NS} (k_{m,S} + E_S^*)}
\frac{\beta_E}{\beta_S}
\text{,} 
\end{align}
\end{subequations}
where $\operatorname{rev}_S$ is the revenue from given ressource $S$ ($S$ is either $L$ or $R$)
under carbon and nitrogen limitation respectively.
$k_{NS}$ is rate of enzyme turnover, $k_{m,S}$ is ressource affinity, $a_E$ is
total enzyme allocation coefficient, $\operatorname{dec}_{S,Pot} = k_S S$ is
enzyme saturated decomposition flux, and $\beta$ are C/N ratios of the
respective pools.

There are two partitionings, $\alpha_C$, with carbon limitated microbial
biomass, and $\alpha_N$, with nitrogen limited microbial biomass. In order to
avoid frequent large jumps under near co-limitation, SEAM implements a smooth
transition between these two cases as a weighted average with weights based
on ratio of required to available biomass synthesis fluxes (derived in
appendix \ref{app:SEAM}).

\begin{subequations}
\label{eq:allocRev}
\begin{align}
\alpha &= \frac{w_{\operatorname{CLim}} \alpha_C + w_{\operatorname{NLim}}
\alpha_N}{w_{\operatorname{CLim}}  + w_{\operatorname{NLim}} } 
\\
w_{\operatorname{CLim}} &= \left( \frac{\text{required}}{\text{available}}
\right)^\delta 
= \left( \frac{ N_{\operatorname{synBN}} \beta_B }{ C_{\operatorname{synBC}} }
\right)^\delta
\\
w_{\operatorname{NLim}} &= \left( \frac{ C_{\operatorname{synBC}} / \beta_B }{
N_{\operatorname{synBN}} } \right)^\delta
\text{,} 
\end{align}
\end{subequations}
where $\delta$ controls the steepness of the transition between the two
limitations. It was arbitrarily set to a value of 200.


\subsection{ Simulation scenarios 
\label{sec:SimScen}}

The consequences of different microbial enzyme allocation strategies for SOM
dynamics are explored by several simulation scenarios (Table \ref{tab:SimScen}). 
All scenarios used parameter values are given in Table \ref{tab:pars}, if not
stated otherwise below. We chose a very low minimum turnover time of the
residue SOM pool of 10 years in order to demonstrate and plot changes together
with the faster litter pool.

\begin{table}[t]
\caption{Simulation scenarios \label{tab:SimScen}}
\vskip4mm
\centering
\begin{tabular}{lp{5.3cm}}
\tophline
Scenario & Explored issue\\
\middlehline
VarN-Incubation & Efficieny of using given fixed ressource levels that
vary by N content \\
Feedback-Steady & Possibility and size of steady state ressource pools\\
Priming & Increased ressource decomposition and mineralization after
addition of fresh litter\\
CO2-Fertilization & SOM dynamics after increased carbon inputs\\
\bottomhline
\end{tabular}
\end{table}

The \textbf{VarN-Incubation} scenario explored how ressources of given
stoichiometry are used more or less efficiently with different enzyme allocation
scenarios. It used a simplified model where all the inputs and feedbacks to the
ressource pools ($L$ and $R$) were neglected and these pools were kept constant
($dL/dt = dR/dt = 0$). We reported the values and fluxes after Hence it
simulates the microbes and enzyme levels have grown into a quasi steady state
with the given ressource supply. Hence, it simulated the end of a short term
incubation. Can microbes adapt their allocation to the given resource
stoichiometries and grow in a balanced fashion where they do not need to discard
excess carbon or nitrogen?

Specifically, we fixed ressource carbon stocks to $L=400 gC/m^2$ and
$R=800gC/m^2$. C/N ratio of the residue pool was set to $\beta_R=6.8$, and the
we explored steady state conditions for a range of C/N ratios for the litter
($\beta_L = [18,..,42]$). 

The \textbf{Feedback-Steady} scenario explored the long term trajectories of
the entire system including feedbacks to the subsrate pools. Specifically, litter
input was set to $\operatorname{input}_L = 300 gC/m^2/yr$ with C/N ratio
$\beta_{\operatorname{input}_L} = 30$. Does the system reach a steady state for
given litter inputs? How much microbial biomass can be sustained?

The \textbf{Priming} scenario explored the effect of rhizosphere priming, i.e
the input of fresh carbon into a bulk subsoil. It looked at the fluxes during
the time the addition of $50 g$ of litter on a soil that otherwise
received a litter input of only $30 gC/m^2/yr$ for 20 years. Is
there an increased depolymerization of the residue pool? Are there changes in nitrogen
mineralization?

The \textbf{CO2-Fertilization} scenario explored the effect of continuous 
increased carbon litter input that is expected with elevated athmospheric CO2.
It started from steady state for a litter input of $\operatorname{input}_L = 300
gC/m^2/yr$ with C/N ratio $\beta_{\operatorname{input}_L} = 30$. During period
years 10 to 60, the carbon input, and hence also the C/N ratio was increased by
20\%, and returned to normal for the next 50 years. Do all the pools increase
due to the increased input? Can the microbes use the Nitrogen stored in the $R$
pool to make better use of the increased carbon availability?

\subsection{Calibration to a fertilized grassland site} 

To test the capability of the SEAM model to simulate the carbon sink of a
grassland site, we calibrated the model to data of an intensive pasture. The
model drivers and most of the parameterization was taken from the publication of
\citep{Perveen14}. The site is a temperate permanent grassland
located at 1040 m in France (Laqueuille, 45\textdegree{38}'N,
2\textdegree{44}'E). Mean annual precipitation and temperature are 1200 mm and 7
\textdegree{C}, respectively. 

Plant litter input and plant uptake of inorganic N were
computed assuming plant to be in steady state with the measured plant biomass and
biomass exports. The N balance of the fertilized grassland is characterized by
high inorganic N-inputs. Part of this N is sequestered in accumulating SOM, part
is leached and part is exported with plant biomass. Parameters were chosen
corresponding to Table 1 in \citep{Perveen14}. See
\ref{app:GrasslandCalibration} for details.

The three calibrated parameters were the maximum decomposition rates of
substrate pools, $k_L$ and $k_R$, and the microbial carbon use efficiency,
$\epsilon$. Initial pools were prescribed to observed values where available or
to the long-term state after a first optimization in order to prevent large
initial fluctuations. The values were selected that minimized differences
between model predictions and observations normalized by the standard deviation
of the observations. The minimum was found using function \textit{optim}
function in the R \textit{stats} package \citep{R07}.
 
\section{Results}

First, several prototypical artificial simulation scenarios clarify the
general behaviour and features of the SEAM model. Next, a calibration to
observations demonstrates the model's ability to simulate general C and N
dynamics of an intensive pasture.

\subsection{Prototypical simulation scenarios}

With the \textbf{VarN-Incubation} scenario, there were differences among
allocation strategies for the dependence of allocation $\alpha$ on the nitrogen
content of the litter ressource. These caused marked changes in
biomass and imbalance fluxes (Fig.
\ref{fig:VarNNoFeedback}).

\begin{figure}[t]
\vspace*{2mm}
\begin{center}
\includegraphics[width=8.3cm]{fig/VarNNoFeedback}
\end{center}
\caption{Allocation partitioning,
microbial biomass stocks, and imbalance N and C
fluxes at steady state with the VarN-Incubation scenario.
\label{fig:VarNNoFeedback}}
\end{figure}

The Match strategy allowed balanced growth and efficient resource usage, but
yieled less biomass than the other scenarios. When the litter contained enough
N, all resources were invested into litter degrading enzymes. The strategy
allowed a balanced microbial growth without stoichiometric imbalance fluxes,
i.e. mineralization of excess N or overflow respiration of excess C, across a
wide range of litter C/N ratios (20 to 42). While for favourable C/N ratios, it
yielded a higher biomass than the fixed strategy, biomass stocks were lower for
litter CN-ratios below 24. This was because we prescribed smaller stocks of
litter, $L$ than residue ressource, $R$.

With the Revenue strategy enzyme allocation also varied with litter N content.
However, under conditions where L contained enough N, about 30\% of the efforts
were invested into R degrading enzymes. This resulted in excretion of
excess N, but in turn allowed for a higher microbial biomass. Imbalance fluxes,
were smaller than with the Fixed strategy.

When including feedbacks to the ressource pools within the model with the
\textbf{SimSteady} scenario, ressource pools approached steady state with the
Fixed and the Revenue strategies, but there was no reasonable steady state for
the Match strategy (Fig. \ref{fig:SimSteady}).

\begin{figure}[t]
\vspace*{2mm}
\begin{center}
\includegraphics[width=8.3cm]{fig/SimSteady}
\end{center}
\caption{Development of ressource pools resulting from constant litter input and
turnover fluxes with the SimSteady scenario.
\label{fig:SimSteady}}
\end{figure}

With the match strategy microbes grew mostly on the
nitrogen rich residues pool and only partly on the litter pool. This led to an
accumulation of the litter and a degradation of the residue pool. Together with
diminishing return from residue pool, microbes starved and declined.
Because of the Match strategy was not able to simulate reasonable
stocks when including feedback to ressource pools in the model, it was omitted
from the remaining simulation scenarios.

After amending a starved subsoil with a pulse of litter in the
\textbf{Priming} scenario, a clear real priming effect was simulated. The
decomposition of the soil residues pool was strongly enhanced after the
amendment. The priming was stronger with the Revenue strategy than with the
Fixed strategy (Fig. \ref{fig:PrimingDec}). 

\begin{figure}[t]
\vspace*{2mm}
\begin{center}
\includegraphics[width=8.3cm]{fig/PrimingDec}
\end{center}
\caption{Increase of residue depolymerization flux after amending a starved
subsoil with a puls of fresh litter with the Priming scenario.
\label{fig:PrimingDec}}
\end{figure}

\begin{figure}[t]
\vspace*{2mm}
\begin{center}
\includegraphics[width=8.3cm]{fig/PrimingMin}
\end{center}
\caption{Increase of Nitrogen mineralization flux after amending a starved
subsoil with a puls of fresh litter with the Priming scenario.
\label{fig:PrimingMin}}
\end{figure}

This stronger priming was due to a higher microbial biomass with Revenue
strategy. Therefore also the N-mineralization flux due to microbial turnover
was larger with the Revenue strategy. After an initial phase of reduced
N-mineralization due to reduced imbalance, the N in mineral biomass was
mineralized by turnover (Fig. \ref{fig:PrimingMin}).

The most marked difference between enzyme allocation strategies was simulated
with the \textbf{CO2-Fertilization} scenario (Fig. \ref{fig:CO2Increase}).
\begin{figure}[t]
\vspace*{2mm}
\begin{center}
\includegraphics[width=8.3cm]{fig/CO2Increase}
\end{center}
\caption{Nitrogen-mining during increased of carbon litter inputs in years 10
to 60 with Revenue strategy under the CO2-Fertilization scenario.
\label{fig:CO2Increase}}

\end{figure}
\begin{figure}[t] \vspace*{2mm}
\begin{center}
\includegraphics[width=8.3cm]{fig/CO2IncreaseImb}
\end{center}
\caption{Decreased overflow and N mineralization due to imbalance with Revenue 
strategy under the CO2-Fertilization scenario.
\label{fig:CO2IncreaseImb}}
\end{figure}
With both, Fixed and Revenue strategies, the litter stock,$L$ increased. The
residues stock $R$, however, slighly increased with the Fixed strategy but
declined with the Revenue strategy. With the Revenue strategy, in addition, the
overflow respiration during increased carbon inputs and the N mineralization
in the period of decreased carbon input was lower compared to the Fixed
strategy (Fig. \ref{fig:CO2IncreaseImb}).

\subsection{Intensive pasture simulation}

The SEAM model successfully simulated the C and N balance of the Laqueuille
intensive pasture (Figure \ref{fig:pastureFit} left), although only 3 parameters
have been calibrated.
The inputs of inorganic N exceeded the losses inorganic N. They fuel a buildup
of a pool of organic N in the recalcitrant SOM, by two pathways. First,
inorganic N is taken up by the plant and supplied via organic N in litter, and
second, microbial biomass immobilizes inorganic N due to stoichiometric
imbalance of substrate. The microbial biomass was substrate N-limited. However,
it actually was C-limited because immobilization of inorganic nitrogen balanced
the substrate limitation. 

\begin{figure}[t] \vspace*{2mm}
\label{fig:pastureFit}
\begin{center}
\includegraphics[width=8.3cm]{fig/pastureFit} 
\end{center}
\caption{Calibrated Seam model predictions match observed values of the
Laqueuille intensive pasture (dots and errorbars left). On altered in C and N
inputs, the predictions show a shift in enzyme allocation ($\alpha$) and SOM
turnover (right).
Increased N substrate limitation either due to elevated CO2 or due to
decreasing inorganic N inputs cause an increase in labile pool, $L$, and a
decrease in mineral N pool, $I$. If the substrate N limitation cannot be balanced by mineral N
input then change rate of the recalcitrant pool, $dR$, decreases down to
negative values, i.e. losses.
}
\end{figure}   

Altered C and N inputs to the system strongly affected the internal SOM and
nutrient cycling (Figure \ref{fig:pastureFit} right). 

\textit{Increased litter C input} by 50\% together with an increased litter
C/N ratio by 25\% caused a shift in enzyme allocation towards enzymes degrading the
N-richer recalcitrant litter. It subsequently caused an increase in the
labile litter pool and a increased demand for mineral N, both for 
the plant to balance increased biomass production and higher stoichiometric
imbalance of microbial biomass. The resulting decrease in mineral N also
decreased leaching losses. Increased N efficiency within the system was
additionally favoured by a faster recycling of N due to increased
microbial activity. 

\textit{Decreased inorganic N inputs} from 22.9
${\textrm{gm}^{-2}\textrm{yr}^{-1}}$ down to 1
${\textrm{gm}^{-2}\textrm{yr}^{-1}}$ together with an doubling of litter C/N
ratio caused a strong shift in enzyme allocation towards enzymes degrading the
N-richer recalcitrant SOM with similar consequences as with increased C input,
such as an increase in labile SOM. However, here, the decreased N inputs caused
a depletion of the mineral N pool.
As a consequence, the microbial biomass could not use immobilization to
balance substrate stoichiometry and became N-limited.
This caused overflow respiration and a decreasing trend in recalcitrant SOM.

\textit{Increased inorganic N inputs} from 22.9 $gm^{-2}yr^{-1}$ up to 25.6
$gm^{-2}yr^{-1}$ together with decrase of litter C/N by 25\% caused a strong
shift in enzyme allocation towards enzymes degrading the labile litter SOM.
Since, the soil system was C-limited before, it could only use a small part of
the additional N for builing up SOM.
Instead, the increased N inputs caused the inorganic carbon pool to increase
with associated increase in leaching losses.

\section{Discussion}
\subsection{Importance of ressource amounts}
The amounts of ressource pools together with ressource stoichiometry are
important for regulating enzyme allocation. The match strategy does not account
for ressource amounts despite their importance. It assumes that microbes can
achieve balanced growth under a wide range of resource stoichiometry,
specifically when describing incubation experiments \citep{Moorhead12,
Ballentine14}. However, microbial biomass stocks declined with the Match
strategy both in the VarN-Incubation scenario (Fig. \ref{fig:VarNNoFeedback})
and in the SimSteady scenarios (Fig. \ref{fig:SimSteady}) of this study. Such
biomass decline was caused by focusing enzyme production on pools that matched
stoichiometry demands but were small compared to the other pools. Microbes
invested enzymes into depolymerizing a small but stoichiometrically balanced
residues pool depite the huge amount of N available in a large but
stoichiometric imbalanced litter pool (Fig.
\ref{fig:SimSteady}). This strategy led to the further accumulation of the
litter pool and the exhaustion of both the residues pool and the microbial
biomass. We argue that such unreasonable behaviour is evidence against the Match
strategy and conclude that an improved strategy must account for ressource
amounts.

\subsection{Increased Biomass and decreased imbalance fluxes}
Revenue strategy consistently supported higher biomass and had lower N
mineralization fluxes compared to the the Fixed strategy with the
VarN-Incubation scenario (Fig. \ref{fig:VarNNoFeedback}). Similar patterns
appeared with the other scenarios (Fig. \ref{fig:SimSteady}
\ref{fig:CO2IncreaseImb}). With the SimSteady scenarios the revenue strategy
yielded higher biomass (XX vs XX) at steady state and lower XX fluxes (XX vs
XX).
Microbes with the Revenue allocation strategy use the available ressource more
efficiently and sustain more biomass for the same litter inputs and
environmental conditions. They use the food more for productive respiration and
turnover that sustains a food chain rather than for imbalance waste fluxes. 
This implies that with increased flexibility within the microbial
community mineralization fluxes with imbalanced ressources may be lower than
inferred from previous modelling studies, as also shown by an indiviual based
model of XX \citep{Kaiser14}.
  

\subsection{Bank mechanism}
The Revenue strategy led to a redistribution of ressource types during high
and low carbon inputs with the  CO2-Fertilization scenario (Fig.
\ref{fig:CO2Increase}). When there was excess litter carbon, microbes
preferentially depolymerized the nitrogen rich residue pool in order to make use of available
carbon. When decreasing carbon inputs, excess nitrogen was used to
build the residue pool again instead of releasing N by mineralization.

The bank mechanism also worked when simulating the intensive
pasture (Fig. \ref{fig:pastureFit}). During simulations of high inorganic
N inputs, N was sequestered in SOM at a high rate. With decreasing inorganic N
inputs, the sequestration rate decreased until it became negative, the N of SOM
was mined.
 
Ultimately, on the long term, the inputs to the system have to balance the
outputs of the system. Hence there are constraints on possible alteration of
total respiration and N mineralization. However, changes can be buffered by a 
depositing or mining a reserve of excess nitrogen in the residue pool.
The limit of this bank capacity is the difference in steady states between
inputs levels. In our simulations this was about twice the annual N
litter input. 

This implies an important point in interpreting the feedbacks and consequences
of global change.  First, the additional storage i XX

\subsection{Nutrient recycling by priming}
The priming effect can help plants to stimulate nutrient release.
Priming effects were simulated for both strategies (Fig. \ref{fig:PrimingDec})
associated with an increased nitrogen mineralization flux.
Hence, the difference in plant ressource inputs to the soil could influence the
recycling of nutrients from SOM.
Nutrient recycling is the other side of the bank mechanism. If carbon is
sequestered in SOM, then also nutrients are sequestered and become unavailable
for plants.
With the revenue strategy, however, plant inputs could influence the
partitioning of SOM between a low C/N SOM pool and a high C/N pool (Fig.
\ref{fig:CO2Increase}) and, hence, also influence the distribution of nitrogen
in the ecosystem and nutrient availability for plants. This active role of
plants has already demonstrated in a soil incubation experiment
\citep{Fontaine11} and has been further conceptualized with the SYMPHONY model
\citep{Perveen14}. Our results are in line with these studies, although our
explanation is on a more abstract level (section \ref{sec:Holistic}). It implies
greater potential for plants to avoid progressive nitrogen limitaiton
\citep{Norby10, Franklin14, Averill15}.

\subsection{How long are enzymes active?}
The turnover time of soil extracellular enzymes governs the timescale of priming
effects. SEAM simulations showed a rather long time scale of the priming effect
\ref{fig:PrimingDec}) of several month, despite the $L$ ressource puls was used
up within about 50 days. This is in contrast with incubation studies that
observe priming effects within days or weeks that rapidely declines after the
ressource has been used up e.g. \citep{Blagodatskaya14}.
Im this study the timescale was set by simulated enzyme turnover,
which we modelled using first order kinetics with a turnover time of one month
\citep{Burns13}.

One option to model the observed faster timescales, the turnover time of enzymes
could be decreased to a few days.
This change, however, would require an immense effort of producting enzymes by
microbial biomass. More sophisticated models of enzyme turnover
including stabilization of a part on mineral surfaces \citep{Burns13},
hopefully will be able to resolve such contradictions.

An alternative explanations is that enzymes such as peroxidases may need to be
fueled by labile organic matter themselves \citep{Rousk14} with no immediate
relationship to microbial biomass dynamics.
This explanation implies, however, that enzyme activity and decomposition of SOM
becomes decoupled from enzyme production and microbial dynamics to a large
extent in the short term.
This is contrary to the assumption of most models of the priming effect. While
such an explanation would greatly complicate comparison between model and data,
we argue that the model generated patterns of enzyme allocation and SOM turnover
are still robust in the long term and longer term activity is still determined
by enzyme production.
 
\subsection{Holisctic view for upscaling
\label{sec:Holistic}} 

The presented SEAM conceptual model does not explicitely describe microbial
diversity but instead takes a holistic view \citep{Panikov10}. In this respect
it differs from the SYMPHONY model \citep{Perveen14} and similar conceptual
models \citep{Fontaine03}, that explicitely modelled several microbial groups,
specifically SOM builders that grow solely on fresh low nutrient material (here
$L$), and SOM builders that can use both ressources (here $L$ and $S$). The
effective behaviour of the presented SEAM model, however, is similar. It assumes
that community composition is driven to a large extent by external drivers
outside the community. The SEAM model describes a modified allocation of
resources into breakdown of different ressources with changed resource
availability by assuming that the community composition adapts in a way so that
it will optimize its revenue.
While the merismatic approach of the SYMPHONY model explicitly represents this
optimization by shifts between microbial groups, the holistic approach
represents the effects of this optimization.
While the meristic approach gives more detailed understanding of the mechanism,
we hypothize, that the holistic approach is more suitable for simplification
\citep{Wutzler13} and application in larger-scale models, such as land surface
components of earth system models.
This implies that that the proposed abstraction of microbial competition by the
revenue strategy is a step forward of better representing couplings of soil
carbon and nutrient cycles in earth system models and a step forward to improved
predictions of long term carbon sequestration.

The holistic SEAM model mostly had qualitatively similar predictions as the
merismatic SYMPHONY model with simulating priming, the bank mechanism, and a
continuous SOM sequestration under high inorganic N inputs. It differed from
SYMPHONY in the prediction of the inorganic N pool during low N inputs. It
predicted a decrease in this pool while SYMPHONY predicted an increase in this
pool due to changed competition \citep{Perveen14}. The difference We think that
this difference is caused by different assumptions on how the DOM pool is shared
among groups of the microbial community and resulting different competition
conditions. In the SEAM model, decomposition products become mixed in a shared
DOM pool, while in the SYMPHONY model the decomposition products are not shared
between the microbial groups.
The truth probably in between, in that decomposition products are mainly used by
the group that is producing the extracellular enzymes, while a part of the DOM
diffuses also to other groups.
The difference in prediction implies that the reasonability of the simplified
model assumptions can be qualitatively tested agains observations.

\subsection{Testable predictions: Change of SOM C/N ratios}
The SEAM model can be used to predict long term patterns of SOM cycling
after changes in resource stoichiometry. Observations of such pattern 
provide evidence for or against the modelling assumptions.
Specifically, we predicted a change in proportions of the litter pool and the
SOM pool. While these abstract pools are not directly comparable to observations
the overall effect is a change in total SOM C/N ratio. The C/N pool is
predicted to increase quite fast in all scnearios, but only with
the Renveue strategy it should it should still increase after a few decades of
FACE and unchanged N inputs/outputs (Fig. \ref{fig:CO2Increase}).
 
\subsection{Outlook}
The biggest limitation of the SEAM model is its focus on a single process:
enzyme allocation. In order to focus we had to ignore the other microbial
strategis of coping with resource stoichiometric imbalance by adapting their
biomass ratio \citep{XXRastetter}. Although the potential of this biomass
adaptation is thougth to be quite limited \citep{Mooshammer14}, it will be
tested whether these two strategies can be combined within a model.

A second route forward is to try of further simplify SEAM by assuming
quasi-steady state of biomass or 

Explore N-immolization, short term


Assumption of allocating a fixed proportion of uptake to enzyme production
dropped an optimality constraint on community development. Presence of cheaters,
i.e. microbes that consume ressource but without producing enzymes, effectively
lower the community-level allocation to enzymes. Community development
can be assumed to maximise biomass production. Such an assumption can be
used to compute the optimal share of cheaters and hence the optimal
proportion of utake that is allocated to enzyme production. 
 
Laqueuille

Differences to Moorhead12 approach

Both litter pools of two elements

maintenance respiration

allocation part of biomass instead of uptake, decoupled

prediction changing C/N ratios

holistic view on microbial community, upscaling


The Match strategy adjusts enzyme production solely based on stoichiometry but
ignores the sizes of the ressource pools. Hence

Duration of priming - litter one year

Other limiting nutrients incorporated into minimum function.


\conclusions  
%% \conclusions[modified heading if necessary]
 
The allocation of resources into the production of different enzymes is an
important means of the microbial community to adapt to changing resource
stoichiometry. Allocation adaptation strategies helped microbial biomass to grow
larger biomass across a wider range of resource stoichiometry (Fig.
\ref{fig:VarNNoFeedback}). The strategy was particularly successful that took
into account both, the amount of resource and their stoichiometry.
The findings imply that models that want to simulate ecosystem balances and
mineralization fluxes of carbon and nutrients (Figs. \ref{fig:PrimingDec} and
\ref{fig:PrimingMin}) must take into account such adaption strategies.
Accounting for such strategies will be especially important, when simulating
competition of soil microorganism and plants for organic and inorganic nutrient
pools, as SOM may function as a storage to sequester surplus elements and
prevent them from leaving the system (Fig. \ref{fig:CO2Increase} and
\ref{fig:CO2IncreaseImb}).

The SEAM model provides a holistic description of community adaptations. It
yields qualitatively similar predictions such as microbial group explicit models
with the ability to represent priming effects, bank mechanism, and
a continuous sequestration with high inorganic N inputs (Fig. \ref{fig:pastureFit}).

Hence, this study provides an important step on providing an abstract
description of microbial community effects and adaptations, with the long-term
goal of including the important mechanisms into abstract earth system models.
Global modelers could already use the presented model to describe resource
stoichiometry effects. However, we hope to further abstract the model for larger
scales by a steady state assumption of the enzyme pools or the microbial
biomass, and by implementing optimality principles to determine the
overall amount of resource invested into enzyme production.


