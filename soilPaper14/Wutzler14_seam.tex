\introduction  %% \introduction[modified heading if necessary] 
The global element cycles of carbon and nitrogen are strongly linked and cannot
be understood without their interactions (XXCStorebyN). The links between nutrient
cycles are especially strong in dynamic of soil organic matter (SOM) because all
of the SOM has to be depolymerized and successively mineralized by a microbial
community with a rather strict homeostatic regulation. Faced with stoichiometric
imbalances between OM substrates and synthesized biomass, decomposers have to
lower their carbon use efficiency (CUE) or nutrient use efficiency (NUE)
(XXSterner, Mooshammer). The regulation of CUE has shown to have large
consequences on prediction of carbon sequestration in SOM (XX Allison follows,
Manzoni). Regulation of NUE has consequences for nutrient recycling and loss of
nutrients from the ecosystem (XXMooshammer) and soil plant feedbacks
(XXRastetter 97). In addition, however, decomposers could also regulate
allocation into production of extracellular enzymes to preferentially
depolymerize food sources of different stoichiometry in order to match their
requirements (XXSterner, Mooshammer). This hypothesis was recently formalized by
XXMoorehead with the conceptual EEZY model. While this model showed the
principle feasibility of this strategy in short term, it did implement feedbacks
to substrate pools and therefore could no look at consequences for longer term
SOM cycling. The associated change in CUE and NUE, however, can have severe
consequences on the long term SOM cycling and our understanding of soil carbon
sequestration and plant- soil- atmosphere elemental feedbacks.

The aim of this study, therefore, is to further develop the conceptual model of
enzyme allocation including feedbacks of turnover to the substrate pools and
explore the different consequences of alternative enzyme allocation schemes on
long term SOM dynamics and nutrient recycling.

To this end we developed a conceptual model of SOM cycling that explicitly
represented several extracellular enzyme pools involved in the depolymerization
of specific substrates which differed in their elemental ratios. Several
variants were developed that differed by the strategy of enzyme allocation:
either fixed or flexible, and either based on stoichiometry only or on pool
sizes too. Simulations scenarios were devised to study effects on mineralization
fluxes and buildup of SOM. Specifically, we first studied under which ranges of
substrate stoichiometry decomposers with flexible allocation could use the
substrate more efficiently and decrease mineralization fluxes (including
overflow respiration as carbon mineralization). Second, we tested with which
strategies sustainable microbial communities could develop under constant litter
input, by simulating a steady state. Third, we were interested in different
responses of subsoil SOM mineralization and nutrient recycling after amendment
of fresh litter (priming). And fourth, we studied consequences for development
of SOM pools in a scenario of CO2 fertilization.

We show that different decomposers enzyme allocation strategies have large
consequences on long term SOM dynamics and nutrient recycling and suggest
further ways to study these strategies by both experimental, modelling and
combined studies.

\section{Methods}
\subsection{Soil Enzyme Allocation Model (SEAM)}
We used a conceptual structural dynamic model to explore consequences of enzyme
allocation strategies for SOM cycling: Soil Enzyme Allocation Model (SEAM). The
modelled system are the carbon and nitrogen pools in soil organic matter in a
representative elemental volume of soil. This could be soil in laboratory
incubation or a soil layer of about $1 \unit{m}^2$ outside roots. Different
pools of carbon and nitrogen are represented as state variables and the mass
fluxes between them govern their change by dynamic equations. We assume that the
single input to the system is by plant litter (both above ground and
rhizodeposition) and the sole outputs are respired carbon dioxide and
mineralized nitrogen. 

Key features are the representation of different SOM pools that differ by
elemental ratios and the explicit representation of specific enzymes that
depolymerize those SOM pools. The quality spectrum is modelled by two
classes: a nitrogen rich more easily degradable labile litter pool,
$\operatorname{L}$, and a pool mainly consisting of nitrogen rich microbial
residues, $\operatorname{R}$ (Fig. \ref{fig:SEAM}).

\begin{figure}[t] \vspace*{2mm}
\label{fig:SEAM}
\begin{center}
\includegraphics[width=8.3cm]{fig/eezy5}
\end{center}
\caption{Model strucutre of SEAM: Two substrate pools ($L$ and
$R$) are depolymerized by respective enzymes
($E_L$ and $E_R$). The simple organic
compounds ($\operatorname{DOM}$) are transported by soil solution and taken up
by the microbial community  and used for synthesizing
new biomass ($\operatorname{B}$), new enzymes, and for catabolic respiration. 
Due to stoichiometric imbalance between $\operatorname{DOM}$ and $\operatorname{B}$
there will be overflow respiration or mineralization of excess nitrogen. During
microbial turnover an additional part of the biomass is mineralized and the
other part is added to the residue pool, $R$. Boxes represent pools, black
arrows mass fluxes, white arrows othr controls, and disks represent
partitioning of fluxes.}
\end{figure}

Appendix \ref{app:SEAM} reports the model equations and details. Here we
describe the most important assumptions.

C/N ratios, $\beta$ of biomass and enzymes are
fixed while those of the substrate pools may change over time due to
fluxes in and out of the pools. Imbalances in stoichiometry of uptake and demand
in microbial biomass are compensated by overflow respiration or nitrogen
mineralization.

Total enzyme allocation is a fixed fraction,
$a_E$, of the microbial biomass, $B$, per time. But microbial community can
adjust their allocation to produce new enzymes $E_R$ or $E_L$: $\alpha = E_R / (E_R + E_L)$. The different strategies
of enzyme allocation are explained in section \ref{sec:AllocStrategies}.

The DOM pool is assumed to be in quasi steady state and all the sum of all
influxes to the DOM pool (decomposition + part of the enzyme turnover) is taken
up by microbial community.

If expenses for maintenance and enzyme production cannot be met, the biomass
starves and declines.
 
\subsection{ Enzyme allocation strategies 
\label{sec:AllocStrategies}}

Three different strategies of allocating investments among production of
alternative enzymes were explored in this study (Table
\ref{tab:AllocStrategies}). Microbes invest porportion $\alpha$ of total enzyme
investments into production of enzymes targeting the nitrogen rich $R$ substrate
and proportion $1 - \alpha$ into production of enzymes targeting the nitrogen
poor but betteer degradable $L$ substrate.

\begin{table}[t]
\caption{Simulation scenarios \label{tab:AllocStrategies}}
\vskip4mm
\centering
\begin{tabular}{ll}
\tophline
Strategy &  Allocation is \\
\middlehline
Fixed & independent, constant \\
Match & adjusted to achieve balanced growth \\
Revenue & proportional to return per investments into enzymes \\
\bottomhline
\end{tabular}
\end{table}


The \textbf{Fixed} strategy assumes that allocation is independent
and not changing with changes in substrat availability. 
\begin{equation}
\label{eq:allocFixed}
\alpha = \operatorname{const.}
\end{equation}
This strategy corresponds to the models where decomposition rate is a function
of microbial biomass \citep{Wutzler08}.
 
The \textbf{Match} strategy assumes that microbes balance their stoichiometric
demands \citep{Moorhead12}. The partitioning coefficient is derived by equating
the C/N ratio of the sum of decomposition fluxes and the C/N ratio of microbial
biomass, $\beta_B$.
\begin{equation}
\label{eq:allocMatchCN}
\frac{\epsilon (\operatorname{dec}_L + \operatorname{dec}_R - r_M)}{
\operatorname{dec}_L/\beta_L + \operatorname{dec}_R/\beta_R } = \beta_B
\text{,}
\end{equation}
where $\operatorname{dec}_L$, and $\operatorname{dec}_R$ are
depolymerization fluxes of the labile and residue substrates
respectively, $r_M$ is maintenance respiration, $\epsilon$ is the 
intrinsic microbial efficiency, and $\beta$ are C/N ratio of
the respective pools i.
When enzyme levels are assumed to be in steady state for
given amounts of substrate and microbial biomass, then equation
\ref{eq:allocMatchCN} can be solved for allocation paritioning, $\alpha$.
\begin{equation}
\label{eq:allocMatch}
\alpha_M &= f_{\operatorname{{\alpha}Fix}}(L,\beta_L,R,\beta_R,B); \,
\alpha &= \begin{cases}
  0,  & \text{if} \alpha_M \le 0 \\
  1,  & \text{if} \alpha_M \ge 1 \\
  \alpha_M, & \text{otherwise}
\end{cases} \\  
\end{equation}
The bounds to zero and one are necessary because the sole
allocation to decompose the carbon rich substrate may not suffice to
cover microbial carbon demands and solving eq. \ref{eq:allocMatchCN} for
$\alpha$ then leads to unreasonable solutions.
Function $f_{\operatorname{{\alpha}Fix}}$ is given in appendix
\ref{app:fAlphaFix} and the sympy script of its 
derivation is given with supplementory material. 

The \textbf{Revenue} strategy assumes that microbes invest into enzyme
production proportional to their revenue, i.e. their return per investment
regarding the currently limiting element.

The return is the current decomposition flux from respective substrate. The
investment needs to be equal to enzyme turnover to keep current enzyme levels.
% : $a_E B = k_{NR} E_R^* + k_{NL} E_L^*$.
% 
\begin{subequations}
\label{eq:allocRev}
\begin{align}
\alpha_C &= \frac{\operatorname{rev}_{RC}}{\operatorname{rev}_{LC} + \operatorname{rev}_{RC}} \\
\alpha_N &= \frac{\operatorname{rev}_{RN}}{\operatorname{rev}_{LN} + \operatorname{rev}_{RN}} \\
\operatorname{rev}_{SC} &= \frac{\text{return}}{\text{investment}} 
= \frac{\operatorname{dec}_{S,Pot} \frac{E_S^*}{k_{m,S} + E_S^*}} {k_{NS}E_S^*} 
= \frac{\operatorname{dec}_{S,Pot}} {k_{NS}(k_{m,S} + E_S^*)} \\ 
\operatorname{rev}_{SN} &= \frac{\operatorname{dec}_{S,Pot}
\frac{E_S^*}{k_{m,S} + E_S^*} / \beta_S} {k_{NS} E_S^* / \beta_E} 
= \frac{\operatorname{dec}_{S,Pot}}{k_{NS} (k_{m,S} + E_S^*)}
\frac{\beta_E}{\beta_S}
\text{,} 
\end{align}
\end{subequations}
where $\operatorname{rev}_S$ is the revenue from given substrate $S$ ($S$ is either $L$ or $R$)
under carbon and nitrogen limitation respectively.
$k_{NS}$ is rate of enzyme turnover, $k_{m,S}$ is substrate affinity, $a_E$ is
total enzyme allocation coefficient, $\operatorname{dec}_{S,Pot} = k_S S$ is
enzyme saturated decomposition flux, and $\beta$ are C/N ratios of the
respective pools.

There are two partitionings, $\alpha_C$, with carbon limitated microbial
biomass, and $\alpha_N$, with nitrogen limited microbial biomass. In order to
avoid frequent large jumps under near co-limitation, SEAM implements a smooth
transition between these two cases as a weighted average with weights based
on ratio of required to available biomass synthesis fluxes (derived in
appendix \ref{app:SEAM}).

\begin{subequations}
\label{eq:allocRev}
\begin{align}
\alpha &= \frac{w_{\operatorname{CLim}} \alpha_C + w_{\operatorname{NLim}}
\alpha_N}{w_{\operatorname{CLim}}  + w_{\operatorname{NLim}} } \\
w_\operatorname{CLim} &= \left( \frac{\text{required}}{\text{available}}
\right)^\delta 
= \left( \frac{ N_{\operatorname{synBN}} \beta_B }{ C_{\operatorname{synBC}} }
\right)^\delta
\\
w_\operatorname{NLim} &= \left( \frac{ C_{\operatorname{synBC}} / \beta_B }{
N_{\operatorname{synBN}} } \right)^\delta
\text{,} 
\end{align}
\end{subequations}
where $\delta$ controls the steepness of the transition between the two
limitations. It was arbitrarily set to a value of 200.


\subsection{ Simulation scenarios 
\label{sec:SimScen}}

The consequences of different microbial enzyme allocation strategies for SOM
dynamics are explored by several simulation scenarios (Table \ref{tab:SimScen}). 
All scenarios used parameter values are given in Table \ref{tab:pars}, if not
stated otherwise below. We chose a very low minimum turnover time of the
residue SOM pool of 10 years in order to demonstrate and plot changes together
with the faster litter pool.

\begin{table}[t]
\caption{Simulation scenarios \label{tab:SimScen}}
\vskip4mm
\centering
\begin{tabular}{lp{5.3cm}}
\tophline
Scenario & Explored issue\\
\middlehline
VarN-Incubation & Efficieny of using given fixed substrate levels that
vary by N content \\
Feedback-Steady & Possibility and size of steady state substrate pools\\
Priming & Increased substrate decomposition and mineralization after
addition of fresh litter\\
CO2-Fertilization & SOM dynamics after increased carbon inputs\\
\bottomhline
\end{tabular}
\end{table}

The \textbf{VarN-Incubation} scenario explored how substrates of given
stoichiometry are used more or less efficiently with different enzyme allocation
scenarios. It used a simplified model where all the inputs and feedbacks to the
substrate pools ($L$ and $R$) were neglected and these pools were kept constant
($dL/dt = dR/dt = 0$). We reported the values and fluxes after Hence it
simulates the microbes and enzyme levels have grown into a quasi steady state
with the given substrate supply. Hence, it simulated the end of a short term
incubation. Can microbes adapt their allocation to the given resource
stoichiometries and grow in a balanced fashion where they do not need to discard
excess carbon or nitrogen?

Specifically, we fixed substrate carbon stocks to $L=400 gC/m^2$ and
$R=800gC/m^2$. C/N ratio of the residue pool was set to $\beta_R=6.8$, and the
we explored steady state conditions for a range of C/N ratios for the litter
($\beta_L = [18,..,42]$). 

The \textbf{Feedback-Steady} scenario explored the long term trajectories of
the entire system including feedbacks to the subsrate pools. Specifically, litter
input was set to $\operatorname{input}_L = 300 gC/m^2/yr$ with C/N ratio
$\beta_{\operatorname{input}_L} = 30$. Does the system reach a steady state for
given litter inputs? How much microbial biomass can be sustained?

The \textbf{Priming} scenario explored the effect of rhizosphere priming, i.e
the input of fresh carbon into a bulk subsoil. It looked at the fluxes during
the time the addition of $50 g$ of litter on a soil that otherwise
received a litter input of only $30 gC/m^2/yr$ for 20 years. Is
there an increased depolymerization of the residue pool? Are there changes in nitrogen
mineralization?

The \textbf{CO2-Fertilization} scenario explored the effect of continuous 
increased carbon litter input that is expected with elevated athmospheric CO2.
It started from steady state for a litter input of $\operatorname{input}_L = 300
gC/m^2/yr$ with C/N ratio $\beta_{\operatorname{input}_L} = 30$. During period
years 10 to 60, the carbon input, and hence also the C/N ratio was increased by
20\%, and returned to normal for the next 50 years. Do all the pools increase
due to the increased input? Can the microbes use the Nitrogen stored in the $R$
pool to make better use of the increased carbon availability?

\section{Results}

With the \textbf{VarN-Incubation} scenario, there were differences among
allocation strategies for the dependence of allocation $\alpha$ on the nitrogen
content of the litter substrate. These caused marked changes in
biomass and imbalance fluxes (Fig.
\ref{fig:VarNNoFeedback}).

\begin{figure}[t]
\vspace*{2mm}
\begin{center}
\includegraphics[width=8.3cm]{fig/VarNNoFeedback}
\end{center}
\caption{Allocation partitioning,
microbial biomass stocks, and imbalance N and C
fluxes at steady state with the VarN-Incubation scenario.
\label{fig:VarNNoFeedback}}
\end{figure}

With the Match strategy, when the litter contained enough N, all resources were
invested into litter degrading enzymes. This allowed a balanced growth without
stoichiometric imbalance fluxes, i.e. mineralization of excess N or overflow
respiration of excess C, across a wide range of litter C/N
ratios (20 to 42).
It also allowed higher microbial biomass stocks for litter C/N ratio below
35 than with the Fixed stratey. However, although the
substrate was used more efficiently, biomass stocks were lower for litter
CN-ratios below 24. This was because we prescribed smaller stocks of litter,
$L$ than residue substrate, $R$.

With the Revenue strategy enzyme allocation also varied with litter N content.
However, under conditions where L contained enough N, about 30\% of the efforts
were invested into R degrading enzymes. This resulted in excretion of
excess N, but in turn allowed for a higher microbial biomass. Imbalance fluxes,
were smaller than with the Fixed strategy.

When including feedbacks to the substrate pools within the model with the
\textbf{SimSteady} scenario, substrate pools approached steady state with the
Fixed and the Revenue strategies, but there was no reasonable steady state for
the Match strategy (Fig. \ref{fig:SimSteady}).

\begin{figure}[t]
\vspace*{2mm}
\begin{center}
\includegraphics[width=8.3cm]{fig/SimSteady}
\end{center}
\caption{Development of substrate pools resulting from constant litter input and
turnover fluxes with the SimSteady scenario.
\label{fig:SimSteady}}
\end{figure}

With the match strategy microbes grew mostly on the
nitrogen rich residues pool and only partly on the litter pool. This led to an
accumulation of the litter and a degradation of the residue pool. Together with
diminishing return from residue pool, microbes starved and declined.
Because of the Match strategy was not able to simulate reasonable
stocks when including feedback to substrate pools in the model, it was omitted
from the remaining simulation scenarios.

After amending a starved subsoil with a pulse of litter in the
\textbf{Priming} scenario, a clear real priming effect was simulated. The
decomposition of the soil residues pool was strongly enhanced after the
amendment. The priming was stronger with the Revenue strategy than with the
Fixed strategy (Fig. \ref{fig:PrimingDec}). 

\begin{figure}[t]
\vspace*{2mm}
\begin{center}
\includegraphics[width=8.3cm]{fig/PrimingDec}
\end{center}
\caption{Increase of residue depolymerization flux after amending a starved
subsoil with a puls of fresh litter with the Priming scenario.
\label{fig:PrimingDec}}
\end{figure}

\begin{figure}[t]
\vspace*{2mm}
\begin{center}
\includegraphics[width=8.3cm]{fig/PrimingMin}
\end{center}
\caption{Increase of residue depolymerization flux after amending a starved
subsoil with a puls of fresh litter with the Priming scenario.
\label{fig:PrimingMin}}
\end{figure}

This stronger priming was due to a higher microbial biomass with Revenue
strategy. Therefore also the N-mineralization flux due to microbial turnover
was larger with the Revenue strategy. After an initial phase of reduced
N-mineralization due to reduced imbalance, the N in mineral biomass was
mineralized by turnover (Fig. \ref{fig:PrimingMin}).

The most marked difference between enzyme allocation strategies was simulated
with the \textbf{CO2-Fertilization} scenario (Fig. \ref{fig:CO2Increase}).
\begin{figure}[t]
\vspace*{2mm}
\begin{center}
\includegraphics[width=8.3cm]{fig/CO2Increase}
\end{center}
\caption{Nitrogen-mining during increased of carbon litter inputs in years 10
to 60 with Revenue strategy under the CO2-Fertilization scenario.
\label{fig:CO2Increase}}
\end{figure}
\begin{figure}[t] \vspace*{2mm}
\begin{center}
\includegraphics[width=8.3cm]{fig/CO2IncreaseImb}
\end{center}
\caption{Decreased overflow and N mineralization due to imbalance with Revenue 
strategy under the CO2-Fertilization scenario.
\label{fig:CO2IncreaseImb}}
\end{figure}
With both, Fixed and Revenue strategies, the litter stock,$L$ increased. The
residues stock $R$, however, slighly increased with the Fixed strategy but
declined with the Revenue strategy. With the Revenue strategy, in addition, the
overflow respiration during increased carbon inputs and the N mineralization
in the period of decreased carbon input was lower compared to the Fixed
strategy (Fig. \ref{fig:CO2IncreaseImb}).


\section{Discussion}
\subsection{Importance of pool sizes}
The Match strategy proposed that microbes can achieve balanced growth under a
wide range of resource stoichiometry when modelling incubation experiemnts
\citp{Moorhead12, Ballentine14}.
However, both in the VarN-Incubation scenario (Fig.
\ref{fig:VarNNoFeedback}) and in the SimSteady scenarios (Fig.
\ref{fig:SimSteady}), microbial biomass stocks with the Match strategy declined
when focusing enzyme production on pools that matched stoichiometry demands but
were small compared to the other pools.
Microbes that have a huge amount of N available in large but stoichiometric
imbalanced litter pool (Fig. \ref{fig:SimSteady}) invest enzymes into
depolymerizing a small but stoichiometrically balanced residues pool. This leads
to the further accumulation of the litter pool and the exhaustion of both the
residues pool and the microbial biomass. This clearly implies, that considering
the pool sizes is important, and that the Match strategy can not be applied.

\subsection{Increased Biomass and decreased imbalance fluxes}
With the VarN-Incubation scenario (Fig.
\ref{fig:VarNNoFeedback}) the Revenue strategy supported consistently higher
biomass than the Fixed strategy, and consistently had lower N mineralization
fluxes due to stoichiometric imbalance at high N contents, i.e low C/N ratios.
With the SimSteady scenarios (Fig. \ref{fig:SimSteady}) also had higher biomass
(XX vs XX) at steady state and lower XX fluxes (XX vs XX). Same pattern appeared
in the CO2-Fertilization scenario (Fig. \ref{fig:CO2IncreaseImb}).

The Revenue allocation strategy made better use of the available
food and sustained more biomass for the same litter inputs and environmental
conditions. Hence, it used the food more for productive respiration and turnover
that sustains a food chain instead of imbalance waste fluxes.  

This implies XX

\subsection{Bank mechanism}
The Revenue strategy led to a redistribution of substrate types during high
and low carbon inputs with the  CO2-Fertilization scenario (Fig.
\ref{fig:CO2Increase}). When there was excess litter carbon, microbes
preferentially depolymerized the nitrogen rich residue pool in order to make use of available
carbon. Otherwise, excess nitrogen was used to build the residue pool again
instead of reasing it as N mineralization.
 
Ultimately, on the long term, the inputs to the system have to balance the
outputs of the system. Hence there are constraints on possible alteration of
total respiration and N mineralization. However, until there is a big reserve of
depositing excess nitrogen or mining for required nitrogen in the residue pool.
The limits of this bank capapcity is the difference in steady states between
inputs levels. In our simulations this was about twice the annual N
litter input. Because the real turnover time and stock of residue pool higher
than in our simulations, also the capacity to deposit N is higher in reality.

This implies an important point in interpreting the feedbacks and consequences
of global change.  First, the additional storage i

\subsection{Nutrient recycling by priming}
There was a simulated priming effect for both strategies (Fig.
\ref{fig:PrimingDec}) associated with an increased nitrogen mineralization flux
with both fixed and revenue strategy. Hence, the plants could influence the
recycling of Nutrients from SOM. Moreover, nutrient recycling is the other side
of the bank mechanism. If carbon is sequestered in SOM, then nutrients become
less for plants. With the revenue strategy, however, plants could influence the
partitioning of SOM between a low C/N SOM pool and a high C/N pool (Fig.
\ref{fig:CO2Increase}) and, hence, also influence the distribution of nitrogen
with in the ecosystem and availability for plants. This active role of plants
has already demonstrated in a soil incubation experiment \ref{Fontaine11} and
has been further conceptualized with the SYMPHONY model \ref{Perveen14}. Our
results are in line with these studies, although our explanation is on a more abstract
level (section \ref{sec:sec:Holistic}). The implications of 

\subsection{How long are enzymes active?}
SEAM simulations showed a rather long time scale of the priming effect
\ref{fig:PrimingDec}) of several month, despite the $L$ substrate puls was used
up within about 50 days. This is in contrast with incubation studies that
observe priming effects within days or weeks that rapidely declines after the
substrate has been used up e.g. \citep{Blagodatskaya14}.
This is caused by simulated enzyme turnover, which we modelled using first
order kinetics with a turnover time of one month. \citep{XXBurnsEnzymeReview}

In order to model the observed decline of priming after the substrate has
been used up the turnover time of enzymes can be decreased to a few days. This,
however, would require an immense effort of producting enzymes by microbial
biomass. More sophisticated models of enzyme turnover might however be able to
resolve this contradiction.

An alternative explanations is that enzymes may need to be fueled by fresh
organic matter themselves \citep{XXPrimingWorkshop Denmark}. This would imply,
however, that enzyme activity and decomposition of SOM becomes decoupled from
enzyme production and microbial dynamics to a large extent in the short term.
This is contrary to the assumption of most models of the priming effect. While
those assumptions would greatly complicate comparison between model and data, we
argue the patterns of enzyme allocation and SOM turnover is still robust in the
long term where enzyme production still determines integrated enzyme activity.
 
\subsection{Holisctic view for upscaling
\label{sec:Holistic}} 

The presented SEAM conceptual model does not explicitely
describe microbial diversity. In this respect it differs from SYMPHONY model
\citep{Perveen14} and similar conceptual models \citep{Fontaine03}, that
explicitely modelled several microbial groups, specifically SOM builders that grow solely on
fresh low nutrient material (here $L$), and SOM builders that can use both
substrates (here $L$ and $S$). The effective model behaviour, however, is
similar. The SEAM model applies a holistic view on the microbial community
\citep{Panikov10}, where the resulting ecosystem functions that depend on the
competition of microbial groups within the community is largely determined by
the external drivers outside the community. The SEAM model describes a modified
allocation of resources into breakdown of different substrates with changed
resource availability by assuming that the community will optimize its revenue.
The SYMPHONY model applies a meristic paradigm and explicitly models this
optimization by shifts between microbial groups. While the meristic approach
gives more detailed understanding of the mechanism, we hypothize, that the
holistic approach is more suitable for simplification \citep{Wutzler13} and
application in larger-scale models, such as land surface components of earth
system models.
This implies that that the proposed abstraction of microbial competition by the
revenue strategy is a step forward of better representing couplings of soil
carbon and nutrient cycles in earth system models and a step forward to improved
predictions of long term carbon sequestration.

\subsection{Testable predictions: Change of SOM C/N ratios}
So far the SEAM model has not be compared to observations or experimental data.
This is, indeed, a difficult task because the model is designed with the purpose
to model the long-term patterns of change substrate stoichiometry and
availability and abstract from many processes that are relevant at time scales
up to years and decades.
However, the model can be used to predict long term patterns of SOM cycling
after change in resource stoichiometry and these can serve provide evidence for
or against the modelling assumptions \ref{XXEintstein}.
Specifically, we predicted a change in proportions of the litter pool and the
SOM pool. While these abstract pools are not directly comparable to observations
the overall effect is a change in total SOM C/N ratio. The C/N pool is
predicted to increase quite fast in all scnearios, but only with
the Renveue strategy it should it should still increase after a few decades of
FACE and unchanged N inputs/outputs (Fig. \ref{fig:CO2Increase}).
 
\subsection{Outlook}
The biggest limitation of the SEAM model is its focus on a single
process: enzyme allocation. In order to focus we had to ignore the other
microbial strategy of copomg with resource stoichiometric imbalance by adapting their
biomass ratio \citep{XXRastetter}. Although the potential of this biomass
adaptation is thougth to be quite limited \citep{Mooshammer14}, it will be
tested whether these two strategies can be combined within a model.  

A second route forward is to try of further simplify SEAM by assuming
quasi-steady state of biomass or 

Explore N-immolization, short term





 




 
 
 

Differences to Moorhead12 approach

Both litter pools of two elements

maintenance respiration

allocation part of biomass instead of uptake, decoupled

prediction changing C/N ratios

holistic view on microbial community, upscaling


The Match strategy adjusts enzyme production solely based on stoichiometry but
ignores the sizes of the substrate pools. Hence

Duration of priming - litter one year

Other limiting nutrients incorporated into minimum function.


\conclusions  %% \conclusions[modified heading if necessary]
TEXT

